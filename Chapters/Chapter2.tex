% Chapter Template
\chapter{!!NOT DONE!! The LHC and the CMS Experiment, !!NOT DONE!!} \label{Chapter2} 


\section{Introduction}
This thesis' chapter outlines the main characteristics of the Large
Hadron Collider (LHC) and of the Compact Muon Solenoid (CMS)
detector. 
``An accelerator propels charged particles, such as protons or electrons, at high speeds, close to the speed of light. They are then smashed either onto a target or against other particles circulating in the opposite direction. By studying these collisions, physicists are able to probe the world of the infinitely small.

When the particles are sufficiently energetic, a phenomenon that
defies the imagination happens: the energy of the collision is
transformed into matter in the form of new particles, the most massive
of which existed in the early Universe. This phenomenon is described
by Einstein’s famous equation E=mc2, according to which matter is a
concentrated form of energy, and the two are interchangeable.

The Large Hadron Collider is the most powerful accelerator in the world. It boosts particles, such as protons, which form all the matter we know. Accelerated to a speed close to that of light, they collide with other protons. These collisions produce massive particles, such as the Higgs boson or the top quark. By measuring their properties, scientists increase our understanding of matter and of the origins of the Universe. These massive particles only last in the blink of an eye, and cannot be observed directly. Almost immediately they transform (or decay) into lighter particles, which in turn also decay. The particles emerging from the successive links in this decay chain are identified in the layers of the detector.''

%----------------------------------------------------------------------------------------
%	SECTION 1
%----------------------------------------------------------------------------------------
\section{The Large Hadron Collider}

The LHC~\cite{Brning2004LHCDR} is a circular particle accelerator
located at the CERN laboratories in Geneva operating since 10
September 2008. It is
designed to accelerate hadrons (like protons, Lead-ions, Xenon-ions) and to
operate at the centre-of-mass energy of 14\TeV.
The circular ring is installed in a tunnel of a 27 kilometers where
the Large Electron Positron collider~\cite{Lep:designReport} was
previously located.\\
A graphic representation of CERN accelerator
complex is shown in Figure~\ref{fig:cern} where the particle
accelerations begins at the LINAC, foregoing booster, PS and SPS, in
order. The LHC consists of accelerating components as well as
superconducting magnets to focus the hadrons, keep them on the right
trajectory and squeeze them tight together right before the
collision point. 

\begin{figure}[h]
\centering
\includegraphics[width=0.75\textwidth]{Figures/c2/Cern-accelerator-complex.png}
\vspace*{3mm}
\caption{The LHC is the largest ring (top) in a complex chain of particle accelerators. The smaller machines are used in a chain to help boost the particles to their final energies and provide beams to a whole set of smaller experiments, which also aim to uncover the mysteries of the Universe~\cite{Mobs:2197559}}
\label{fig:cern}
\end{figure}

The protons are accelerated in opposite directions in two distinct
accelerator tubes which cross at four interactions points where the
protons are made to collide. At each of the interaction points along the ring,
four experiments are located with the aim to reconstruct the
sub-atomic particles which are made at the moment of a high energy
collision. The protons are grouped in bunches which are
accelerated in steps using the full accelerator chain consisting in a linear
accelerator, boosters, synchrotrons and, at the end, they are injected
into the LHC with an energy of 540\GeV where a system of
superconducting magnets further accelerate them up to 13\TeV. Every
25 ns collisions between proton bunches occur meaning 40 million
bunch crossing per second. At full regime during data taking time
period about 2800 bunches travel in the LHC rings and each bunch is
made of up to $1.1\times10^{11}$ protons.

The four experiments located at the four interaction points are
ALICE~\cite{alice_2008} (A Large Ion Collider Experiment),
ATLAS~\cite{atlas_2008} (A Toroidal LHC ApparatuS),
CMS~\cite{cms_2008} and LHCb~\cite{lhcb_2008} (Large Hadron Collider
beauty), refer to Figure~\ref{fig:cern}. ALICE experiment is designed
to study the presence and the properties of the hypothetical
quark-gluon plasma formed during heavy ions collisions, LHCb is
designed to be very sensitive in analyzing the properties of the B
mesons. The last two, ATLAS and CMS are general purpose detectors
designed to investigate a vast range of physics scenarios starting
from the search and discovery of the Higgs boson to extra dimensions
and dark matter. \\

The accelerator-dependent features and parameters which are important for a
physics analysis are the instantaneous and integrated luminosity, the
number, in the same bunch crossing, of simultaneous collisions and the
center-of-mass energy of the proton-proton collisions.

The instantaneous luminosity is defined as a time dependent
parameter, $d\mathcal{L}/dt$, which correlates the number of collisions
($N$) in a certain amount of time ($t$) and the cross section of a
given process through the relation:
\begin{equation}
\label{eq:instalumi}
\frac{dN}{dt} \: = \: \frac{d\mathcal{L}}{dt}\sigma
\end{equation}

The unit of the instantaneous luminosity is $b^{-1}s^{-1}$, where 1
barn $= 10^{-24} \ cm^2$ and it depends on the number of bunches in
the proton beam, on the number of protons per bunch and on the beam
optics. \\
The integrated luminosity is the integral of the instantaneous
luminosity over time, and relates the cross section of a
given process to the number of events $N$ of that process:
\begin{equation}
\label{eq:intelumi}
\mathcal{L} \:=\: \int \frac{d\mathcal{L}}{dt} dt \: = \: \frac{N}{\sigma}
\end{equation}


\begin{figure}[h]
  \noindent
  \makebox[\textwidth]{
  \includegraphics[width=.50\textwidth]{Figures/c2/int_lumi_cumulative_pp_2.pdf}
  \includegraphics[width=.50\textwidth]{Figures/c2/pileup_allYears_run2.pdf}}\\
  \makebox[\textwidth]{\includegraphics[clip,trim=0.2cm 0.2cm 0.2cm 0.2cm, width=.60\textwidth]{Figures/c2/Lumi.png}}
  \caption{Top-left: integrated luminosity collected by the CMS
    experiment; top-right: distribution of the average number of
    interactions per crossing (pileup) for pp collisions in 2015
    (purple), 2016 (yellow), 2017 (azure), 2018 (periwinkle), and
    full Run2 (gray),~\cite{webpage_lumi}. Bottom: scheduled and
    projected integrated and instantaneous luminosity at the LHC~\cite{webpage_lhc}.}
  \label{fig:lumi}
\end{figure}

The LHC was designed to deliver an instantaneous luminosity
of $10^{34}cm^{-2}s^{-1}$. Figure~\ref{fig:lumi} (top-left and central
plots) shows the schedule of the Large Hadron Collider from the start
to the following years of operations. The LHC has delivered two
outstanding runs of data taking: the first phase, Run1 (2010-2012) at
center-of-mass energy of 7 and 8\TeV and total delivered integrated
luminosity of $29.4\ fb^{-1}$; the first 3 years of data taking proved
the physics potentiality of the LHC with, among others, the Higgs boson
discovery. The second run, Run2 (2015-2018) started after 2 years Long
Shutdown when the machine and the detectors were confirmed and
consolidate to be able to run at the full capacity with 
center-of-mass energy of 13\TeV and total delivered integrated
luminosity of $162.9\ fb^{-1}$.\\
The increase in luminosity over the
years was the result of improvements in the beam quality and optics which
led to an higher number of pp collisions per bunch crossing. This
quantity is referred as pileup, PU which is shown in the top-right plot
in Figure~\ref{fig:lumi}. The average \textlangle{}PU\textrangle{} for
Run2 is 34. On one hand this large
number of collision per bunch crossing 
expands the physics reach of CMS and ATLAS because of
the higher probability of an episode of a rare collision; however
most of the PU interactions pollute the information of the
event being mostly soft and less interesting to look for
new physics models. Thus it is challening for the detector and the for
the reconstruction algoritms to be able 
to disentangle and reconstruct each single pp collision per single
bunch crossing.

%-----------------------------------
%	SECTION 2
%-----------------------------------
\section{The Compact Muon Solenoid}

The Compact Muon Solenoid (CMS) detector is located at one of the four
collision points along the LHC ring, precisely at LHC P5 in Cessy in
France.  

CMS is a multi-purpose detector designed to observe any new physics
phenomena that could appear at proton-proton collision. CMS behaves
like a high-speed camera capturing instant frames of the particle
collisions up to 40 million times per second. Then by trying to
identify the particles produced and created after the collision,
measuring their energies and momenta the detector aims to recreate a
photograph of the collision for offline analysis. \\
The idea behind the design was to create, around the place where the
two proton beams cross each-other, a structure of concentric cylindrical layers
in order to be able to track and measure the path of the particle
escaping from the center.   

\subsection{The CMS coordinate system} 
The coordinate system used by CMS is a right-handed system defined
locating its center in
\begin{wrapfigure}{r}{0.5\textwidth}
  \begin{center}
    \includegraphics[clip,trim=0cm 0cm 0cm 0.1cm, width=0.48\textwidth]{Figures/c2/cms_coordinate_system.png}
  \end{center}
  \caption{A scheme of the coordinates system used by CMS~\cite{coordinate_cms}.}
\label{fig:coordinates}
\end{wrapfigure}
 the nominal interaction point,the
\emph{y}-axis is vertical pointing upwards, the \emph{x}-axis is
radial pointing inward towards the center of the LHC ring and the
\emph{z}-axis coincides with the direction of the beam
(counter-clockwise beam); refer to
Figure~\ref{fig:coordinates}. The azimuthal angle $\phi$ is defined
from the \emph{x}-axis in the \emph{x-y} plane and the polar angle
$\theta$ is measured from the \emph{z}-axis in the same transverse
plane meaning \emph{x-y} plane.\\
For an object of energy $E$ and momentum $\overrightarrow{p}$,
rapidity, $y$ and pseudorapidity, $\eta$ are defined as:
\begin{equation}
\label{eq:pseudo}
y \: = \: \frac{1}{2} \ln \frac{E + p_z}{E - p_z} \;\; \approx \;\;
\eta \: = \: \frac{1}{2} \ln \frac{|\overrightarrow{p}| +
  p_z}{|\overrightarrow{p}| - p_z} \: = \: -\ln \tan (\frac{\theta}{2})
\end{equation}
The approzimation of the rapity with the pseudorapity is possible for
relativistic particles with $p_{T} \gg m$. The rapidity is used to
measured the angular distance between particles, $\Delta R =
\sqrt{\Delta y ^2 + \Delta \phi ^2}$ which is Lorentz invariant under
boots along $z$-axis the beam direction. Knowing the approximation above,
the $\Delta R$ quantity is often defined as $\Delta R =
\sqrt{\Delta \eta ^2 + \Delta \phi ^2}$.\\
Finally, using the $x$ and $y$ components, the transverse variables
are defined: the transverse momentum, $p_T$ and the transverse energy,
$E_T$.


\subsection{CMS detector} 
The schematic representation of the CMS detector and its parts is
shown in Figure~\ref{fig:detector}.
\begin{figure}[h]
\centering
\includegraphics[width=0.98\textwidth]{Figures/c2/cms_160312_06-compressed.pdf}
\vspace*{3mm}
\caption{A scheme of the CMS detector and its parts~\cite{webpage_cms}.}
\label{fig:detector}
\end{figure} 

With CMS detector is possible to measure photons, electrons, muons and hadrons (neutral and charged). 
To be able to acchieve such results, CMS is made of a system of subdetectors each of which contributes with
measurements of specific properties and quantities of different 
particles; the overlap and combination of all the
informations from the subdetectors allow to identify and measure the
properties of the particles produced during the collision~\cite{Sirunyan_2017}. Beginning
with the region at the immediate proximity to the interaction point, a
particle meets first the tracker where its trajectory, if charged, is
measured. This measure is possible thanks to the presence of the
magnetic field, created by the solenoid, which bends charge particles
and thus tracks reconstruction provides insigth on electric charge and
momenta of the particle itself. Subsequently there are the
electromagnetic (ECAL) and hadronic (HCAL) calorimeters where
electrons/photons and hadrons are repectivily absorbed and their
energies measured. Finally the muons, getting through the
calorimeters, enter into the muon chambers where complete trajectory
is then measured. 

The subsystems of the CMS experiment are listed and briefly described
in the following paragraphs.

\subsubsection{The superonducting solenoid}
The central part of CMS is a solenoid magnet of 6 m internal diameter
which is made of a cylindrical coil of superconducting fibres
provinding a magnetic field of 3.8 T. 
\subsubsection{The tracking system}\label{sec:tracking}
The design of the CMS tracking system is optimized to
effeciently and precisesly measure the trajectories of charged
particles and to effectively reconstruct secondary vertices. This latter feature is of particular
importance in the context of displeced vertices and lepton searches as
described in Chapter~\ref{Chapter6}.

\begin{figure}[h]
\centering
\includegraphics[width=0.98\textwidth]{Figures/c2/las}\\
\vspace{0.5cm}
\includegraphics[width=0.68\textwidth]{Figures/c2/Phase1_Tracker_1Quarter.pdf}

\caption{Top: a quarter of the CMS silicon tracker in an $r-z$
  view. The strip tracker comprises several parts: the tracker inner
  barrel (TIB), outer barrel (TOB), inner disks (TID) and endcaps
  (TED)~\cite{Adam:1171503}. Bottom:
sketch of one quarter of the current CMS tracking system in
  r-z view, 2017-2018 data taking. The pixel detector is shown in
  green with the additional modules~\cite{trackingPU}.}
\label{fig:tracker}
\end{figure} 
Additionally the tracking system has to feature high granularity and
fast response in order to correclty associate each reconstructed track
to the respective bunch crossing and the primary interaction vertex.

The CMS tracker consists of a pixel detector (pixel Tracker) and a
silicon strip detector (strip Tracker), see Figure~\ref{fig:tracker}.
The original pixel detector was made of three barrel
layers at radii of 4.4, 7.3, and 10.2 cm and two endcaps modules in
the forward region. 
During the short shutdown between the data takings 2016 and 2017, it
was installed an
upgraded version of the pixel detector; the detector has currently four
barrel layers at radii of 3.0, 6.8, 10.2, and
16.0 cm and three layers in the forward region~\cite{Dominguez:1481838}. The
recent innermost layer and modules
are positioned closer to the beam pipe in order to improve the
precision on the position of the interaction vertices.\\
The silicon strip detector consists of many parts: the tracker inner and
outer barrels (TIB and TOB), in total ten layers of strip modules in
the barrel; the 6 tracker inner disks (TID), three each
side; and the nine disks on each side of the tracker endcap (TEC).\\
In total the tracking system is 5.8 m long and 2.6 m high,
extending the coverage of the tracker up to $|\eta|$ = 2.5. The total
amount of sensors is 66 million for the pixel and 124 million for the
strip detector. 

\subsubsection{The electromagnetic calorimeter}
The ECAL detector is a fine-grainded and homogeneous calorimeter
made up of lead
\begin{wrapfigure}{r}{0.5\textwidth}
  \begin{center}
    \includegraphics[clip,trim=1cm 1cm 1cm 1.5cm, width=0.48\textwidth]{Figures/c2/ecal}
  \end{center}
  \caption{Geometric view of one quarter of the ECAL~\cite{Benaglia_2014}.}
\label{fig:ecal}
\end{wrapfigure}
 tungstate crystals. Those single crystals are extremely transparent
 and ``scintillates''  when photons and electrons pass through
 them; the light produced is proportional to the particle's energy
 allowing a fast and very precise measurement of the momentum
 property. The single crystal length in barrel region is 230 mm (220
 mm in endcap) comparable to $\sim$26 (25) radiation lengths meaning it
 absorbs more than 98\% of the energy deposited by the particle~\cite{Biino_2015}.\\
A scheme of the ECAL is shown in Figure~\ref{fig:ecal}.
ECAL modules are placed in the barrel ($\eta<$ 1.479) and endcap
(1.635 $<\eta<$ 3.0) regions, and a preshower detector is located just
before the endcap crystals. The preshower detectors help CMS to
distinguish between single high-energy photons and the less
interesting pairs of low-energy photons very close to each other, \ie
coming from the decay of a $\pi^0$. 

\subsubsection{The hadron calorimeter}
The HCAL detector is a hermetic sampling calorimeter which means it
consists of alternative layers of ``absorber'' and ``scintillator''
materials that measure a particle’s position, energy and arrival time.
The quantity of light in a given position is summed up over several
layers of tiles in depth, called a “tower”, thus this total amount of
light is a measure of a particle’s energy.\\
The HCAL is located both inside the solenoid, the main part, and
outside it, the outer barrel (HO). 
Inside the magnet coil, the barrel (HB) and endcap parts (HE) cover
respectively the pseudorapidity
ranges $\eta<$ 1.3 and 1.3 $<\eta<$ 3.
The forward region of pseudorapidity is covered by the forward hadron
calorimeter (HF) up to $\eta<$ 5. It is made up of iron radiators and
quartz-fibre sensors and it measures both the electromagnetic and the hadronic shower. 
The outer barrel, HO, ensures no energy leaks out the
back of the HB undetected.

\subsubsection{The muon system}\label{sec:muonsystem}
The muon system is constructed to detect muons and to measure their trajectories.\\ 
It is composed by three different kind of gaseous particle
detectors inserted in the steel yoke. There are the drift tubes
modules, DTs, the cathode strip chambers, CSCs and the resistive plate
chambers, RPCs.\\
The four layers of DTs are located in the barrel and they cover up to
$\eta<$ 1.2 pseudorapidity range in the detector. The four layers of
CSCs are installed in the endcap covering the pseudorapidity range 0.9
$<\eta<$ 2.4. Finally there are the RPCs positionated both in barrel
and endcap parts up to $\eta<$ 1.6. The whole pseudorapidity range of
the muon system allow to measure muons up to $\eta<$ 2.4.

\begin{figure}[h]
\centering
\includegraphics[width=0.99\textwidth]{Figures/c2/cms_quadrant_run_ii.pdf}
\caption{Geometric view of one quadrant of CMS. The grey areas are
  tracker, ECAL and HCAL systems, previously explained; the colored
  areas show the muon system and its subsystem. The drift tube, DTs,
  modules are labelled MB (muon barrel) and the cathode strip
  chambers, CSCs, are labelled ME (muon endcap). Resistive plate
  chambers, RPCs, are labelled RB and RE and they are mounted in both the barrel and endcaps of CMS
~\cite{muonsystemPU}. }
\label{fig:muonsystem}
\end{figure} 

\subsubsection{The CMS trigger system}\label{sec:trigger}

At center of the CMS detector, proton-proton collisions occour every
25 ns which means a frequency of 40 MHz. However not every collision is
necessarly of potential interest for the CMS physics program and
moreover there are technical limitations on the rate the collision data can be
saved on disk to be analyzed offline. Thus a trigger is needed to be
able to sort between potentially interesting events and the extent of
inelastic scattering events.
Since the reading and storing time of the collision data is larger
than the collision frequency the viable solution is
to store the informations in pipelines that hold and process data
from several collision at the same time.
To work without mixing particles from two different events, it is
required detectors have very good time
resolution and the signal from the millions of channels of different
systems to be synchronized and integrate.

The CMS trigger has a two-stage architecture. The first level, L1 is
implemented in custom hardware and uses informations from the 
all muon systems and the calorimeters to identify events applying a fast
basic identification of measured particles. The first step reduces the
event rate to $\sim$100 kHz.\\
The events sorted by the L1 are further refined by the
high-level trigger, HLT. It is a software farm that employes informations from all subdetectors to perform a
refined event reconstruction reducing the rate down to a few kHz. The 
events are then saved for offline analysis.
The trigger selection is a irreversible process, what was not selected
by it is lost and it can not be recovered~\cite{Khachatryan_2017}.

A complete sequence of L1
and HLT selection criteria, including any prescale, is re- ferred to
as a trigger path.



\clearpage

%-----------------------------------
%	SECTION 2
%-----------------------------------

\section{Event reconstruction}\label{sec:reconstruction}
\subsection{Track reconstruction} 
Electron reconstruction is based on the combination of tracker and
ECAL information in a Gaussian Sum Filter (GSF)
track~\cite{Khachatryan:2015hwa}, which accounts for possible
bremsstrahlung from the electron.
Electrons are reconstructed within the geometrical acceptance of the
CMS tracking system, $|\eta|<2.5$.
Identification criteria based on the electromagnetic shower shape, track
quality, track impact parameters with respect to the primary vertex,
and isolation are used to select signal electrons and reduce the rate
of mis-identified and background electrons (referred to as ``fake
electrons'' hereafter).

Muons are reconstructed by combining the information of the tracker
and of the muon
spectrometer~\cite{Sirunyan:2018fpa}.
The geometric compatibility between these separate measurements is
used in the further selection of muons. Muons are required to have
$\abseta<2.4$ to fall inside the geometric acceptance of the muon
detector.
All muons considered for analysis must pass the loose working point as
specified by the MUO POG, in addition to a number
of other loose criteria on isolation and their impact parameters with
respect to the PV.
It is also possible to require muons to be synchronized with the bunch
crossing that has triggered, using the time measurements provided by
the muon sub-detectors, the RPCs (``RPC time'' or $t_{\mathrm{RPC}}$)
and the combined measurements of the DTs and CSCs (``combined time''
or $t_{\mathrm{comb}}$)~\cite{muon_oot}.
In particular, $t_{\mathrm{RPC}}$ ($t_{\mathrm{comb}}$) is only used
if it is measured with more than 1 (7) degrees of freedom.
If $t_{\mathrm{RPC}}$ and $t_{\mathrm{comb}}$ are both available,
they must lie within $-10\ns$ and $+10\ns$.
If only $t_{\mathrm{comb}}$ is available, then it must be within
$-45\ns$ and $+20\ns$.
If $t_{\mathrm{comb}}$ is unavailable, no timing requirement is
applied.

\subsection{Reconstruction performances}
\clearpage
