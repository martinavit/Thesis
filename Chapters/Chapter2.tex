% Chapter Template
\chapter{The LHC and the CMS Experiment --> 30 April} % Main chapter title

\label{Chapter2} % Change X to a consecutive number; for referencing this chapter elsewhere, use \ref{ChapterX}


\section{Introduction}

%----------------------------------------------------------------------------------------
%	SECTION 1
%----------------------------------------------------------------------------------------
\section{The LHC and the LHC experiments}
\subsection{The Large Hadron Collider}
\subsection{Experiments at the LHC}
\subsection{Design and performance of the LHC}

%-----------------------------------
%	SECTION 2
%-----------------------------------
\section{The CMS experiment}
\subsection{The CMS coordinate system} 
\subsection{The solenoid magnet} 
\subsection{The charged-particle tracker} 
\subsection{The electromagnetic calorimeter} 
\subsection{The hadronic calorimeter} 
\subsection{The muon detector} 
\subsection{Triggering and data acquisition} 
\section{Event reconstruction}\label{sec:reconstruction}
\subsection{Track reconstruction} 
Electron reconstruction is based on the combination of tracker and
ECAL information in a Gaussian Sum Filter (GSF)
track~\cite{Khachatryan:2015hwa}, which accounts for possible
bremsstrahlung from the electron.
Electrons are reconstructed within the geometrical acceptance of the
CMS tracking system, $|\eta|<2.5$.
Identification criteria based on the electromagnetic shower shape, track
quality, track impact parameters with respect to the primary vertex,
and isolation are used to select signal electrons and reduce the rate
of mis-identified and background electrons (referred to as ``fake
electrons'' hereafter).

Muons are reconstructed by combining the information of the tracker
and of the muon
spectrometer~\cite{Sirunyan:2018fpa}.
The geometric compatibility between these separate measurements is
used in the further selection of muons. Muons are required to have
$\abseta<2.4$ to fall inside the geometric acceptance of the muon
detector.
All muons considered for analysis must pass the loose working point as
specified by the MUO POG, in addition to a number
of other loose criteria on isolation and their impact parameters with
respect to the PV.
It is also possible to require muons to be synchronized with the bunch
crossing that has triggered, using the time measurements provided by
the muon sub-detectors, the RPCs (``RPC time'' or $t_{\mathrm{RPC}}$)
and the combined measurements of the DTs and CSCs (``combined time''
or $t_{\mathrm{comb}}$)~\cite{muon_oot}.
In particular, $t_{\mathrm{RPC}}$ ($t_{\mathrm{comb}}$) is only used
if it is measured with more than 1 (7) degrees of freedom.
If $t_{\mathrm{RPC}}$ and $t_{\mathrm{comb}}$ are both available,
they must lie within $-10\ns$ and $+10\ns$.
If only $t_{\mathrm{comb}}$ is available, then it must be within
$-45\ns$ and $+20\ns$.
If $t_{\mathrm{comb}}$ is unavailable, no timing requirement is
applied.

\subsection{Reconstruction performances}
\clearpage
