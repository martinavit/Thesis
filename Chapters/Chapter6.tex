\chapter{Search for long-lived HNL in final states
with three charged leptons and displaced vertices} \label{Chapter6} 

\section{Change of scenario}
While the SM is in excellent agreement with the LHC measurements, many problems
should be still resolved.

So far an explicit signal of New Physics from either direct or indirect
searches at the LHC, or direct detection of Dark Matter is
missing. Furthermore, theory presents no clear direction on the New Physics scale.
This forces a refining and widening of the experimental effort in the
investigation for New Physics. We should explore various ranges of
interaction strengths and masses in addition with respect to
what is already probed and explored by the current experimental
settings and analysis.

Among the plethora of possible new physics theories, quite eccentric
possibilities of new particles could long-lived particles (LLPs)
decaying in the detector or long-lived detector-stable particle which decay outside the detector volume. 
These two require a change of scenario with respect to the usual SM
prompt searches probed till these days and they do require the
inclusion of unconventional signature searches.

Long-lived particles could be foreseen by several models due to:
\begin{itemize}
\setlength\itemsep{-0.1em}
\item \emph{small coupling constants} -- \ie RPV SUSY (R-parity violating
  SUSY scenario), HNL, etc;
\item \emph{very off-shell intermediate decay products} -- \ie split SUSY where
heavy intermediate squarks enhance
the gluino lifetime~\cite{ATLAS-CONF-2019-006,2020135114};
\item \emph{limited decay phase space} --- \ie
anomaly mediated SUSY breaking
model where the lightest neutralino
and chargino are nearly degenerate~\cite{Sirunyan:2019wau,Sirunyan:2019gut}.
\end{itemize}

CMS and ATLAS were 
initially designed to optimize object
identification for prompt particles, thus to integrate displaced
signatures into CMS framework we have to face quite a diverse number
of challenges:
\begin{itemize}
\setlength\itemsep{-0.1em}
\item \emph{triggers}: timing information at trigger level is not always
  available and the majority of current triggers uses primary vertex
  seeds to identify the matching tracks;
\item \emph{reconstruction}: we need the possibility to use algorithms which
  fit the eventual secondary vertices and adopt such infos in the
  reconstruction procedure;
\item \emph{background}: additional sources of background have to be
  considered such as noise from the detector itself (instrumental
  background), cosmic rays traveling through CMS's barrel, in-time
  and out-of-time pileup and long-lived standard model hadrons which
  could mimic the interesting BSM particle.
\end{itemize}

\subsection{Displaced signature at CMS}
In the sketch in Fig.~\ref{fig:c6antonelli}, several displaced
signatures are shown.

If the
LP decay into charged objects, we can have visible secondary vertices
made either of displaced leptons or of displaced jets; in case the LP decays into two $\gamma$,
then we should rely on the possibility of the CMS
ECAL to measure $\gamma$ arrival times with enough precision to detect
signatures of delayed photons produced at displaced
vertices~\cite{Sirunyan:2019wau}.
There is also the possibility of having the LP that decays within the
volume of the silicon tracker producing events with an isolated track
which "disappears'' \ie it misses hits in the outermost layers of the
silicon tracker, it is associated with little energy deposited in the
calorimeters and it does not show hits in the muon detectors~\cite{Sirunyan_2020disapp}. 
\begin{figure}[h]
\centering
\includegraphics[width=0.85\textwidth]{Figures/c6/antonelli_skech.pdf}
\caption{Didactic sketch, by J. Antonelli, showing the numerous
  Long-lived scenarios which could be seen and probed inside the CMS tracker. }
\label{fig:c6antonelli}
\end{figure}

These are just few examples of the almost endless configurations which can
be studied including the \emph{long-lived} signatures in the
current particle physics searches. 

\subsubsection{Long-lived HNL}
Along these lines, extending the HNL search described in
Chapter~\ref{Chapter5} including the long-lived HNL decays it feels like a logical
continuation. 

Recalling the considerations illustrated in Sec.~\ref{sec:promptll}, the lifetime of a HNL is strongly dependent on \mhnl and \mixpar,
and increases rapidly at small masses and low values of the mixing
parameter (see Fig.~\ref{fig:hnlLifetime}):
\(\tau_\hnl\propto\mathrm{\mhnl^{-5}|V_{\hnl\ell}|^{-2}}.\)
As a result, the kinematics and acceptance of HNLs with masses
below about 20\GeV are significantly affected by their long lifetimes,
and must be accounted for in the signal simulation and in the result
interpretation.
If $\hnl$ has a long lifetime, in particular, its decay products
($\ell^{\pm\prime}$, $\ell^{\mp\prime\prime}$, $\nu_{\ell^{\prime\prime}}$ or
$\nu_{\ell^{\prime}}$, $\ell^{\pm\prime\prime}$, $\ell^{\mp\prime\prime}$)
emerge from a secondary vertex, spatially displaced with respect to
the primary vertex of the process, and distinguishable from it.

We present here the search of long-lived Heavy Neutral Leptons in final states
with three charged leptons and displaced vertices. 

The analysis strategy we follow here shares some similarities with the
prompt HNL search (~\cite{Sirunyan:2018mtv}-Chapter~\ref{Chapter5})
and uses the lessons learned from that
analysis. 
The main objective of the current search is to extend the
sensitivity to low HNL masses and mixing parameters, namely HNL masses
below about 20 GeV. This is obtained by optimizing the identification
of leptons produced in the decay of long-lived HNLs and by fitting
their displaced decay vertices. More precisely, the search is
performed by designing search regions formed by the displacement of
the HNL vertex and the invariant mass of the \displ leptons. The
maximum sensitivity is reached by simultaneously fitting these search
regions in each lepton flavor channels. The mixing of HNL to electron
neutrino, \mixpare, is probed using the leptonic channels 
$\Pe\Pe\Pe$, $\Pepm\Pemp\PGm$, and $\Pepm\Pepm\PGm$ (collectively
called \eex),
while the $\PGm\PGm\PGm$,  $\PGmpm\PGmmp\Pe$, and $\PGmpm\PGmpm\Pe$
channels (collectively called \mmx) are used to probe the coupling to
muon neutrino, \mixparm.
With these lepton flavor/charge configurations, the search is
sensitive to both LNV and LNC scenarios.
The search uses the full Run-II data set with an integrated luminosity
of 137\fbinv, and the major backgrounds are estimated using a
data-driven technique.

%%%%%%%%%%%%%%%%%%%%%%%%%%%%%%%%%%%%%%%%%%%%%%%%%%%%
\section{Analysis setup}
\subsection{Data and simulation samples}
The current analysis uses three sets of $pp$ collision data at a
center-of-mass energy of 13\TeV, corresponding to integrated
luminosities of 35.92\fbinv (2016), 41.53\fbinv (2017), and 59.97\fbinv
(2018). Several primary data sets (PD) are used to search for HNLs decaying
to different lepton flavors, as well as to build control regions for
background estimation:
\begin{itemize}
\setlength\itemsep{-0.1em}
\item SingleElectron (EGamma in 2018);
\item SingleMuon;
\item DoubleEG (EGamma in 2018);
\item DoubleMuon.
\end{itemize}
Possible overlaps among different data sets are
removed by checking run, lumi-section, and event numbers in order to
not have twice the same event coming from different PDs.

Data samples in \texttt{MiniAOD} format are used, including all the
latest/greatest detector and object calibrations available at the time
the analysis is conducted.
To ensure the best quality for the analyzed data, only certified data
events included in the so-called "golden'' JSON files.

This analysis employs Monte Carlo samples generated in the
\texttt{Summer16} (2016), \texttt{Fall17} (2017), and
\texttt{Autumn18} (2018) campaigns.
To reproduce the correct multiplicity of $\Pp$ interactions per
bunch crossing ("pileup'' or PU) observed in data,
simulated minimum-bias events are mixed to the MC signal and
background events, following appropriate PU scenarios for each data
set (2016, 2017, and 2018).
An event re-weighting procedure is applied \textit{a posteriori} to
correct possible residual discrepancies. The luminosity scenario has a 25~ns bunch crossing separation with an
average of about 25 pileup interactions per bunch crossing. 

A number of signal samples were used for the optimization of the
selection and the interpretation of the results, using the modeling
described in Section~\ref{Chapter4}.
Signal samples include HNL decays with three charged leptons and a
neutrino in the final state, coming from both the
$\hnl\to\PW(\ell\PGn)\ell$ and $\hnl\to\PZ(\ell\ell)\PGn$ decay modes
of \hnl.

\subsection{Trigger strategy}\label{sec:trigger}
Every HNL signal event contains one prompt lepton and two (generally)
\displ leptons. Since (most of) the CMS leptonic triggers are
optimized for prompt lepton identification,
we decided for the use of single-electron and single-muon triggers for the
signal selection, as listed in Table~\ref{tab:sgnlTriggers}.
\begin{table}[h]
{\small
  \begin{center}
    \caption{\label{tab:sgnlTriggers} List of triggers used for the
      signal selection in the three data-taking periods.}
      \begin{tabular}{|l|c|c|c|}
      \hline
      \multirow{2}{*}{Primary data set} & \multicolumn{3}{c|}{Trigger name}\\
      \cline{2-4}
      & 2016 & 2017 & 2018 \\
      \hline\hline
      SingleElectron & \texttt{\scriptsize Electron > 27} & \texttt{\scriptsize Electron > 32} & --- \\
      \hline
      EGamma         & --- & --- & \texttt{\scriptsize Electron > 32} \\
      \hline
      \multirow{2}{*}{SingleMuon} & \multirow{2}{*}{\texttt{\scriptsize Muon > 24}} & \texttt{\scriptsize Muon > 24} & \multirow{2}{*}{\texttt{\scriptsize Muon > 24}} \\
      & & \texttt{\scriptsize Muon > 27} & \\
      \hline
    \end{tabular}    
  \end{center}}
\end{table}
The trigger efficiency in  Monte Carlo (MC) samples is corrected
according to the efficiency observed in data, by using per-event scale
factors (SFs) as a function of the prompt lepton \pt and $\eta$.
In order to make this correction possible, the prompt lepton
identified in each event is matched geometrically to the relevant
"trigger lepton'' (\ie, the lepton reconstructed by the CMS
high-level trigger software) that fired the event. Prompt electrons
(muons) must be matched to the trigger electron (muon) that fired one
of the single-electron (-muon) triggers in
Table~\ref{tab:sgnlTriggers}.
The matching is ensured by requiring that the angular distance
\(\Delta R=\sqrt{\left(\Delta\phi\right)^2+\left(\Delta\eta\right)^2}\) 
between the reconstructed trigger lepton and offline lepton be less
than 0.3. This is the same \DR\ cut used to obtain the trigger SFs
with a tag-and-probe method (see Sec.~\ref{sec:triggereff}).

\subsection{Object selection}\label{sec:object}
For the rigorous explanation of the single object reconstruction in
CMS see Chapter~\ref{Chapter2}, section~\ref{sec:reconstruction}.
In the same section the general reconstruction performances of
displaced object is presented; in the coming subsections, we will
present the efficiency of the current object selection for displaced
object using the HNL-signal MC samples. \\

Signal leptons coming from the \PW\ boson are required to originate
from the event's primary $\Pp$ interaction vertex (PV), defined as
the reconstructed vertex with the highest sum of squared transverse
momenta of associated charged particles, plus the \ptmiss.
The distance of closest approach of the lepton track from the event's
PV (impact parameter, IP) is used to evaluate three variables:
the projection of the IP onto the plane transverse to the beam line
(\textbf{transverse impact parameter} $\boldsymbol{\dxy}$),
the projection of the IP along the beam line (\textbf{longitudinal impact
parameter} $\boldsymbol{\dz}$),
and the significance of the IP
($\mathrm{SIP} = \mathrm{IP}/\sigma_{\mathrm{IP}}$,
where $\sigma_{\mathrm{IP}}$ is the uncertainty associated to IP). Signal leptons coming from the HNL decay are selected by requiring a
minimum \dxy value, and do not have additional constraints on \dz nor
on the IP significance (see Tables~\ref{tab:electronSelection} and
\ref{tab:muonSelection}).

Signal leptons are required to be isolated from any hadronic activity
in the event.
An isolation variable (\Irel) is computed as the scalar 
\pt sum of charged hadrons originating from the PV, neutral hadrons,
and photons within a cone of $\Delta R<0.3$ around the lepton
candidate direction at the vertex,
divided by the transverse momentum $\pt^\ell$ of the lepton candidate:
\begin{linenomath}
  \begin{equation}
    \label{eq:irel}
    \Irel = \frac{1}{\pt^\ell}
    \left(\sum_{ch.hadr.}\pt^{\mathrm{PV}} +
    \max{\left[0, \sum_{neu.hadr.}\pt + \sum_{pho.}\pt
        - \rho\cdot A_{\mathrm{eff}}\right]}\right).
  \end{equation}
\end{linenomath}
The term $\rho\cdot A_{\mathrm{eff}}$ is used to mitigate the
contribution of pileup to the isolation calculation: the average
transverse-momentum flow density $\rho$ is calculated in each event
using a ``jet area'' method~\cite{CACCIARI2008119}, and the effective
area $A_{\mathrm{eff}}$ is the geometric area of the isolation cone
times an $\eta$-dependent correction factor that accounts for the
residual dependence of the isolation on pileup.

\subsubsection{Electrons}\label{sec:llelectron}
Electron reconstruction is based on the combination of tracker and
ECAL information in a Gaussian Sum Filter (GSF)
track~\cite{Khachatryan:2015hwa}, which accounts for possible
bremsstrahlung from the electron.
Electrons are reconstructed within the geometrical acceptance of the
CMS tracking system, $|\eta|<2.5$.
Identification criteria based on the electromagnetic shower shape, track
quality, track impact parameters with respect to the primary vertex,
and isolation are used to select signal electrons and reduce the rate
of mis-identified and background electrons (referred to as ``fake
electrons'' hereafter).
The electron selection criteria are summarized in
Table~\ref{tab:electronSelection}. 
Different criteria are applied to identify prompt and \displ
electrons.
\subparagraph {Prompt electrons}
Prompt electrons are identified using a multi-variate discriminator
(EGM POG 90\% working point MVA ID),
must have \pt greater than 30 (32, 32)\GeV in the 2016 (2017, 2018)
data set,
and must be matched to a ``trigger electron'' (see
Sec.~\ref{sec:trigger}).
The offline \pt threshold is driven by the trigger \pt threshold in
each data set, such that the offline cut always falls in the plateau
of the trigger efficiency turn-on curve.
Prompt electrons passing this selection will be referred to as ``tight
prompt electrons'' hereafter.
\subparagraph {Displaced electrons}
\Displ electrons are identified using a cut-based selection (``EGM
POG cut-based Loose ID''), but removing a veto
on photon conversions and a requirement on the maximum number of
missing inner hits, which could affect the efficiency for electrons
not emerging from the primary vertex.
\Displ electrons passing this selection will be referred to as
``tight \displ electrons''.
They must have \pt greater than 7\GeV.
Samples enriched in fake electrons are selected with
identification criteria looser than those used for tight \displ
electrons. These ``loose \displ electrons'' or ``fakeable objects''
(FO) are employed to determine the rate of fake electrons
mis-identified as signal electrons (``fake rate'' or FR).
\begin{table}[h!]
  \centering
  \caption{\label{tab:electronSelection} Requirements for an electron
    to pass each of the defined selection working points.
    Where three values are given for a single variable, they
    correspond to electrons with $\abseta< 0.8$, $0.8<\abseta<1.479$,
    and $1.479<\abseta<2.5$. The cut on the prompt electron MVA
    discriminator depends on the electron \sigeta and \pt. }
  %  \(f(\sigeta,\pt) = \min{\left\{ x_{15}(\sigeta),\,
   %   \max{\left\{x_{25}(\sigeta),\,x_{15}(\sigeta) +
   %     0.1\cdot(x_{25}(\sigeta)-x_{15}(\sigeta)) \cdot
     %   (\pt-15\GeV)\right\}}\right\}},\)\\
   % where $x_{15}(\sigeta) = (0.77,\,0.56,\,0.48)$ and
    %$x_{25}(\sigeta) = (0.52,\,0.11,\,-0.01)$, with the three values
   % applying to the same \abseta bins defined above.}
  \resizebox{1.0\textwidth}{!}{
    \begin{tabular}{c|c|c|c}
      %% \hline
      %% Cut & Tight prompt & Tight \displ & Loose \displ (FO) \\
      %% \hline
      \hline
      Selection name & Tight prompt & Tight \displ & Loose \displ (FO) \\
      \hline
      EGM POG ID & MVA ID --- 90\% w.p. &
      \multicolumn{2}{c}{Modified cut-based Loose ID} \\
      \hline
      \hline
      $\abseta$ & $<2.5$ & $<2.5$ & $<2.5$ \\
      $\pt$ & $>30$--$32\GeV$ & $>7\GeV$ & $>7\GeV$ \\
      $|d_{xy}|$ & $<0.05$ cm & $>0.01$ cm & $>0.01$ cm \\
      $|d_z|$ & $<0.1$ cm & --- & --- \\
      %$\mathrm{SIP_{3D}}$ & $<4$ & --- & --- \\
      \Irel & $<0.1$ & $<0.2$ & $<2.0$ \\
      $\sigma_{i\eta i\eta}$ & --- & $<(0.011,\,0.011,\,0.030)$ & $<(0.011,\,0.011,\,0.030)$ \\
      H/E & --- & $<(0.10,\,0.10,\,0.07)$ & $<(0.10,\,0.10,\,0.07)$ \\
      $\Delta\eta_{\textrm{in}}$ & --- & $<(0.01,\,0.01,\,0.008)$ & $<(0.01,\,0.01,\,0.008)$ \\
      $\Delta\phi_{\textrm{in}}$ & --- & $<(0.04,\,0.04,\,0.07)$ & $<(0.04,\,0.04,\,0.07)$ \\
      $1/E-1/p$ & --- & $<(0.010,\,0.010,\,0.005)$ & $<(0.010,\,0.010,\,0.005)$ \\
      MVA estimator & $>f(\sigeta,\pt)$ & --- & --- \\
      \hline
    \end{tabular}
    }

\end{table}

\subsubsection{Muons}\label{sec:llmuon}
Muons are reconstructed by combining the information of the tracker
and of the muon
spectrometer~\cite{Sirunyan:2018fpa}.
The geometric compatibility between these separate measurements is
used in the further selection of muons. Muons are required to have
$\abseta<2.4$ to fall inside the geometric acceptance of the muon
detector.
All muons considered for analysis must pass the loose working point as
specified by the MUO POG, in addition to a number
of other loose criteria on isolation and their impact parameters with
respect to the PV.
It is also possible to require muons to be synchronized with the bunch
crossing that has triggered, using the time measurements provided by
the muon sub-detectors, the RPCs (``RPC time'' or $t_{\mathrm{RPC}}$)
and the combined measurements of the DTs and CSCs (``combined time''
or $t_{\mathrm{comb}}$)~\cite{muon_oot}.
In particular, $t_{\mathrm{RPC}}$ ($t_{\mathrm{comb}}$) is only used
if it is measured with more than 1 (7) degrees of freedom.
If $t_{\mathrm{RPC}}$ and $t_{\mathrm{comb}}$ are both available,
they must lie within $-10\ns$ and $+10\ns$.
If only $t_{\mathrm{comb}}$ is available, then it must be within
$-45\ns$ and $+20\ns$.
If $t_{\mathrm{comb}}$ is unavailable, no timing requirement is
applied.
This timing requirement is found to be fully efficient for signal
muons. However, we also find that the background from out-of-time
muons is effectively removed by other muon and event selections (as
described in this section and in Section~\ref{sec:selection}), thus
making the timing requirement redundant. For this reason, it is not
applied.  
Muon selection criteria are summarized in
Table~\ref{tab:muonSelection}.

Different selections are applied to prompt and \displ muons.
\subparagraph {Prompt muons}
Prompt muons must pass identification criteria based on track quality
and matching of the inner tracker track with the measurements in the
muon detectors (``MUO POG Medium ID''), must have \pt greater than
25 (28, 25)\GeV in the 2016 (2017, 2018) data set, and must be matched
to a ``trigger muon'' (see Sec.~\ref{sec:dataset}).
As in the case of electrons, the offline \pt cut is driven in each
data set by the trigger \pt threshold.
Prompt muons passing this selection will be referred to as ``tight
prompt muons'' hereafter.
\begin{table}[h!]
  \centering
  \caption{\label{tab:muonSelection} Requirements for a muon
    to pass each of the defined selection working points.}
  \resizebox{1.0\textwidth}{!}{
    \begin{tabular}{r|c|c|c|c}
      %% \hline
      %% \multicolumn{2}{c|}{Cut} & Tight prompt & Tight \displ & Loose \displ (FO) \\
      %% \hline
      \hline
      \multicolumn{2}{c|}{Selection name} & Tight prompt & Tight \displ & Loose \displ (FO) \\
      \hline
      \multicolumn{2}{c|}{MUO POG ID} & Cut-based Medium ID &
      \multicolumn{2}{c}{Modified cut-based Medium ID} \\
      \hline
      \hline
      \multicolumn{2}{c|}{$\abseta$} & $<2.4$ & $<2.4$ & $<2.4$ \\
      \multicolumn{2}{c|}{$\pt$} & $>25$--$28\GeV$ & $>5\GeV$ & $>5\GeV$ \\
      \multicolumn{2}{c|}{$|d_{xy}|$} & $<0.05$ cm & $>0.01$ cm & $>0.01$ cm \\
      \multicolumn{2}{c|}{$|d_z|$} & $<0.1$ cm & $<10$ cm & $<10$ cm \\
      %\multicolumn{2}{c|}{$\mathrm{SIP_{3D}}$} & $<4$ & --- & --- \\
      \multicolumn{2}{c|}{\Irel} & $<0.1$ & $<0.2$ & $<2.0$ \\
      \multicolumn{2}{c|}{Loose ID} & True & True & True \\
      \multicolumn{2}{c|}{Fraction of valid tracker hits} & $>0.8$ & --- & --- \\
      \hline
      \multirow{5}{*}{Global muon} & Global muon fit & True & True & True \\
      & Global track $\chi^2$/dof & $<3$ & --- & --- \\
      & Track--muon matching $\chi^2$/dof & $<12$ & $<12$ & $<12$ \\
      & ``Kink finder'' estimator & $<20$ & $<20$ & $<20$ \\
      & Segment-compatibility estimator & $>0.303$ & $>0.303$ & $>0.303$ \\
      \hline
      Tracker muon & Segment-compatibility estimator & $>0.451$ & $>0.451$ & $>0.451$ \\
      %% \hline
      %% \multicolumn{2}{c|}{Muon timing (NOT USED)} &
      %% \multicolumn{3}{c}{$t_{\mathrm{RPC}},t_{\mathrm{comb}}\in[-10,+10]\ns$
      %%   ~OR~ $t_{\mathrm{comb}}\in[-45,+20]\ns$} \\
      \hline
    \end{tabular}
    }
\end{table}
\subparagraph {Displaced muons}
\Displ muons must pass requirements similar to the Medium ID, but
removing all selections that may reduce the efficiency for muons not
emerging from the primary vertex.
These \displ muons will be referred to as ``tight \displ muons''
and must have \pt greater than 5\GeV.

In addition, ``loose \displ muons'' or ``muon FOs'' are selected
with identification criteria looser than those used for tight
\displ muons, and are employed to determine the muon FR.
\subparagraph{Modified Medium ID for \displ muons}\label{sec:modifiedMedium}

Figure~\ref{fig:modMedium_common}
show the efficiency of the different selections included in the
standard recommended Medium ID for all \displ leptons from HNL
decays (\ltwo and \lthree), for two typical signal scenarios
($\mhnl=4\GeV$, $\mixparm=8.44\times10^{-6}$ and
$\mhnl=8\GeV$, $\mixparm=3\times10^{-7}$),
as a function of the transverse position of the dilepton vertex
(\Deltwod).
In particular, Fig.~\ref{fig:modMedium_common}(A) shows the efficiency of
reconstructing a muon track in the inner tracker, while 
Fig.~\ref{fig:modMedium_common}(B) shows the sequential
efficiencies of the plain PFMuon requirement, Loose ID selection, and
valid tracker hit fraction cut, with respect to the reconstructed
inner-tracker track. It is clear that the efficiency of
this last requirement falls rapidly with increasing HNL
displacement. The valid-hit fraction cut is thus removed from the
\displ-muon Medium ID, and is not applied in the following
efficiencies.
Figure~\ref{fig:modMedium_common}(C) compares the efficiency of the
tracker-muon and global-muon requirements (as described in
Table~\ref{tab:muonSelection}) and their logical OR (\ie, the full
modified Medium ID). The main contribution to the efficiency comes
from the tracker-muon selection, while the addition of the global-muon
selection only increases the overall efficiency by few percent.
Finally, Fig.~\ref{fig:modMedium_common}(D) compares the efficiency of
the full modified Medium ID with the efficiencies obtained by removing
individual cuts from the global-muon definition. None of these cuts
appears to hurt the overall efficiency significantly---the larger
effect is observed for the segment-compatibility cut of 0.303, and it
amounts to 1--2\%. Additionally, none of the cuts on the global-track
variables depends on the displacement.
\begin{figure}[h]
\noindent
\makebox[\textwidth]{  \subfloat[]{\includegraphics[width=.27\textwidth]{Figures/c6/object/tracking_M-4_V-0p00290516780927_rho.png}
  \includegraphics[width=.27\textwidth]{Figures/c6/object/tracking_M-8_V-0p000547722557505_rho.png}}
  \subfloat[]{\includegraphics[width=.27\textwidth]{Figures/c6/object/loose_validFraction_M-4_V-0p00290516780927_rho.png}
  \includegraphics[width=.27\textwidth]{Figures/c6/object/loose_validFraction_M-8_V-0p000547722557505_rho.png}}}\\
 \makebox[\textwidth]{ \subfloat[]{\includegraphics[width=.27\textwidth]{Figures/c6/object/goodTrack_goodGlobal_M-4_V-0p00290516780927_rho.png}
  \includegraphics[width=.27\textwidth]{Figures/c6/object/goodTrack_goodGlobal_M-8_V-0p000547722557505_rho.png}}
\subfloat[]{\includegraphics[width=.27\textwidth]{Figures/c6/object/globalTrack_cuts_M-4_V-0p00290516780927_rho.png}
  \includegraphics[width=.27\textwidth]{Figures/c6/object/globalTrack_cuts_M-8_V-0p000547722557505_rho.png}}}
  \caption{Efficiency of different Medium-ID selections with respect
    to tracks for \displ
    muons in signal scenarios as a function \Deltwod.}
  \label{fig:modMedium_common}
\end{figure}

\textbf{Jet selection and $\boldsymbol{p_{T}^{miss}}$}: the same selection as described in Sec.~\ref{sec:jet} is deployed.


\section{Analysis strategy}\label{sec:llanalisi}
Signal events are
characterized by the presence of a prompt lepton (in the following
often referred to as \lone, or ``\lept from \PW'' in some figures),
two \displ leptons (\ltwo and \lthree, or
``\lept from \hnl'' and ``\lept from $\hnl\to\PW^\ast$'' in some
figures), and a neutrino, see Fig.~\ref{fig:c6llsketch}.

\begin{figure}[h]
\centering
\includegraphics[width=0.85\textwidth]{Figures/c6/llsketch}
\caption{Diagram for the production of a long-lived HNL (left) with colors
  underlining the nomenclature adopted in the following
  sections. Prompt lepton often referred to as \lone, or ``\lept from
  \PW'' in some figures, and two \displ leptons (\ltwo and \lthree, or
``\lept from \hnl'' and ``\lept from $\hnl\to\PW^\ast$'' in some
figures. Didactic sketch on the right showing the usual signature with
 \lone back-to-back wrt \ltwo and \lthree which form a SV.}
\label{fig:c6llsketch}
\end{figure}

In the following sections, events are split into  
final states in which \lone and at least one of \ltwo or \lthree are
electrons (\eex) or muons (\mmx).
Events with \eex final states are sensitive to the \mixpare parameter,
while \mmx events are sensitive to the \mixparm parameter.

Figure~\ref{fig:llfeatures} illustrates some
kinematic properties of the leptons in signal events, both at the
generator and reconstruction levels.
Given the low HNL masses considered in this analysis (\mhnl < 20\GeV),
\lone has the typical \pt spectrum expected for \PW decays, with a
Jacobian peak around 40\GeV,
while \ltwo and \lthree have very soft \pt spectra (Fig.~\ref{fig:llfeatures}(A)), invariant mass smaller than \mhnl (Fig.~\ref{fig:llfeatures}(B)), and a small opening angle
(Fig.~\ref{fig:llfeatures}(C)).
In the absence of significant hadronic activity, \lone and \hnl are
typically separated by a large azimuthal angle (Fig.~\ref{fig:llfeatures}(D)).
These features, along with the possible displacement of \ltwo and
\lthree, can be used to identify the two leptons coming from the HNL
decay.

\begin{figure}[h]
\noindent
\makebox[\textwidth]{  \subfloat[]{\includegraphics[width=.27\textwidth]{Figures/c6/selection/genE_vs_pt_sortby_prov_m4.png}
  \includegraphics[width=.27\textwidth]{Figures/c6/selection/genE_vs_eta_sortby_prov_m4.png}}
  \subfloat[]{\includegraphics[width=.27\textwidth]{Figures/c6/selection/l2l3_mass_gen.png}
  \includegraphics[width=.27\textwidth]{Figures/c6/selection/l2l3_mass_rec.png}}}\\
 \makebox[\textwidth]{ \subfloat[]{\includegraphics[width=.27\textwidth]{Figures/c6/selection/l2l3_dR_gen.png}
  \includegraphics[width=.27\textwidth]{Figures/c6/selection/l2l3_dR_rec.png}}
\subfloat[]{\includegraphics[width=.27\textwidth]{Figures/c6/selection/l1l3_dPhi_gen.png}
  \includegraphics[width=.27\textwidth]{Figures/c6/selection/l1l3_dPhi_rec.png}}}
  \caption{A) Generated lepton \pt (left) and \sigeta (right), separately
    for \lone (``from \PW''), \ltwo (``from \hnl''), and \lthree
    (``from $\hnl\PW$''), produced in the decay of a HNL of mass
    $\mhnl=4\GeV$ and $\mixpar=10^{-4}$.\\
B) Invariant mass of \ltwo
    and \lthree 
    at generator level (left) and at reconstructed level (right), for
    a HNL of mass $\mhnl=4\GeV$ or 10\GeV, and $\mixpar=10^{-4}$.
    Events in these distributions are pre-selected requiring the
    presence of three reconstructed leptons, as defined in
    Tables~\ref{tab:electronSelection} and \ref{tab:muonSelection},
    with the prompt lepton firing a single-lepton trigger, as per
    Table~\ref{tab:sgnlTriggers}.
C) \DR\ between \ltwo 
    and \lthree.
D) \Dphi between \lone
    and \lthree.}
  \label{fig:llfeatures}
\end{figure}

\subsection{HNL candidate selection}
As shown in Tables~\ref{tab:electronSelection} and
\ref{tab:muonSelection},
the displacement of \ltwo and \lthree is requested by imposing a
minimum \absdxy cut of 0.1~mm to the reconstructed leptons.
Given the rapid variation of the \hnl displacement with \mhnl and
\mixpar, this preliminary displacement requirement must be rather
mild, and does not resolve completely possible ambiguities between
prompt and \displ reconstructed leptons.

Other than the two leptons from the HNL decay, additional
leptons---real or misidentified---may satisfy the \absdxy requirements
and pass the \displ-lepton selection. In this case, criteria must
be put in place to resolve the ambiguities and correctly identify the
two leptons from the HNL decay.
To this purpose, variables such as the invariant mass of \ltwo and
\lthree, \mtwol, and the \DR\ separation between \ltwo and \lthree, \DRtwol, are found to be
effective, and with similar performance.
Therefore, the three leptons from HNL decay are identified as follows.
Among all the leptons that pass the prompt selection of
Tables~\ref{tab:electronSelection} and \ref{tab:muonSelection}, the one
with highest \pt is chosen as \lone and it is not considered anymore for the selection of \ltwo and \lthree.
Among all the selected \displ leptons, the two leptons of any
flavor with the lowest invariant mass and opposite charge are selected
as \ltwo and \lthree,
($\Pe^\pm\Pe^\mp$, $\Pe^\pm\PGm^\mp$, $\PGm^\pm\PGm^\mp$).
If there is no opposite-charge \displ leptons pair, the
event is rejected. 

This reduces background processes with
mis-identified leptons, while retaining almost full efficiency for the
signal. In the following, we will label \ltwo (\lthree) the lepton in
the pair with higher (lower) \pt.

This selection strategy correctly identifies the one prompt and two
\displ leptons in more than 99\% of signal events (99.9\% of
signal events that pass the full analysis selection, described in
Section~\ref{sec:baselinesel}).

\subsection{Secondary vertex fit and lepton extrapolation}

Once \ltwo and \lthree have been identified, we can reconstruct their
common vertex of origin, \ie the decay vertex of the HNL. This is done
by fitting the two tracks to a common point with a Kalman-filter
approach~\cite{kvfTwiki},
using the
\texttt{KalmanVertexFitter} class implemented in \texttt{CMSSW} framework.
The class returns the least-$\chi^2$ estimator of the two-track vertex
position and covariance matrix, along with the $\chi^2$ of the fit, as
an indicator of its goodness.
Figure~\ref{fig:svPulls} shows the pulls of the distance of the fitted
secondary vertex (SV) from the PV of the interaction, in three
dimensions (denoted as $R$ in the left figure) or projected on the
transverse plane (denoted as $\rho$ in the right figure). The pulls
are computed as the difference between the measured distance ($R$ or
$\rho$) and its true value from simulation, divided by the uncertainty
on the measured value. A standard deviation of about 1.2 is found from
a Gaussian fit to the pull distributions, revealing a slight
underestimation of the uncertainties.
\begin{figure}[h!]
  \centering
  \includegraphics[width=.34\textwidth]{Figures/c6/selection/leptons_fromN_fromNW_vtx_r_pull.png}
  \includegraphics[width=.34\textwidth]{Figures/c6/selection/leptons_fromN_fromNW_vtx_rho_pull.png}
  \caption{Pull distributions of the three-dimensional (left) and
    transverse (right) distance of the fitted secondary vertex from
    the PV of the interaction, for HNL masses from 1 to 4\GeV,
    all with $\mixpar=10^{-4}$. The red curve overlaid to the blue
    histogram shows a Gaussian fit to the core of the \mhnl= 2\GeV
    distribution.}
  \label{fig:svPulls}
\end{figure}
Figure~\ref{fig:svResidVsRho_all} shows the residuals of the
transverse PV--SV distance $\rho$ with respect to the true generated
value, $\Delta\rho(\mathrm{gen,rec})$ (left), and the relative residuals
$\Delta\rho(\mathrm{gen,rec})/\rho(\mathrm{gen})$ (right), as a
function of the true $\rho$, for signal events satisfying the final
signal selection that will be described in
Section~\ref{sec:baselinesel} and summarized in
Table~\ref{tab:baselinesel}. No significant biases are observed and
the tails are small: the fraction of events with
$\Delta\rho(\mathrm{gen,rec})/\rho(\mathrm{gen})$ larger than
10\%---indicated by the dashed horizontal lines in
Fig.~\ref{fig:svResidVsRho_all} (right)---is about 2.6\% for
transverse displacement $\rho$ < 0.5cm, and less than 1\% for larger
displacements.
Figure~\ref{fig:svResidVsRho_all}(A-B-C) shows the same
$\Delta\rho(\mathrm{gen,rec})$ distribution in different dilepton \pt
ranges for all flavor combinations: 15--20\GeV (A), 20--30\GeV
(B), and $>30\GeV$ (C).
Figure~\ref{fig:svResidVsRho_all}(D) shows the
$\Delta\rho(\mathrm{gen,rec})$ profiles for different flavor final
states ($\Pe\Pe$, $\Pe\PGm$, $\PGm\PGm$) for any dilepton transverse
momentum.
No large biases are observed at any dilepton \pt, nor for
any dilepton flavor.
\begin{figure}[t]
  \noindent
\makebox[\textwidth]{ 
  \includegraphics[width=.38\textwidth]{Figures/c6/selection/genvtx_recvtx_Drho_vs_rho_afterSel_zoom.pdf}
  \includegraphics[width=.38\textwidth]{Figures/c6/selection/genvtx_recvtx_RelDrho_vs_rho_afterSel_zoom.pdf}}\\
\makebox[\textwidth]{  \subfloat[]{
  \includegraphics[width=.38\textwidth]{Figures/c6/selection/pt15to20_genvtx_recvtx_Drho_vs_rho_afterSel_zoom.pdf}}
  \subfloat[]{\includegraphics[width=.38\textwidth]{Figures/c6/selection/pt20to30_genvtx_recvtx_Drho_vs_rho_afterSel_zoom.pdf}}}\\
\makebox[\textwidth]{  \subfloat[]{
  \includegraphics[width=.38\textwidth]{Figures/c6/selection/pt30toInf_genvtx_recvtx_Drho_vs_rho_afterSel_zoom.pdf}}\hspace{7mm}
  \subfloat[]{\includegraphics[width=.32\textwidth]{Figures/c6/selection/finalstates_genvtx_recvtx_Drho_vs_rho_afterSel_zoom.pdf}}}
  \caption{A) Residuals (left) and relative residuals (right) of the
    transverse distance $\rho$ between the fitted secondary vertex and
    the PV of the interaction, as a function of the true $\rho$ value
    from simulation. All lepton flavors are included.
    The red graph shows the mean value of the
    residuals in each $\rho$ bin and the uncertainty on the mean
    value. The dashed green line in the left plot indicates the
    distance between the generated primary and secondary vertices: any
    reconstructed vertex below this line would lie in the ``wrong''
    hemisphere, opposite to the direction of flight of the HNL.
    The horizontal dashed lines in the right plot delimit the region
    where $\Delta\rho(\mathrm{gen,rec})$ is less than 10\% of
    $\rho(\mathrm{gen})$. 
    The three plots A-B-C correspond to
    different values of the dilepton \pt (thus different average
    opening angles of the two leptons), for all lepton flavors:
    15--20\GeV (A), 20--30\GeV (B), and $>$30\GeV
    (C). The bottom right plot shows separate profiles for
    the three flavor final states ($\Pe\Pe$, $\Pe\PGm$, $\PGm\PGm$),
    for any dilepton \pt.
    To have enough statistics in all the $\rho$ bins, a combination of
    all the signal samples (2018 only) is used.}
  \label{fig:svResidVsRho_all}
\end{figure}
The fitted SV estimates the production vertex of \ltwo and \lthree.

\clearpage
\subsection{Event kinematics and baseline selection}
\label{sec:llbaselinesel}

Starting from all the events with one prompt lepton and two \displ
leptons forming a SV (following the definitions and strategies
described above), the kinematic properties and particle content of
typical HNL signal events are used to suppress the SM backgrounds:
most importantly, processes with one or two fake leptons,
such as top-quark production
(\ttbar, $t\PW$, $t$- and $s$-channel single top),
\(\PZ/\PGg^{\ast}+\mathrm{jets}\), and \(\PW+\mathrm{jets}\);
and processes with a real photon that converts into lepton
pairs---mostly electrons---in the detector material, such as
$\PW\PGg$ and $\PZ\PGg$. \\

The baseline selection includes the following requirements, summarized
in Table~\ref{tab:baselinesel}. Please
consider Fig.~\ref{fig:c6llsketch} as reference.

\begin{table}[h]
  \centering
  \caption{\label{tab:baselinesel} Baseline selection requirements
    applied to all data sets.}
  \begin{tabular}{l|l}
    \hline
    Variable     & Requirement       \\
    \hline
    \hline
       \DRtwol      & $<1$              \\
    \minDphi     & $>1$ rad          \\
    \mlll     & $\in [50,80]\GeV$ \\
    N. \PQb jets & $=0$              \\
    \pt (\ltwo $+$ \lthree) & $> 15 \GeV$              \\
    \costheta    & $>0.99$            \\
    SV probability & $> 0.001$              \\
    $\Delta (PV-SV)_{2D} / \sigma$& $>20$              \\ 
    resonance vetoes & applied      \\
    \hline
    \hline
  \end{tabular}
\end{table}

As explained above, \ltwo and \lthree are expected to have small
opening angle, given the small mass and relatively large momentum of
the HNL. As can be seen in
Figures.~\ref{fig:selection_electrons},~\ref{fig:selection_muons} (top-left),
the variable \DRtwol discriminates the signal from all background
processes. To retain high signal efficiency, we select events with
$\boldsymbol{\DRtwol<1}$.
\vspace{2mm}

In the absence of energetic jets in signal events, \lone is expected
to recoil at a large angle in the transverse plane from the HNL, and
thus from \ltwo and \lthree. Since both $\Dphi(\lone,\ltwo)$ and
$\Dphi(\lone,\lthree)$ are large and close in value (which may not be
true for other processes with two uncorrelated nonprompt leptons), we
apply a cut on the smaller of the two angles,
$\boldsymbol{\minDphi>1\,\mathrm{rad}}$ (see Figures.~\ref{fig:selection_electrons},~\ref{fig:selection_muons} (top-center)). 
\vspace{2mm}

The invariant mass of the three charged leptons \mlll, \mthreel, is limited
by the mass of the on-shell \PW boson. Given the relatively low
momentum carried away by the neutrino, \mlll tends to peak just
below the \PW mass, with a steep fall above 80\GeV and a larger tail
at lower masses (see Figures.~\ref{fig:selection_electrons},~\ref{fig:selection_muons} (top-center)). We select
events with $\boldsymbol{\mlll}$ values \textbf{between 50 and 80\GeV}. This requirement
proves particularly effective against the \Zgs background, where the
photon radiated by one of the leptons from the \PZ decay undergoes an
asymmetric conversion: one of the leptons receives most of the \PGg
momentum, while the other is too soft to be detected or
identified. This explains the peak at about 91\GeV in the \mthreel
spectrum. 
\vspace{2mm}

Events with a \PQb jet (see Section~\ref{sec:object}) with \pt greater
than 25\GeV are rejected, in order to substantially reduce the
background from the top processes.
Figures.~\ref{fig:selection_electrons},~\ref{fig:selection_muons} (top-right) show the number of \PQb jets with \pt greater than 25\GeV.
\vspace{2mm}

The vector sum of the momenta of \ltwo, \lthree, and the neutrino
corresponds to the momentum of the HNL, and points back exactly to the
PV. Unfortunately the total momentum of the neutrino is unknown, and
its transverse momentum, estimated by \ptmiss, has limited resolution.
The decay products of the HNL, however, are emitted at small opening
angles with respect to the HNL direction. We can thus expect the
vector sum of the \ltwo and \lthree momenta to follow closely, though
not exactly, the direction of the HNL. Figures.~\ref{fig:selection_electrons},~\ref{fig:selection_muons} (bottom-left) show a distribution of the cosine of the angle between the SV
position, which estimates the direction of flight of the HNL before
decaying, and the vector sum of the \ltwo and \lthree momenta
(``back-pointing angle''):
$\costheta = \vec{r}_{\mathrm{SV}}\cdot\vec{p}_{\mathrm{\ltwothree}}/
\left(|\vec{r}_{\mathrm{SV}}||\vec{p}_{\mathrm{\ltwothree}}|\right)$.
As expected, the signal events peak at values very close to 1,
$\boldsymbol{\costheta > 0.99}$.
Processes with two uncorrelated nonprompt leptons should exhibit a
flat \costheta distribution. The requirement on \DRtwol, however,
selects events with two relatively close-by leptons, therefore the vector sum of the pT of the two displaced leptons makes a small angle with the vector drawn from the PV to SV. In principle this is a genuine property of the boosted displaced HNL, however the angular cuts on the two leptons biases the background to have small angle as well.For this reason, all
background processes exhibit a peaking structure, randomly at $-1$ or
$+1$.
\vspace{2mm}

The quality of the secondary vertex (SV probability) necessarily correlates with the precision of the trajectories from \ltwo and \lthree as well as how close they are at the intersection point. It is represented as a probability based on the maximum likelihood fit from the kinematic vertex fitter. One can observe that for low probability values, the background dominates. This motivates a requirement of the \textbf{probability to be larger than 0.001}. Figures.~\ref{fig:selection_electrons},~\ref{fig:selection_muons} (bottom-center) show the distribution of the SV probability.
\vspace{2mm}

The quality of the SV fit is used as well as discriminating variable against random vertices. The significance of the distance between PV and SV is shown in Figures.~\ref{fig:selection_electrons},~\ref{fig:selection_muons} (bottom-center). The requirement of the \textbf{significance to be larger than 20} is applied.
\vspace{2mm}

The dilepton (\ltwo, \lthree) transverse momentum  is required to be
larger than 15\GeV, since the low-\pt region is heavily dominated by
the nonprompt lepton background.
The dilepton mass \mtwol distribution has an upper cut that
corresponds to the largest mass among all the considered signal
samples. Due to the presence of the neutrino in the final state, the
mass is always lower than the HNL mass.
Figures.~\ref{fig:selection_electrons},~\ref{fig:selection_muons} (bottom-right) show the dilepton transverse
momentum spectra. 
\vspace{2mm}

In addition to the selections mentioned above, numerous dilepton
resonances within the search region are removed.
In case of \ltwo and \lthree having the same flavor and opposite
charges and with the transverse position of the dilepton vertex
$\Deltwod<1.5\cm$, the following resonance masses (\mtwol) are
removed: 
\JPsi $(3.10 \pm 0.08$\GeV), \Pgy $(3.69 \pm 0.08$\GeV), $\Omega$
$(0.78 \pm 0.08$\GeV), and $\phi$ $(1.02 \pm 0.08$\GeV).
In case \lone and \ltwo/\lthree have the same flavor and opposite
charges, events are removed if \mlonetwo or \mlonethree are in the
ranges listed above, or consistent with other higher-mass resonances,
including \PgUa $(9.46 \pm 0.08$\GeV), \PgUb $(10.02 \pm 0.08$\GeV),
\PgUc $(10.36 \pm 0.08$\GeV), and \PZ $(91.19 \pm 10.00$\GeV).
\begin{center}
\framebox{
       \parbox{\textwidth}{
{\footnotesize
The convention for the legends in all following plots in Chapter~\ref{Chapter6} backgrounds
will be grouped into macro-categories of histograms as follows.
When there are only MC predictions:
\begin{itemize}
\setlength\itemsep{-0.2em}
\item {\color{gray}MC nonprompt DF:} processes that give rise to two nonprompt
  leptons, such as \PW + jets, \ttbar + jets, single-top. To be defined as DF it has to be defined "double-fake" according to the definition in Sec.~\ref{sec:llbackground};
\item {\color{gray}MC nonprompt SF:} processes that give rise to one or two nonprompt
  leptons, such as \PW + jets, \ttbar + jets, single-top and DY+ jets. To be defined as SF it has to be defined "single-fake" according to the definition in Sec.~\ref{sec:llbackground};
\item {\color{gray}Conversions:}  processes with photon conversions, such as Z$\PGg$ and W$\PGg$;
\item {\color{gray}$\PZ\PGg^{\ast}$:} we consider DY events with prompt leptons;
\item {\color{gray}Other:} processes like diboson and triboson.
%\item MuonEG.
\end{itemize}
When there are both MC and data-driven predictions:
\begin{itemize}
\setlength\itemsep{-0.2em}
\item {\color{gray}Nonprompt DF/SF:} data-driven predictions that are explained in Sec.~\ref{sec:llbackground};
\item {\color{gray}Conversions:}  processes with photon conversion, like Z$\PGg$ and W$\PGg$, and DY;
\item {\color{gray}Other:} processes like diboson and triboson.
%\item MuonEG.
\end{itemize}
}
}
}
\end{center}

\begin{figure}[h]
\noindent
\makebox[\textwidth]{
  \includegraphics[width=.28\textwidth]{Figures/c6/selection/18/e_DeltaR_l2_l3__0.pdf}
  \includegraphics[width=.28\textwidth]{Figures/c6/selection/18/e_minDeltaphil1_other__0.pdf}
  \includegraphics[width=.28\textwidth]{Figures/c6/selection/18/e_M_lll__0.pdf}
  \includegraphics[width=.28\textwidth]{Figures/c6/selection/18/e_numberofb_jets__0.pdf}}\\
\makebox[\textwidth]{
  \includegraphics[width=.28\textwidth]{Figures/c6/selection/18/e_cosSVposl2_l3dir__0.pdf}
  \includegraphics[width=.28\textwidth]{Figures/c6/selection/18/e_vertexchi2_probabilityl2_l3__0.pdf}
  \includegraphics[width=.28\textwidth]{Figures/c6/selection/18/e_sigmaDeltaPV_SV_2D__0.pdf}
  \includegraphics[width=.28\textwidth]{Figures/c6/selection/18/e_sum_Pt_L2L3__0.pdf}}\\
  \caption{Distributions of the event selection's variables listed in
    Table~\ref{tab:baselinesel}. Simulated backgrounds (shaded histograms, stacked),
    using the 2018 MC samples, 
    after the selection of the three leptons \lone, \ltwo, and \lthree,
    in \eex final states.
    Signal models for different values of \mhnl and \mixpar are shown
    as empty histograms.}
  \label{fig:selection_electrons}
\end{figure}

\begin{figure}[h]
\noindent
\makebox[\textwidth]{
  \includegraphics[width=.28\textwidth]{Figures/c6/selection/18/mu_DeltaR_l2_l3__0.pdf}
  \includegraphics[width=.28\textwidth]{Figures/c6/selection/18/mu_minDeltaphil1_other__0.pdf}
  \includegraphics[width=.28\textwidth]{Figures/c6/selection/18/mu_M_lll__0.pdf}
  \includegraphics[width=.28\textwidth]{Figures/c6/selection/18/mu_numberofb_jets__0.pdf}}\\
\makebox[\textwidth]{
  \includegraphics[width=.28\textwidth]{Figures/c6/selection/18/mu_cosSVposl2_l3dir__0.pdf}
  \includegraphics[width=.28\textwidth]{Figures/c6/selection/18/mu_vertexchi2_probabilityl2_l3__0.pdf}
  \includegraphics[width=.28\textwidth]{Figures/c6/selection/18/mu_sigmaDeltaPV_SV_2D__0.pdf}
  \includegraphics[width=.28\textwidth]{Figures/c6/selection/18/mu_sum_Pt_L2L3__0.pdf}}\\
  \caption{Distributions of the event selection's variables listed in
    Table~\ref{tab:baselinesel}. Simulated backgrounds (shaded histograms, stacked),
    using the 2018 MC samples, 
    after the selection of the three leptons \lone, \ltwo, and \lthree,
    in \mmx final states.
    Signal models for different values of \mhnl and \mixpar are shown
    as empty histograms.}
  \label{fig:selection_muons}
\end{figure}

The following plots are a summary view of the entire selection
(~\ref{tab:baselinesel}) showing two different kind of "cutflows". The
first set, fig~\ref{fig:cutflow1}, is the classic cutflow where each
bin displays the number of left events after each cut and it goes in a
"cascade" way (one after the other). The second one,
fig~\ref{fig:cutflow2}, is a $N-1$ plot where each bin has the number
of events when that particular cut is not applied; in this way it is
possible to appreciate which is the real impact of each cut no matter
its position in the even selection list. 

As it is seen in Fig.~\ref{fig:cutflow1}, the most effective background rejection is achieved with the \mthreel selection. It is in particular effective in \eex final states as it rejects DY events where at least one electron comes from conversion. In such events \mthreel populates the region of $M_{\PZ}$ which remains outside the [40-80]\GeV signal region.

\begin{figure}[h]
  \noindent
  \makebox[\textwidth]{
  \includegraphics[width=.34\textwidth]{Figures/c6/selection/18/e_cutflow__0.pdf}
  \includegraphics[width=.34\textwidth]{Figures/c6/selection/18/mu_cutflow__0.pdf}}\\
  \makebox[\textwidth]{
   \includegraphics[width=.34\textwidth]{Figures/c6/selection/eff_e_cutflow__0.pdf}
  \includegraphics[width=.34\textwidth]{Figures/c6/selection/eff_mu_cutflow__0.pdf}}
  \caption{In the top, cutflow plots showing the full selection applied in simulated backgrounds using the 2018 MC samples,
    in \eex (left) and \mmx (right) final states. In the bottom, cutflow plots but showing the
    efficiency of the selection for the single cut (each bin is $yield_{bin}/yield_{no\:selection}$).
    Signal models for different values of \mhnl and \mixpar
    are shown as empty histograms.}
  \label{fig:cutflow1}
\end{figure}

\begin{figure}[t]
  \noindent
  \makebox[\textwidth]{
 \includegraphics[width=.34\textwidth]{Figures/c6/selection/18/e_cutflow_n_1__0.pdf}
  \includegraphics[width=.34\textwidth]{Figures/c6/selection/18/mu_cutflow_n_1__0.pdf}}\\
  \makebox[\textwidth]{
   \includegraphics[width=.34\textwidth]{Figures/c6/selection/eff_e_cutflow_n_1__0.pdf}
  \includegraphics[width=.34\textwidth]{Figures/c6/selection/eff_mu_cutflow_n_1__0.pdf}}
  \caption{In the top, $N-1$ plots showing the real impact of each cut of the full selection in simulated backgrounds using the 2018 MC samples,
    in \eex (left) and \mmx (right) final states. The last bin has the number of events when all the cuts are applied.
    In the bottom, $N-1$ plots but showing the impact in percentage of the single cut (each bin is $yield_{bin}/yield_{full\:selection}$).
    Signal models for different values of \mhnl and \mixpar
    are shown as empty histograms.}
  \label{fig:cutflow2}
\end{figure}
\vspace{2mm}

%One key feature of an analysis involving displaced vertices is the
%high sensitivity and low statistics of regions with large
%displacements.
Figure~\ref{fig:reco_displ} shows the transverse
distance of the SV from the PV, \Deltwod, for events
passing the full baseline selection.
\begin{figure}[h]
  \noindent
  \makebox[\textwidth]{
  \includegraphics[width=.28\textwidth]{Figures/c6/selection/18/e_DeltaPV_SV_2D_zomm__final.pdf}
  \includegraphics[width=.28\textwidth]{Figures/c6/selection/18/mu_DeltaPV_SV_2D_zomm__final.pdf}
  \includegraphics[width=.28\textwidth]{Figures/c6/selection/18/e_M_ll_l2_l3_zoom__final.pdf}
  \includegraphics[width=.28\textwidth]{Figures/c6/selection/18/mu_M_ll_l2_l3_zoom__final.pdf}}
  \caption{\Deltwod (left), and $M_{\ltwo \lthree}$ (right) in events
    passing the baseline selection of Table~\ref{tab:baselinesel}}
  \label{fig:reco_displ}
\end{figure}

In the region of large displacement, the sensitivity of the analysis
greatly increases, with a relatively low abundance of nonprompt lepton
background and being virtually free of prompt lepton background
contributions. This represents the key feature of the analysis.
On the contrary, signals with higher masses are short-lived \hnl
decaying in low displacement regions, where both the prompt and
nonprompt lepton contributions are large, similarly to the previous
CMS search presented in Chapter~\ref{Chapter5}.
Although the sensitivity of the analysis strongly depends on the
secondary vertex displacement, no selection is applied on this
variable. This is due to the fact that both small and large
displacement regions are interesting for the search. The small
displacement region represents the region of the previous CMS search
and can be used to cross-check its results, while the large
displacement region provides the highest sensitivity to long-lived
HNLs. 
Thus, the final signal region will be divided into bins of different
displacement, in addition to the invariant mass of \ltwo and \lthree,
\mtwol (fig.~\ref{fig:reco_displ}).\\

The residual background, dominated by \Xg,
$\PZ/\PGg^{\ast}+\mathrm{jets}$, and top processes, accumulates at low
displacement. The HNL signal models populate higher \Deltwod bins, depending on \mhnl and \mixpar. In particular, \Deltwod
is found to provide the best discrimination between signal and
background, and among different signal models.
Therefore we select four event categories, based on the value of
\Deltwod, with different sensitivities for different signal
scenarios:
$\Deltwod<0.5$\cm (fully background dominated), $0.5<\Deltwod<1.5$\cm,
$1.5<\Deltwod<4$\cm and $\Deltwod>4$\cm.\\
In addition, the events are split into 3 different final states. In case 
the muon coupling is been probed the 3 final states are $\mu\mu\mu$,  $\mu^{\mp}\mu^{\pm} e$ 
and $\mu^{\pm}\mu^{\pm}e$; in case of electron coupling the 3 final states
are: $eee$,  $e^{\mp}e^{\pm}\mu$ and $e^{\pm}e^{\pm}\mu$. This latter categorization
is necessary to provide separated limits for the case of LNC ($\mu^{\mp}\mu^{\pm} e$ and $e^{\mp}e^{\pm}\mu$)
and for the case LNV ($\mu^{\pm}\mu^{\pm} e$ and $e^{\pm}e^{\pm}\mu$), (see Sec.~\ref{sec:lnv_lnc}).
The different categories are also characterized by different
background levels and composition, thus providing different
sensitivities.

Finally we notice that the top background processes, where \ltwo and
\lthree are typically produced in the semi-leptonic decay of $B$
hadrons, populate a region of $\mtwol<4$\GeV, and are basically absent
at higher dilepton masses. Therefore we further split the data into
two categories: $\mtwol<4$\GeV and $\mtwol>4$\GeV. 
\vspace{12mm}

Due to the lack of statistic (in background) at large displacement bins for high mass region, it has been decided to reduce the number of bins 
in the following way:

\begin{itemize}
\setlength\itemsep{-0.5em}
\item for $\mtwol<4$\GeV
  \begin{itemize}
    \setlength\itemsep{-0.4em}
    \item $\Deltwod<0.5$\cm
    \item $0.5<\Deltwod<1.5$\cm
    \item $1.5<\Deltwod<4$\cm
    \item $\Deltwod>4$\cm
  \end{itemize}  
\item for $\mtwol>4$\GeV
\begin{itemize}
    \setlength\itemsep{-0.4em}
    \item $\Deltwod<0.5$\cm
    \item $\Deltwod>0.5$\cm
  \end{itemize}  
\end{itemize}

\comm{
All the distributions previously shown contain background predictions using only MC samples and 
do not have any data points.
At this stage of the analysis process we do not want to show any data-yield since it is not clear yet if 
the background estimation using only MC can predict well the behavior of the data. To ensure so, 
we should first check in regions orthogonal wrt signal region if the MC can predict the data; if it is not the case 
it is important to find a way to predict the contribution of the background using directly data: data-driven estimation.

All the distributions previously shown contain background predictions using only the 2018 MC samples; 2016 and 2017 can be found in Appendix~\ref{app:appendix_MC_2017_2018}.}
\clearpage
%%%%%%%%%%%%%%%%%%%%%%%%%%%%%%%%%%%%%%%%%%%%%%%%%%%%
%%%%%%%%%%%%%%%%%%%%%%%%%%%%%%%%%%%%%%%%%%%%%%%%%%%%
%%%%%%%%%%%%%%%%%%%%%%%%%%%%%%%%%%%%%%%%%%%%%%%%%%%%
\section{Background estimation}\label{sec:llbackground}
\subsection{Background composition}\label{sec:llcomposition}

The main SM background processes mimicking the signal final states can be
classified as follows.
\begin{itemize}
\item \textbf{``Fake'' nonprompt leptons (\Pe or \PGm):}
  muons from light-flavor mesons that decay in flight, or electrons from
  unidentified conversions of photons in jets
  can mimic the \displ leptons from the HNL decay, \ltwo and
  \lthree.
  This background is dominated by top-quark and Drell--Yan processes,
  and is measured with the data-driven method described in
  Sec.~\ref{sec:llfakeBkg}.\\
  Fake nonprompt lepton backgrounds can be split into two categories:
  \begin{itemize}
  \item single fake nonprompt leptons:
    a single reconstructed lepton produced via one of the above
    mechanisms, \eg the (semi-)leptonic decay of a meson.
    Multiple single-fake leptons can be found in an event,
    produced in the independent decays of different mesons.
    In this case, the probabilities to select different fake nonprompt
    leptons for the analysis can be treated as uncorrelated, as
    explained in Sec.~\ref{sec_llfakelepton}.
  \item double fake nonprompt leptons:
    a pair of reconstructed leptons produced in the decay chain of the
    same hadron, typically a $B$ meson (\eg via
    $\PQb\to\ell^-\bar{\PGn}_{\ell}\PQc\to\ell^-\bar{\PGn}_{\ell}\ell^{\prime+}\PGn_{\ell^{\prime}}\PQs$)
    or a quarkonium state (\eg $\JPsi\to\ell^-\ell^+$).
    In such decays the two leptons are manifestly not independent and
    their selection probabilities are correlated.
    Therefore the lepton pair must be treated as a single system, as
    explained in Sec.~\ref{sec:doubleFakeBkg}.
  \end{itemize}
\item \textbf{Photon conversions:}
  \PW or \PZ production processes can
  contribute to the trilepton final state when an initial- or
  final-state photon is radiated and interacts with the material of
  the beam pipe or detector, converting into a pair of electrons.
  These processes are sometimes referred to as ``external''
  conversions, as opposed to ``internal'' conversions of virtual
  photons: $\PGg^\ast\to\ell^-\ell^+$ (\lept can be a muon as well, in
  this case).
  If the photon conversion is strongly asymmetric---\ie one electron
  inherits most of the photon momentum and the other has very low
  \pt---, only the more energetic electron is reconstructed and
  identified. E.g., in $\PZ\to\ell^-\ell^+\PGg$ events with an
  asymmetric conversion of the FSR photon, the final state is
  characterized by three leptons with an invariant mass spectrum
  peaking at the nominal \PZ mass.
  This background gives minor contributions
  to final states with at least a nonprompt electron.
  It is estimated from simulation and verified in control regions with
  three leptons peaking at the \PZ mass (Sec.~\ref{sec:conversion}).
  Such conversions are considered as a
  source of a background, as well as an excellent probe for nonprompt
  lepton efficiency
 measurement, in many searches for displaced vertices.

\item Multibosons and other rare processes with prompt
  leptons:
  processes with multiple bosons (\PW, \PZ, \PGg, \PH) and/or multiple
  top quarks can give rise to final states with three or more prompt
  leptons.
  The largest contribution in this category comes from $\PW\PZ$ and
  $\PZ\PZ$ diboson production, while processes involving more than two
  bosons or top quarks have very small production rates, and can be
  further suppressed by the lepton and \PQb-jet vetoes.
  These processes constitute an almost negligible background and are
  estimated from simulation.
\end{itemize}
 
\subsection{Fake-lepton background}\label{sec_llfakelepton}
The fake-lepton background is estimated using the ``tight-to-loose''
method.\\
A general description of the fake-rate method can be found in
Sec.~\ref{sec:tight_loose_method}.\\

Here ``tight'' refers to the tight identification criteria for
\displ leptons (\ltwo, \lthree) listed in
Tables~\ref{tab:electronSelection} and \ref{tab:muonSelection},
while  ``loose'' refers to the corresponding looser selections
(\ie fakeable objects).

The main difference with respect to the HNL-prompt analysis
(Chapter~\ref{Chapter5}) comes from the different source of fake
leptons; here fake leptons are typically found in the proximity of a
jet which is in common for both \ltwo and \lthree. Each loose
lepton is therefore associated to the jet of $\pt>10\GeV$ closest in
\DR, only if $\DR<0.4$.
If two loose leptons 
are either associated to the same jet or have $\DR<0.45$, they are considered double-fake
leptons and treated as a single dilepton system.
In all other cases --- \ie, if the two loose leptons are associated to
different jets and the $\DR>0.45$ ---,
the loose leptons are considered single fakes and (if more than one)
treated independently.

An important clarification has to be made regarding the estimation of 
the fake-lepton background. This prediction only accounts for fake
nonprompt (``displaced'', \ltwothree) leptons. The contribution from
fake prompt (\ie \lone) leptons is expected to be very small, on
account of the much higher probability for fake \ltwo and \lthree to
pass the tight \displ selection. This assumption has been verified in
MC using the information at generated level (MC truth); at the
beginning of the selection only 1\textperthousand\ background
events have nonprompt \lone, at the end of the selection all \lone
leptons are prompt leptons. 
For this reason, the contribution from fake \lone is neglected.

\subsubsection{Estimation of double-fake background}
\label{sec:doubleFakeBkg}

Events in the application region are classified as double-fake
background if the two nonprompt leptons are either associated to the same
jet or the distance between them is $\DR<0.45$, and at least one of them is loose-not-tight.
In the first case (same jet associated), \ptm is estimated using the \pt of the associated
jet, $\ptm=\ptj$, calibrated using jet energy corrections (JEC, see
Sec.~\ref{sec:jet}) after subtracting the \pt of the two leptons:
\begin{linenomath}
  \begin{equation}
    \label{eq:ptjet}
    \ptm = \ptj =
    \left(\pt^{\mathrm{raw\,jet}}-\pt^{\ltwo}-\pt^{\lthree}\right)\times\mathrm{JEC}
    \;+\; \pt^{\ltwo} \;+\; \pt^{\lthree}.
  \end{equation}
\end{linenomath}
In this formula all the \pt are 4-vectors.
Similar is the case when the distance between them is $\DR<0.45$:
\begin{linenomath}
  \begin{equation}
    \label{eq:ptjet_45}
    \ptm = \ptj = p^{jet,2}_T + p^{jet,3}_T 
  \end{equation}
\end{linenomath}
where the $p^{jet,i}$ is the jet associated to the lepton \emph{i} and then corrected using the JEC right correction:
\begin{linenomath}
  \begin{equation}
    \label{eq:ptjet_45_single}
    p^{jet,i} = \left(\pt^{\mathrm{raw\,jet_i}}-\pt^{\ell_{i}}\right)\times\mathrm{JEC}\;+\; \pt^{\ell_i} .
  \end{equation}
\end{linenomath}

Double fake rates (\dfr) are parametrized and applied as a
function of the \pt of the associated jet:
$\dfr=\dfr(\ptj)$.
The definition of the measurement region is similar to that of the
signal region, with the following differences:
\begin{itemize}
\setlength\itemsep{-0.2em}
\item the two nonprompt leptons, \ltwo and \lthree, are selected with
  the loose identification criteria (see
  Tables~\ref{tab:electronSelection} and \ref{tab:muonSelection})
  and must be either associated to the same jet or have $\DR<0.45$;
\item a \PQb-tagged jet with $\pt>25$\GeV must be present (not
  necessarily the same that \ltwo and \lthree are associated to).
\end{itemize}
The requirement of a \PQb-jet enriches the sample in top-quark
backgrounds, which are the main source of double-fake leptons, see Table~\ref{tab:measurement_sel}.
Events in this measurement region have two loose (\emph{fakeable object}) nonprompt leptons, of
which zero, one, or both can be tight.
\begin{table}[h!]
  \centering
  \caption{\label{tab:measurement_sel} Measurement region selection requirements
    for double fake.}
    \begin{tabular}{l|l}
    \hline
    Variable     & Requirement       \\
    \hline
    \hline
    N. \PQb jets & $\geq 1$              \\
    \mtwol in $\Pe\Pe\Pe$& $> 0.5$\GeV              \\ 
    resonance vetoes & applied      \\
    \hline
   %% \mthreel  and  N. \PQb jets  & ($<50$\GeV or $>80$\GeV N.\PQb jets= 0) OR  (N.\PQb jets $\neq$ 0)\\
     \hline
     \mthreel in $\Pe\Pe\Pe$ and $\PGm\PGm\Pe$ OS & $<80$\GeV or $>101$\GeV \\
      \mtwol in $\Pe\Pe\Pe$& $> 0.5$\GeV              \\ 
    \hline
    \hline 
  \end{tabular}
\end{table}

The double fake rate \dfr is defined as the fraction of events in this
region where both nonprompt leptons are tight.
Figure~\ref{fig:dFR_ptjet_16}(top) shows
the measured values of \dfr for events with two electrons, an electron
and a muon, and two muons, respectively, as a function of \ptj.
Figure~\ref{fig:dFR_ptjet_16}(bottom) shows
the measured values of \dfr for events with two electrons, an electron
and a muon, and two muons, respectively, as a function of \Deltwod. The \Deltwod bins are the four bins chosen for the SR in order to evaluate 
with is the dependance of the \dfr wrt the displacement of the SV of the \hnl decay. 

\begin{figure}[h]
  \noindent
\makebox[\textwidth]{
  \includegraphics[width=.35\textwidth]{Figures/c6/backgrounds/FR/dFR/16/c_LeptonPt___eta_FR1.pdf}
  \includegraphics[width=.35\textwidth]{Figures/c6/backgrounds/FR/dFR/17/c_LeptonPt___eta_FR1.pdf}
  \includegraphics[width=.35\textwidth]{Figures/c6/backgrounds/FR/dFR/18/c_LeptonPt___eta_FR1.pdf}}
  \noindent
\makebox[\textwidth]{\includegraphics[width=.35\textwidth]{Figures/c6/backgrounds/FR/dFR/16/c_displacement_SR___eta_FR1.pdf}
  \includegraphics[width=.35\textwidth]{Figures/c6/backgrounds/FR/dFR/17/c_displacement_SR___eta_FR1.pdf}
  \includegraphics[width=.35\textwidth]{Figures/c6/backgrounds/FR/dFR/18/c_displacement_SR___eta_FR1.pdf}}
  \caption{Double fake-lepton rates measured in 2016/17/18 (left,
    middle, right) data as a function of \ptj (top) and \Deltwod (bottom),
    for events with two electrons (black), with
    an electron and a muon (red), and with two muons (blue).}
  \label{fig:dFR_ptjet_16}
\end{figure}

The application region for double-fake lepton backgrounds (dFR)
is defined using the same selection as the signal region (see
Table~\ref{tab:baselinesel}), but with one or two loose-not-tight
nonprompt leptons that either be associated to the same jet or have $\DR<0.45$. 
The background from double-fake leptons in the signal region is
estimated from the event yields in the application region dFR using
the formula
\begin{linenomath}
  \begin{equation}
    \label{eq:dFRbkg}
    N_{\mathrm{dFR}}^{\mathrm{sig}} ~=~ 
    \sum_i\frac{\dfr(\ptjs{i})}{1-\dfr(\ptjs{i})}\;,
  \end{equation}
\end{linenomath}
where the sum runs over all the events in dFR, with jet transverse momentum \ptcs{i}.

\subsubsection{Estimation of single-fake background}
\label{sec:singleFakeBkg}

The measurement of the single-fake background exaclty follows the same
procedure as the one explained in Sec.~\ref{sec:singleFR}. Thus, to
avoid repetitions, only the final measured FR are presented here; for
additional plots see Appendix~\ref{AppendixA}, Sec.~\ref{app:sfr}.

Figure~\ref{fig:sfr_data} shows the measured
values of \sfr for electrons and muons in simulation and in data,
respectively, as a function of \ptc and in different \abseta bins.


\begin{figure}[h]
  \centering
  \includegraphics[height=6cm, width=6.4cm]{Figures/c6/backgrounds/FR/sFR/data/electrons.pdf}
  \includegraphics[height=6cm, width=6.4cm]{Figures/c6/backgrounds/FR/sFR/data/muons.pdf}\\
  \caption{Single fake rates measured in 2016 data for electrons (left) and
    muons (right) as a function of \ptc, for \abseta ranges $[0,0.8]$
    (top), $[0.8,1.479]$ (middle), and $[1.479,2.5]$ (bottom).}
  \label{fig:sfr_data}
\end{figure}

Figures~\ref{fig:sfr_dxy} show that \sfr values are fairly independent
of the lepton \dxy.
\begin{figure}[h]
  \centering
  \includegraphics[width=.48\textwidth]{Figures/c6/backgrounds/FR/sFR/QCD/dxy_ele_eta1_FR2.pdf}
  \includegraphics[width=.48\textwidth]{Figures/c6/backgrounds/FR/sFR/QCD/dxy_mu_eta1_FR2.pdf}\\
  \includegraphics[width=.48\textwidth]{Figures/c6/backgrounds/FR/sFR/QCD/dxy_ele_eta2_FR2.pdf}
  \includegraphics[width=.48\textwidth]{Figures/c6/backgrounds/FR/sFR/QCD/dxy_mu_eta2_FR2.pdf}\\
  \includegraphics[width=.48\textwidth]{Figures/c6/backgrounds/FR/sFR/QCD/dxy_ele_eta3_FR2.pdf}
  \includegraphics[width=.48\textwidth]{Figures/c6/backgrounds/FR/sFR/QCD/dxy_mu_eta3_FR2.pdf}
  \caption{Single fake rates obtained in 2016 simulation for electrons
    (left) and muons (right) as
    a function of \dxy, for \abseta ranges $[0,0.8]$ (top),
    $[0.8,1.479]$ (middle), and $[1.479,2.5]$ (bottom).}
  \label{fig:sfr_dxy}
\end{figure}
\clearpage
\subsubsection{Application of the fake-lepton estimation methods}\label{sec:frChecks}
To check the accuracy of the background estimates obtained with the
single- and double-fake-lepton methods, \emph{closure tests} are
performed in simulation, as well as in control data samples. 
The prediction of the nonprompt background is obtained as the
combination of the two contributions described above. The total
nonprompt background is thus estimated as follows:
\begin{linenomath}
  \begin{align*}
    %\label{eq:dFRbkg_plus_sFRbgk}
    N_{\mathrm{nonprompt}}^{\mathrm{sig}} & =~ 
    \sum_i\frac{\sfr(\sigeta_i,\,\ptcs{i})}{1-\sfr(\sigeta_i,\,\ptcs{i})}
    +~
    \sum_i\frac{\dfr(\sigetaj_i,\,\ptjs{i})}{1-\dfr(\sigetaj_i,\,\ptjs{i})}\;,
  \end{align*}
\end{linenomath}

\paragraph{Closure test in simulation}
This test is performed using \ttbar MC samples, with the
following selection: one prompt lepton and two OS \displ leptons and no \PQb jets.\\
The results are shown in Fig.~\ref{fig:mcClosure} for 2016. The MC
events are 
classified based on the presence of one or two single-fake leptons
(blue histograms) or a double-fake lepton pair (green histograms),
according to the definitions given in the introduction to
Sec.~\ref{sec:llcomposition}. MC events are compared with the estimates
obtained from the fake-rate methods described in
Sec.~\ref{sec:singleFakeBkg}
and \ref{sec:doubleFakeBkg}, combined in the red histogram.
We notice that the double-fake background contributes mostly to
the low-\DRtwol and low-\mtwol regions. We can conclude that the
fake-rate background estimations correctly reproduce the expected
single- and double-fake backgrounds, both the overall yields and the
kinematic spectra.
\begin{figure}[h]
  \centering
  \includegraphics[width=.35\textwidth]{Figures/c6/backgrounds/FR/closureTest/MC/ttbar_DeltaR_st_mu___0.pdf}
  \includegraphics[width=.35\textwidth]{Figures/c6/backgrounds/FR/closureTest/MC/ttbar_Mll_naked_mu___0.pdf}\\
  \includegraphics[width=.35\textwidth]{Figures/c6/backgrounds/FR/closureTest/MC/ttbar_dxy_t_mu___0.pdf}
  \includegraphics[width=.35\textwidth]{Figures/c6/backgrounds/FR/closureTest/MC/ttbar_2D_Delta_PV_SV_mu___0.pdf}
  \caption{Distributions of \DRtwol (top left), \mtwol (top right),
    \lthree \dxy (bottom left), and \Deltwod (bottom right) in the
    2016 \ttbar MC samples,
    comparing the MC events containing SF leptons
    and DF leptons with the
    prediction, for the sum of all final states.
    The blue and green histograms are stacked, the black points show
    their sum.}
  \label{fig:mcClosure}
\end{figure}
\clearpage
\paragraph{Closure test in data sample: sideband control region}\label{sec:llsidebandcr}
This control region is defined in Table~\ref{tab:side_band_sel}.
Data in this region are compared to our data-driven (nonprompt DF and
SF) and MC-based (the other) background estimates.

Results for the Run2 data are shown in
Fig.~\ref{fig:sb_sr_Ctrl}.\\
Additional plots divided in separated years can be found in
Appendix~\ref{app:appendix_DFR_SFR_2017_2018}.
\begin{table}[h]
  \centering
{\small
  \caption{\label{tab:side_band_sel} Control region selection requirements
    applied to all data sets.}
    \begin{tabular}{l|l}
    \hline
    Variable     & Requirement       \\
    \hline
    \hline
    \DRtwol      & $<1$              \\
    \minDphi     & $>1$ rad          \\ 
    \mlll     & $< 50$ OR $> 80\GeV$ \\
    N. \PQb jets & $=0$              \\
    (\ltwo $+$ \lthree) \pt & $> 15 \GeV$              \\
    \costheta    & $>0.99$            \\
    \mtwol& $<20\GeV$              \\ 
    SV probability & $> 0.001$              \\
    $\sigma$ \Deltwod& $>20$              \\ 
    resonance vetoes & applied      \\
    \hline
     \hline
     \mthreel in $\Pe\Pe\Pe$ and $\PGm\PGm\Pe$ OS & $<50$\GeV or $>101\GeV$ \\
    \hline
    \hline 
  \end{tabular}
}
\end{table}

\begin{figure}[h]
  \noindent
  \makebox[\textwidth]{
  \includegraphics[width=.55\textwidth]{Figures/c6/backgrounds/FR/closureTest/Data/onlySideBand/M-6_V-0p00202484567313_mu_muo_datacard_combined_SR.pdf}
  \includegraphics[width=.55\textwidth]{Figures/c6/backgrounds/FR/closureTest/Data/onlySideBand/M-6_V-0p00202484567313_e_ele_datacard_combined_SR.pdf}}\\
\noindent
  \makebox[\textwidth]{
  \includegraphics[width=.35\textwidth]{Figures/c6/backgrounds/FR/closureTest/Data/onlySideBand/M-6_V-0p00202484567313_mu_muo_mass3_datacard_combined_mass3.pdf}
  \includegraphics[width=.35\textwidth]{Figures/c6/backgrounds/FR/closureTest/Data/onlySideBand/M-6_V-0p00202484567313_mu_muo_mass_datacard_combined_massl2l3.pdf}
   \includegraphics[width=.35\textwidth]{Figures/c6/backgrounds/FR/closureTest/Data/onlySideBand/M-6_V-0p00202484567313_mu_muo_disp_datacard_combined_displacement.pdf}}\\
\noindent
  \makebox[\textwidth]{
  \includegraphics[width=.35\textwidth]{Figures/c6/backgrounds/FR/closureTest/Data/onlySideBand/M-6_V-0p00202484567313_e_ele_mass3_datacard_combined_mass3.pdf}
  \includegraphics[width=.35\textwidth]{Figures/c6/backgrounds/FR/closureTest/Data/onlySideBand/M-6_V-0p00202484567313_e_ele_mass_datacard_combined_massl2l3.pdf}
   \includegraphics[width=.35\textwidth]{Figures/c6/backgrounds/FR/closureTest/Data/onlySideBand/M-6_V-0p00202484567313_e_ele_disp_datacard_combined_displacement.pdf}}\\
  \caption{At the top, search region-like distributions with the selection of the
    sideband control region, in data and background
    predictions, in final states with more than two muons (left) and
    more than two electrons (right). In the middle and at the bottom, distributions of three-lepton invariant mass (left), \mtwol
    (middle), and \Deltwod (right) in the combined Full-Run2 sideband control region,
    for the sum of all final states with muon couplings (middle) and
    electron couplings (bottom).}
  \label{fig:sb_sr_Ctrl}
\end{figure}

\paragraph*{Closure test in data sample: \PQb jet control region}
This control region is defined in Table~\ref{tab:ttbar_sel}.
Data in this region are compared to our data-driven (nonprompt DF and
SF) and MC-based (the other) background estimates.\\
Results for the Run2 data  are shown in
Fig.~\ref{fig:sb_ttbar_Ctrl}.\\
Additional plots divided in separated years can be found in
Appendix~\ref{app:appendix_DFR_SFR_2017_2018}.

\begin{table}[h]
  \centering
{\small
  \caption{\label{tab:ttbar_sel} \PQb jet control region selection requirements
    applied to all data sets.}
   \begin{tabular}{l|l}
    \hline
    Variable     & Requirement       \\
    \hline
    \hline
    \DRtwol      & $<1$              \\
    \minDphi     & $>1$ rad          \\ 
    %%\mthreel     & $< 50$ OR $> 80\GeV$ \\
    N. \PQb jets & $\neq 0$              \\
    (\ltwo $+$ \lthree) \pt & $> 15 \GeV$              \\
    \costheta    & $>0.99$            \\
    \mtwol& $<20\GeV$              \\ 
    SV probability & $> 0.001$              \\
    $\sigma$ \Deltwod& $>20$              \\ 
    resonance vetoes & applied      \\
    \hline
     \hline
     \mlll in $\Pe\Pe\Pe$ and $\PGm\PGm\Pe$ OS & $<50$\GeV or $>101\GeV$ \\
    \hline
    \hline 
  \end{tabular}
}
\end{table}

\begin{figure}[h]
   \noindent
  \makebox[\textwidth]{
  \includegraphics[width=.55\textwidth]{Figures/c6/backgrounds/FR/closureTest/Data/onlyBjet/M-6_V-0p00202484567313_mu_muo_datacard_combined_SR.pdf}
  \includegraphics[width=.55\textwidth]{Figures/c6/backgrounds/FR/closureTest/Data/onlyBjet/M-6_V-0p00202484567313_e_ele_datacard_combined_SR.pdf}}\\
\noindent
  \makebox[\textwidth]{
  \includegraphics[width=.35\textwidth]{Figures/c6/backgrounds/FR/closureTest/Data/onlyBjet/M-6_V-0p00202484567313_mu_muo_mass3_datacard_combined_mass3.pdf}
  \includegraphics[width=.35\textwidth]{Figures/c6/backgrounds/FR/closureTest/Data/onlyBjet/M-6_V-0p00202484567313_mu_muo_mass_datacard_combined_massl2l3.pdf}
   \includegraphics[width=.35\textwidth]{Figures/c6/backgrounds/FR/closureTest/Data/onlyBjet/M-6_V-0p00202484567313_mu_muo_disp_datacard_combined_displacement.pdf}}\\
\noindent
  \makebox[\textwidth]{
  \includegraphics[width=.35\textwidth]{Figures/c6/backgrounds/FR/closureTest/Data/onlyBjet/M-6_V-0p00202484567313_e_ele_mass3_datacard_combined_mass3.pdf}
  \includegraphics[width=.35\textwidth]{Figures/c6/backgrounds/FR/closureTest/Data/onlyBjet/M-6_V-0p00202484567313_e_ele_mass_datacard_combined_massl2l3.pdf}
   \includegraphics[width=.35\textwidth]{Figures/c6/backgrounds/FR/closureTest/Data/onlyBjet/M-6_V-0p00202484567313_e_ele_disp_datacard_combined_displacement.pdf}}\\
  \caption{Search region-like distributions with the selection of the
    \PQb jet control region, in data and background
    predictions, in final states with more than two muons (left) and
    more than two electrons (right). In the middle and at the bottom, distributions of three-lepton invariant mass (left), \mtwol
    (middle), and \Deltwod (right) in the combined Full-Run2 sideband control region,
    for the sum of all final states with muon couplings (middle) and
    electron couplings (bottom).}
  \label{fig:sb_ttbar_Ctrl}
\end{figure}




\paragraph{Closure test in data sample: data control region (sideband plus bjet)}
Taking into account the additional informations just presented and considering the low statistic of the fig~\ref{fig:sb_sr_Ctrl}, a new CR has been defined. 
\begin{table}[h!]
  \centering
{\small
  \caption{\label{tab:sideband_plus_bjet_table} Control region selection requirements
    applied to all data sets.}
    \begin{tabular}{l|l}
    \hline
    Variable     & Requirement       \\
    \hline
    \hline
    \DRtwol      & $<1$              \\
    \minDphi     & $>1$ rad          \\ 
    (\ltwo $+$ \lthree) \pt & $> 15 \GeV$              \\
    \costheta    & $>0.99$            \\
    \mtwol& $<20\GeV$              \\ 
    SV probability & $> 0.001$              \\
    $\sigma$ \Deltwod& $>20$              \\ 
    resonance vetoes & applied      \\
    \hline
    \multirow{2}{*}{\mthreel  and  N. \PQb jets} & ($<50$\GeV or
                                                   $>80$\GeV and
                                                   N.\PQb jets= 0) \\
      & OR
                                                   (N.\PQb jets $\neq$
                                                   0)\\
     \hline
     \mthreel in $\Pe\Pe\Pe$ and $\PGm\PGm\Pe$ OS & $<50$\GeV or $>101$\GeV \\
    \hline
    \hline 
  \end{tabular}
}
\end{table}
     




This control region includes the selection listed in table~\ref{tab:side_band_sel} and the region with more than zero \PQb jets. It is obvious that the combination of these two regions 
is done in a fully orthogonal region wrt to signal region. \\
  


\begin{figure}[h]
  \noindent
  \makebox[\textwidth]{
  \includegraphics[width=.55\textwidth]{Figures/c6/backgrounds/FR/closureTest/Data/SideBandPlusBjet/M-6_V-0p00202484567313_mu_muo_datacard_combined_SR.pdf}
  \includegraphics[width=.55\textwidth]{Figures/c6/backgrounds/FR/closureTest/Data/SideBandPlusBjet/M-6_V-0p00202484567313_e_ele_datacard_combined_SR.pdf}}\\
\noindent
  \makebox[\textwidth]{
  \includegraphics[width=.35\textwidth]{Figures/c6/backgrounds/FR/closureTest/Data/SideBandPlusBjet/M-6_V-0p00202484567313_mu_muo_mass3_datacard_combined_mass3.pdf}
  \includegraphics[width=.35\textwidth]{Figures/c6/backgrounds/FR/closureTest/Data/SideBandPlusBjet/M-6_V-0p00202484567313_mu_muo_mass_datacard_combined_massl2l3.pdf}
   \includegraphics[width=.35\textwidth]{Figures/c6/backgrounds/FR/closureTest/Data/SideBandPlusBjet/M-6_V-0p00202484567313_mu_muo_disp_datacard_combined_displacement.pdf}}\\
\noindent
  \makebox[\textwidth]{
  \includegraphics[width=.35\textwidth]{Figures/c6/backgrounds/FR/closureTest/Data/SideBandPlusBjet/M-6_V-0p00202484567313_e_ele_mass3_datacard_combined_mass3.pdf}
  \includegraphics[width=.35\textwidth]{Figures/c6/backgrounds/FR/closureTest/Data/SideBandPlusBjet/M-6_V-0p00202484567313_e_ele_mass_datacard_combined_massl2l3.pdf}
   \includegraphics[width=.35\textwidth]{Figures/c6/backgrounds/FR/closureTest/Data/SideBandPlusBjet/M-6_V-0p00202484567313_e_ele_disp_datacard_combined_displacement.pdf}}\\
 \caption{Distributions of three-lepton invariant mass (left), \mtwol
    (middle), and \Deltwod (right) in the combined Full-Run2 control region in table~\ref{tab:sideband_plus_bjet_table},
    for the sum of all final states with muon couplings (top) and
    electron couplings (bottom).}
  \label{fig:CR_dist}
\end{figure}




\paragraph{Closure test in data sample: validation of internal and external conversions }
\label{sec:conversion}
The photon conversion background is modeled using simulated events,
 but its overall normalization is checked in a
 $\PZ\to\ell^-\ell^+\PGg^{(\ast)}$
 control region in data, as defined in Table~\ref{tab:conv_sel}.
 \begin{table}[h!]
  \centering
  \caption{\label{tab:conv_sel} Conversion control region selection requirements
    applied to all data sets.}
    \begin{tabular}{l|l}
    \hline
    Variable     & Requirement       \\
    \hline
    \hline
    \DRtwol      & $<1$              \\
    \minDphi     & $>1$ rad          \\
    \mthreel     & between 81 and 101\GeV, \ie consistent with an on-shell \PZ boson \\
    N. \PQb jets & $=0$              \\
    (\ltwo $+$ \lthree) \pt & $> 15 \GeV$              \\
    \costheta    & $>0.99$            \\
    \mtwol& $<20\GeV$              \\ 
    SV probability & $> 0.001$              \\
    $\sigma$ \Deltwod& $>20$              \\ 
    resonance vetoes &  applied      \\
    \hline
    \hline
  \end{tabular}
\end{table}

 This region is dominated by $\PZ\to\ell^-\ell^+\PGg\to\ell^-\ell^+\Pe$
 events, with an asymmetric conversion of the real photon.
 The \mthreel and \mtwol spectra of such events are shown in
 Fig.~\ref{fig:phoConv}. A fair agreement is found between data and
 simulation, with no
 corrections required.
 

 
 \begin{figure}[h]
    \noindent
\makebox[\textwidth]{\includegraphics[width=.49\textwidth]{Figures/c6/backgrounds/FR/closureTest/Data/conversion/M-6_V-0p00202484567313_e_ele_datacard_combined_SR.pdf}}\\
  \makebox[\textwidth]{
  \includegraphics[width=.38\textwidth]{Figures/c6/backgrounds/FR/closureTest/Data/conversion/M-6_V-0p00202484567313_e_ele_mass3_datacard_combined_mass3.pdf}
  \includegraphics[width=.38\textwidth]{Figures/c6/backgrounds/FR/closureTest/Data/conversion/M-6_V-0p00202484567313_e_ele_mass_datacard_combined_massl2l3.pdf}
  \includegraphics[width=.38\textwidth]{Figures/c6/backgrounds/FR/closureTest/Data/conversion/M-6_V-0p00202484567313_e_ele_disp_datacard_combined_displacement.pdf}}
    \caption{Distribution of the three-lepton and two-lepton invariant
    masses, \mthreel and \mtwol, in the photon-conversion control
    region with events with at least 2 electrons in the final state. Full-Run2.}
  \label{fig:phoConv}
\end{figure}

\clearpage
\section{Corrections and efficiencies}\label{sec:llcorrection_efficiencies}

\subsection{Prompt lepton efficiency} \label{sec:promptleptoneff}
Prompt electron and muon identification and isolation efficiencies
are measured in data and simulation using a tag-and-probe method,
applied to samples of inclusive Z boson events.
The data-to-MC ratios of efficiencies (``scale factors'')
measured centrally by the \emph{muon physics object group} as a function of the lepton \pt and \sigeta, are
used to correct the simulated events. Additionally, the scale factors
are measured for the isolation requirement and the impact
parameter cuts. The results are presented in
Fig.~\ref{fig:isoip_muon_prompt}.
\begin{figure}[h]
  \centering
  \includegraphics[width=.48\textwidth]{Figures/c6/efficiencies/muons/2018/isoip_prompt_sf_2018.pdf}
\hfill{}
  \includegraphics[width=.48\textwidth]{Figures/c6/efficiencies/muons/2018/isoip_prompt_syst_2018.pdf}
  \caption{Data/MC efficiency scale factors (left) and associated
  systematic uncertainty (right) for impact parameter and isolation requirement efficiency
  for prompt muons. 2016 and 2017 SF can be seen in the
  Appendix~\ref{AppendixB},~\ref{fig:appisoip_muon_prompt}}
  \label{fig:isoip_muon_prompt}
\end{figure}

For the prompt electrons, the SFs are obtained following the EGM POG prescriptions using the same
tag- and probe techniques as the one for the muons. The results as a function of transverse momentum or pseudorapidity are shown in fig~\ref{fig:id_sf_electrons}.
\begin{figure}[h]
  \centering
   \includegraphics[width=.38\textwidth]{Figures/c6/efficiencies/ID_electrons/2018/leptonSF_SFvspT_HNLprompt.pdf}
\hspace{2cm}
  \includegraphics[width=.38\textwidth]{Figures/c6/efficiencies/ID_electrons/2016/leptonSF_SFvseta_HNLprompt.pdf}
  \caption{Prompt electron efficiencies and data-MC scaling factors as
    a function of transverse momentum (left) or pseudorapidity
    (right). SF for 2016/17 can be seen in the
  Appendix~\ref{AppendixB},~\ref{fig:appid_sf_electrons}}
  \label{fig:id_sf_electrons}
\end{figure}

\subsection{Trigger efficiency} \label{sec:triggereff}
Events for this analysis are selected by means of single-electron and
single-muon triggers, geometrically matched to the prompt lepton \lone
(see Sec.~\ref{sec:trigger}).
Similarly to the prompt-lepton identification and isolation,
trigger efficiencies are also measured in data and simulation with a
tag-and-probe technique. The resulting data-MC scaling factors are used to correct the
simulated event yields. 

Trigger efficiencies and data-MC scaling
factors for electrons are shown in Figure~\ref{fig:trigger_electrons},
they include both statistical and systematic uncertainties.

\begin{figure}[h]
  \centering
  \includegraphics[width=.38\textwidth]{Figures/c6/efficiencies/trigger_electrons/2018/passEle32/leptonSF_SFvspT_passEle32.png}
\hspace{2cm}
  \includegraphics[width=.38\textwidth]{Figures/c6/efficiencies/trigger_electrons/2018/passEle32/leptonSF_SFvseta_passEle32.png}
  \caption{Trigger efficiencies and data-MC scaling factors as a function of transverse momentum (left) or pseudorapidity (right)
    for HLT\_Ele32WPTightGsf in 2018 data. SF for 2016/17 can be seen in the
  Appendix~\ref{AppendixB},~\ref{fig:app_trigger_electrons}
  }
  \label{fig:trigger_electrons}
\end{figure}

\subsection{\Displ-lepton reconstruction and ID efficiency} \label{sec:displeptoneff}
\paragraph{Muons}\label{sec:eff_disp_muon}
The identification and isolation requirement selection
efficiency corrections for \displ muons are measured
using $\PZ\to\mu^-\mu^+$ events. The results are presented
in Fig.~\ref{fig:idiso_muon_nonprompt}. 

\begin{figure}[h]
  \centering
 \includegraphics[width=.48\textwidth]{Figures/c6/efficiencies/muons/2018/idiso_nonprompt_sf_2018.pdf}
\hfill{}
  \includegraphics[width=.48\textwidth]{Figures/c6/efficiencies/muons/2018/idiso_nonprompt_syst_2018.pdf}
  \caption{Data/MC efficiency scale factors (left) and associated
  systematic uncertainty (right) for ID selection and isolation
  requirement efficiency for nonprompt muons. 2016/17 SFs can be seen in the
  Appendix~\ref{AppendixB},~\ref{fig:app_idiso_muon_nonprompt}}
  \label{fig:idiso_muon_nonprompt}
\end{figure}

\paragraph{Electrons}\label{sec:eff_disp_ele}

Different reference processes are identified could which mimic---to
some extent---the main characteristics of \displ electrons 
stemming from HNL decays, such as the photon conversions in the
detector material.



\comm{
  Also this process presents limitations:
  \begin{itemize}
  \item we cannot obtain an absolute efficiency measurement;
  \item at least in the case of asymmetric conversions, there is no
    direct measurement of the displacement of the electron production
    vertex;
  \item possible discrepancies between data and simulation cannot be
    attributed unambiguously to the electron reconstruction, and could
    be due instead to a mis-modeling of the photon conversion process
    or to the detector material mapping.
  \end{itemize}}

We select asymmetric photon conversions in events
\(\PZ\to\ell^-\ell^+\PGg\to\ell^-\ell^+\Pe^\pm(\Pe^\mp),\)
where $(\Pe^\mp)$ represents a very-low-\pt electron that fails
reconstruction and/or identification. The other three leptons,
therefore, have invariant mass
\(m(\ell^-\ell^+\Pe^\pm)\)
close to that of the \PZ boson.
The events are selected as follows:
\begin{itemize}
\setlength\itemsep{-0.2em}
\item a pair of oppositely-charged prompt electrons (muons) is selected with pT > 35 (28)GeV and
10 (10)GeV requirements applied on the leading and sub-leading lepton transverse momenta,
 respectively. Electrons (muons) are reconstructed and selected with
 the identification listed in Tables.~\ref{tab:electronSelection}-~\ref{tab:muonSelection}.
  The higher-\pt threshold is driven by trigger constraints, while the
  lower-\pt threshold is chosen to improve the statistical power of
  the sample; 
\item the higher-\pt prompt lepton must be matched geometrically to a
  trigger object and fire the single-electron or single-muon trigger;
\item a \displ tight electron (see
  Table~\ref{tab:electronSelection}) with $\pt>7$\GeV, to match the
  selection on \displ electrons in the analysis;
\item the invariant mass of the three leptons must lie within 10\GeV
  of the Z boson mass.
\end{itemize}
By comparing the rates in data and simulation we can deduce the level
of disagreement in \displ electron reconstruction and
identification efficiency.
As explained above, it is not straightforward to disentangle different
contributions to this discrepancy, therefore any data-MC difference is
to be taken as a maximum possible difference in the efficiency of
\displ electrons.

Since the position of the \displ electron production vertex is
unknown, as a proxy for its displacement we use the following
variable~\cite{convz}:

\begin{align}
\vcenter{\hbox{\includegraphics[width=.42\textwidth]{Figures/c6/efficiencies/figs_Displacement.png}}}
&\qquad
\begin{aligned}
\mathrm{d = \sqrt{2Rd_{xy}+d_{xy}^2}};
\\
\mathrm{R [m] = \frac{\pt [GeV]}{0.3 \cdot B [T]}},
\end{aligned}\\
\vcenter{\hbox{\begin{minipage}{4cm}
\end{minipage}}}
& \notag
\end{align}

\noindent where $\mathrm{R}$ is the radius of curvature of the path of
a charged particle in magnetic field, $\mathrm{d_{xy}}$ is the
transverse impact parameter of the associated track, $\mathrm{B}$ is the
magnitude of the magnetic field. The obtained data/MC ratio measured in bins of displacement
(Figure~\ref{fig:convdispl}) is used as a scale factors per
electron applied to signal events.  Half of the difference of the
scale-factor with respect to unity is considered as an uncertainty.

A similar effect was also observed in a study where we use $K_s^0$  and $\lambda$ mesons, which we use to calibrate \displ muon reconstruction efficiencies. See Section~\ref{sec:displacedvertex}.
The details on how the scale factors and the systematics from each of these studies implemented in the analysis can be found in Sec.~\ref{sec:nonpromptleptoneffsysts}.

\begin{figure}[h]
  \centering
  \includegraphics[width=.45\textwidth]{Figures/c6/efficiencies/Convz/Nonprompt1DisplCoarse_lle_2016.pdf}
  \includegraphics[width=.45\textwidth]{Figures/c6/efficiencies/Convz/Nonprompt1DisplCoarse_lle_2017.pdf}\\
  \includegraphics[width=.45\textwidth]{Figures/c6/efficiencies/Convz/Nonprompt1DisplCoarse_lle_2018.pdf}
  \caption{The displacement of the \Displ electron in data
  and simulation measured in $\PZ\to\ell^-\ell^+\Pe$ events, 
  for 2016 (left), 2017 (middle) and 2018 (right)~\cite{convz}.}
  \label{fig:convdispl}
\end{figure}


\subsection{\Displ tracking and vertexing efficiency}
\label{sec:displacedvertex}

In this subsection, we investigate the discrepancy between simulation
and data concerning heavily displaced track and vertex
reconstruction. The method is based on the displaced decay of the
neutral hadrons \PKzS to two charged particles (see
Fig.~\ref{fig:diagrams}), giving the signature of two displaced tracks
originating from a common vertex. Here we will discuss the methodology
(briefly) and the most important results. More information can be found at~\cite{AN-20-111_KshortStudy}.
\begin{figure}[h]
    \centering
    \includegraphics[width=0.35\textwidth]{Figures/c6/efficiencies/diagram_ks}
    \caption{Diagrams for the decays of \PKzS studied in this method.}
    \label{fig:diagrams}
\end{figure}

The \PKzS hadrons are reconstructed in simulation and data focusing on $Z\rightarrow \mu\mu$-events, where the Drell-Yan process can be used for overall simulation-to-data comparison and normalization, and the targeted particles occur in the jets that are produced along with the main process. The study is carried out for the three data taking years of LHC's Run II.

\noindent The events in the selected data and simulated sets must pass the requirements listed below:
\begin{itemize}
\setlength\itemsep{-0.2em}
    \item exact 2 muons in the event;
    \item the leading muon must have a \pt of at least 30 \GeV, the trailing muon of at least 25 \GeV;
    \item the selected muon invariant mass is required to be within 10 GeV of the $Z$-boson mass;
    \item Events containing one or more loosely tagged $b$-jet are vetoed.
\end{itemize}

\paragraph{Vertex fitting}
\label{sec:vertexfitting}
The vertex reconstruction algorithm developed for this study closely
resembles the built-in $V^0$-vertex fitter in the CMS software (where
$V^0$ is used to refer to \PKzS vertex), with however some small
alterations to account for the fact that we are working on MINIAOD
level instead of RECO. Furthermore, we re-optimized some of the cut
values in order to retain more $V^0$ candidates at small radial
displacements from the primary vertex, at the cost of a slightly
larger overall background. This design choice was inspired by the need
for a large number of candidates close to the primary vertex, as these
will be used for normalization (see further on). 

As a check of the vertex fitting algorithm, we plot the invariant mass distributions of the reconstructed \PKzS in Fig.~\ref{fig:invmass_mumuskim}.
\begin{figure}[h]
    \centering
    \includegraphics[clip,trim=0 0 16cm 0,width=0.35\textwidth]{Figures/c6/efficiencies/invmass_mumuskim}
    \caption{\PKzS candidate spectrum with linear background fit and
      double gaussian signal fit. Note that in these plots the result for the full 2017E data taking period is shown. }
    \label{fig:invmass_mumuskim}
\end{figure}

\paragraph{Methodology}\label{sec:methodology}
The yields of reconstructed $V^0$-particles is compared between data and simulation, binned in radial distance from the primary vertex. As a first remark, we plot the yield of \PKzS particles, using a relatively fine binning in Fig.~\ref{fig:2017_detector}. Notice how the pixel detector layout has a clear structural influence on the yield, both in simulation and data.
\begin{figure}[h]
    \centering
    \includegraphics[width=0.40\textwidth]{Figures/c6/efficiencies/2017E_detector}
    \caption{Results for \PKzS particles using a fine binning and linear $y$-axis scale with the radial distances of the pixel detector layers superimposed. Note that in this plot, the total number of \PKzS particles in simulation was normalized to that in data as we merely wanted to perform a shape comparison.}
    \label{fig:2017_detector}
\end{figure}

In the following, we normalize the yield in simulation to the yield in
data at small radial distances.This approach allows us to correct for
overall mis-modeling effects of $V^0$-particles
and to isolate the effect under consideration, \ie a systematic
difference in vertex reconstruction efficiency due to its being
displaced. The exact range for normalization was chosen to be 0 - 0.5
cm for $K^0$. Furthermore, a fit to the sidebands in the invariant mass spectrum per bin of radial distance was used to subtract the background contribution.

\subparagraph{Main results and conclusions}\label{sec:mainresults}
The obtained data-to-simulation ratio plots for the three data taking
years of Run II are shown in Fig.~\ref{fig:1dplots_2016_2017_2018}. \\
Note that the discrepancy is very large in the last bin for 2016 data. This can be explained by the reduced tracking performance in eras 2016 B to 2016 F due to the HIP-effect, which in the end turned out to be caused by a saturation effect in the electronics \cite{hipeffect}. The electronics were reconfigured between run 2016 F and 2016 G. It can be checked in the dedicated note ~\cite{AN-20-111_KshortStudy} that the simulation to data agreement is in fact much better for eras 2016 G and H. \\ \\
In Fig.~\ref{fig:2dplots}, we extend these results by performing a two-dimensional binning: the $x$-axis again represents the radial distance from the primary vertex, while the $y$-axis now holds the reconstructed transverse momentum of the particle. 

\begin{figure}[h]
\noindent
\makebox[\textwidth]{
		\includegraphics[width=0.35\textwidth]{Figures/c6/efficiencies/1dplots/2016tot/ksnorm3small}
		\includegraphics[width=0.35\textwidth]{Figures/c6/efficiencies/1dplots/2017tot/ksnorm3small}
		\includegraphics[width=0.35\textwidth]{Figures/c6/efficiencies/1dplots/2018tot/ksnorm3small}}
	\caption{Data to simulation ratio for data taking year 2016
          (left), 2017 (middle) and 2018 (right). The error bars on data points and error bands on simulated results represent statistical errors only.}
	\label{fig:1dplots_2016_2017_2018}
\end{figure}

\begin{figure}[h]
	\centering	
	\includegraphics[width=0.45\textwidth]{Figures/c6/efficiencies/2dplots/2016tot/smallrange}
	\includegraphics[width=0.45\textwidth]{Figures/c6/efficiencies/2dplots/2017tot/smallrange}
	\\
	\includegraphics[width=0.45\textwidth]{Figures/c6/efficiencies/2dplots/2018tot/smallrange}
	\caption{Uncertainty factors per bin in radial distance and transverse momentum. The red-coloured band overlaying the figure shows the normalization range, i.e. 0 - 0.5 cm. Note that an error of $\pm 0.00$ indicates that the (statistical) error is smaller than 0.005.}
	\label{fig:2dplots}
\end{figure}

\clearpage
\section{Systematic uncertainties}\label{sec:llsystematic}
Several sources of systematic uncertainty affect the expected signal
and background yields in different search regions. 
The effect of the main uncertainties on the signal yields in each
search region and lepton channels are summarized in
Fig.~\ref{fig:systematics_all}. 
Most of the experimental uncertainties are relatively small (less than
a few \%), where in some search regions the uncertainties are
dominated by the uncertainties due to the limited signal MC samples statistics. 
{\color{red}
  $\leftarrow$ this
  will change with the new re-weithing procedure which quadruple the
  stat and reduce the stat error on the signal samples}

\begin{figure}[h]
\noindent
\makebox[\textwidth]{  \includegraphics[width=.58\textwidth]{{Figures/c6/systematics/2018/e_ele/M-6_V-0.00202484567313_normalized}.pdf}
  \includegraphics[width=.58\textwidth]{{Figures/c6/systematics/2018/mu_muo/M-6_V-0.00202484567313_normalized}.pdf}}
  \caption{Relative up and down variation from the nominal yields for
    the signal sample with $\mhnl=6\GeV$ and
    $\mixpar=4\times10^{-6}$. 
    The variation is shown for each of the search regions for
    electrons (left) and muons (right), for the 2018 data taking year.}
  \label{fig:systematics_all}
\end{figure}

\subsection{Uncertainties on the signal yields}
All uncertainties summarized in this section are evaluated per search
region and per lepton channel and considered as shape uncertainties in
the final fit, with the exception of those that only affect the
overall normalization, such as the signal cross section and
luminosity.

\paragraph{Integrated luminosity}
The uncertainties on the measurements of the LHC integrated
luminosities are 2.5\% (2016), 2.3\% (2017), and 2.5\% (2018).
They affect the yields of the signal and of the MC-based background
estimations in a fully correlated way (within the same data-taking
period).

\paragraph{Pileup}
The uncertainty in the modeling of the pileup is evaluated by
decreasing and increasing the minimum bias cross section by 5\%~\cite{pileuppaper},
resulting in an uncertainty of 1--4\% for most search regions,
with some larger deviations for statistically limited search
regions.

\paragraph{Trigger efficiency}
Trigger scale factors are used to correct the simulated event yields
(see Sec.~\ref{sec:triggereff}), and the uncertainties from the
tag-and-probe fits are used to assess systematic uncertainties,
with the same procedure as described in
Sec.~\ref{sec:promptleptoneffsysts}.
These are found to be less than 1\% for electron triggers and 
less than 1\% for muon triggers.

\paragraph{Prompt lepton reconstruction and identification efficiency}
\label{sec:promptleptoneffsysts}
To find the uncertainties associated with the prompt-lepton
identification and isolation efficiency scale factors (see
Sec.~\ref{sec:promptleptoneff}),
the total yields in each search region are recomputed with the scale
factors varied up and down by the tag-and-probe fit uncertainties,
treating all bins as correlated.
Uncertainties of 2--5\% on the signal yields are found
for events with a prompt electron, while the systematic on prompt
muons stays consistently below 1\% for all search regions.

\paragraph{\Displ lepton identification efficiency and displaced vertex efficiency }
\label{sec:nonpromptleptoneffsysts}
Data events with Z bosons decaying into lepton pairs provide an excellent testing ground for the prompt lepton 
reconstruction and identification efficiencies in simulation.
The main source of the systematic uncertainty in our analysis is the displaced track and vertex reconstruction modeling. 
We make use of two types of processes to estimate this uncertainty. On one side the converted photons in Z events (described in section~\ref{sec:eff_disp_ele}) 
are used to asses the modeling of \displ electron reconstruction,  long-lived \PKzS 
mesons are used to scrutinize the displaced track and vertex reconstruction in simulation. 

Both of these studies studies show consistent results in terms of data/MC agreement across the data taking periods. 
In both cases we observe larger data/MC scale factors in 2016 data, which is caused by a known tracking inefficiency 
in data which is not simulated in MC. In this dataset the discrepancy
between data and simulation can reach up to 30\% (see
Figs.~\ref{fig:convdispl}--\ref{fig:2dplots}).
Based on these studies we apply data/MC scale factors per \displ lepton to be applied in simulation and half of this scale factor is considered as a systematic uncertainty.  
In the channels with \displ $\Pe\Pe$ ($\PGm\PGm$) the scale factors
are derived from conversion (\PKzS) and in events with \displ $\Pe\PGm$ the 
scale factor for electron comes from the conversion while for muon
comes from \PKzS study. The \Displ-muon reconstruction and ID
efficiency (~\ref{sec:eff_disp_muon}) is added per muon in 
$\PGm\PGm$ and $\Pe\PGm$ events on top of the corrections for the displaced
track and vertex reconstruction.
\begin{table}[h]
  \centering
{\footnotesize
  \caption{\label{tab:summary_SVl} {\color{red}
  $\rightarrow$ these
  values have to be checked again} Summary of the application of the
    efficiencies and systematic uncertainties related to \displ leptons and displaced vertices. The impact for each year is roughly stated.}
  \begin{tabular}{l|c|c|c|c|c}
    \hline
    \ltwo, \lthree    & efficiency & uncert. & 2016  & 2017 &2018    \\
    \hline
    \hline
   $\Pe\Pe$           & per $\Pe$ ID/track eff-cy ($\PZ\gamma$ study(~\ref{sec:eff_disp_ele}))          & $\pm$1/2 SF    & $< 20 \%$  & $< 5 \%$  & $< 10 \%$\\
   $\PGm\PGm$   &per $\mu$ (ID eff-cy) $\times$ SV eff-cy (\PKzS\ study(~\ref{sec:vertexfitting}))                      & $\pm$1/2 SF    & $< 30 \%$  & $< 10 \%$  & $< 5 \%$\\
   $\PGm\Pe$       & per $\Pe$ ($\PZ\gamma$ study) $\times$ per $\mu$ (ID
                     eff-cy) $\times$ $\sqrt{SV_{eff-cy}}$                 & $\pm$1/2 SF    & $< 20 \%$  & $< 7 \%$  & $< 7 \%$\\
    \hline
    \hline
  \end{tabular}
}
\end{table}

The uncertainties in the final yields in the different search regions
and data taking periods are calculated by varying the MC event weights
by half of these scale factors for each \displ lepton (\ie two per
event). The resulting uncertainties in each signal region/year and
lepton channel can be seen in Fig.~\ref{fig:systematics_all}. 

\paragraph{\Displ muon momentum scale and resolution}
\label{sec:nonpromptleptonscaleresol}
The study of \PKzS\ decays (Section~\ref{sec:displacedvertex}) also
provides useful information about the momentum scale and resolution of
the \displ-muon tracks. For \pt below 200\GeV, the muon momentum 
estimate is fully dominated by the tracker-only fit. It is easy to
prove that the muon momentum resolution $\sigma(\pt)/\pt$ is
approximately equal to the dimuon mass resolution
$\sigma(M_{2\ell})/M_{2\ell}$. The Gaussian fits to the \PKzS\ mass
profiles, such as Fig.~\ref{fig:invmass_mumuskim}, can thus be used to
estimate the momentum scale and resolution through the
fit parameters $\mu$ and $\sigma/\mu$, respectively. The width of the
central mass peak is estimated by the smaller of $\sigma_1$ and
$\sigma_2$, whereas the larger $\sigma$ represents the mass tails.
Figure~\ref{fig:displMuScaleResol} shows the values of the \PKzS\ mass
peak and width as a function of the transverse displacement \Deltwod,
in data and simulation, for the three data taking years.
\begin{figure}[h!]
  \centering
  \includegraphics[width=.4\textwidth]{Figures/c6/systematics/scale_displMu.png}
  \includegraphics[width=.4\textwidth]{Figures/c6/systematics/resolution_displMu.png}
  \caption{Values of the \PKzS\ mass peak (left) and width (right) as
    a function of the transverse displacement \Deltwod, in data and
    simulation, for the three data taking years. The mass peak and
    width are estimated, respectively, by the $\mu$ and smaller
    $\sigma$ parameters of the double-Gaussian fit to the
    \PKzS\ mass profile (\eg, see Fig.~\ref{fig:invmass_mumuskim}).}
  \label{fig:displMuScaleResol}
\end{figure}

As can be seen, the differences between data and simulation are
relatively small. Table~\ref{tab:displMuResolA} reports the relative
data-MC difference in the scale parameter ($\mu$). The difference is
at the per-mil level, thus negligible. 
\begin{table}[h]
\caption{}
{\scriptsize
\subfloat[\label{tab:displMuResolA} Relative difference in mass (or
    momentum) scale between data and simulation, estimated as
    $(\mu_\mathrm{data}-\mu_\mathrm{sim})/\mu_\mathrm{sim}$.]{\begin{tabular}{l|c|c|c}
    \hline
    \multirow{2}{*}{\Deltwod [cm]} &
    \multicolumn{3}{c}{Data-MC scale difference [\%]} \\
    & \quad\quad2016\quad\quad & \quad\quad2017\quad\quad & \quad\quad2018\quad\quad \\
    \hline
    \hline
    $<$ 0.5  & $-0.092$ & $-0.074$ & $-0.088$ \\
    \hline                                       
    0.5--1.5 & $-0.088$ & $-0.110$ & $-0.096$ \\
    \hline                                       
    1.5--4.0 & $-0.086$ & $-0.098$ & $-0.104$ \\
    \hline                                       
    $>$ 4.0  & $-0.074$ & $-0.090$ & $-0.098$ \\
    \hline
  \end{tabular}}
\qquad
\subfloat[\label{tab:displMuResolB} Difference in quadrature in mass
    (or momentum) resolution between data and simulation, where the
    resolution is estimated as $\sigma/\mu$.]{\begin{tabular}{l|c|c|c}
    \hline
    \multirow{2}{*}{\Deltwod [cm]} &
    \multicolumn{3}{c}{Data-MC resolution difference [\%]} \\
    & \quad\quad2016\quad\quad & \quad\quad2017\quad\quad & \quad\quad2018\quad\quad \\
    \hline
    \hline
    $<$ 0.5  & 0.20 & 0.37 & 0.33 \\
    \hline                           
    0.5--1.5 & 0.19 & 0.26 & 0.15 \\
    \hline                           
    1.5--4.0 & 0.25 & 0.26 & 0.22 \\
    \hline                           
    $>$ 4.0  & 0.29 & 0.19 & 0.22 \\
    \hline
  \end{tabular}}
}
\end{table}

Table~\ref{tab:displMuResolB} reports the difference (in quadrature) in
resolution between data and simulation, where the resolution is
estimated as $\sigma/\mu$. The mass (or momentum) resolution is below
1\% in both data and simulation, and the difference in quadrature is
0.15--0.40\%. No corrections and no uncertainties are applied.

\paragraph{Statistical uncertainty due to limited MC sample}
For the processes estimated from simulation, the available number of
events in the MC samples limits the precision of the modeling.
The statistical uncertainty on the event yield in each search bin,
corresponding to the sum in quadrature of the MC event weights, is
therefore taken as a systematic uncertainty on the shape of the
distributions used in the analysis.

\paragraph{Uncertainty on signal MC cross section}{\color{red}
  $\rightarrow$ does this part have to be in Chapter 4??? } 
The heavy Neutrino model used for generation of Monte Carlo HNL events does not allow for NLO QCD calculations. The simulation of HNL events happens therefore at LO, resulting in large theoretical uncertainties on the cross section (up to $15\%$) that have to cover the effect of the missing higher order QCD corrections and PDF uncertainties.

Instead of relying on these LO uncertainties, a general correction factor for the cross section from LO to NNLO can be derived based on the SM production of \PW\ $\rightarrow$ $\ell$ $\bar{\nu}$ . In HNL production, the only difference from this SM process is the exchange of the SM neutrino by a HNL. The effect of the mass and coupling of the HNL can be factorized in the calculation of the HNL cross section and is not affected by the PDF and scale variations. The dominant effect of these uncertainties appears at the production of the \PW boson, therefore it can be studied in the SM process \PW\ $\rightarrow$ $\ell$ $\bar{\nu}$ , for which recommended values for the cross section at NNLO exist with their corresponding theory uncertainties. Our approach is to get a LO cross section for \PW\ $\rightarrow$ $\ell$ $\bar{\nu}$ calculated with Madgraph using the same exact conditions as the HNL MC production. A correction factor from LO to NNLO can be derived based on this and can then be applied on HNL MC. The PDF and scale uncertainties at NNLO are applied as flat systematic uncertainties to cover the remaining uncertainty on the MC cross section.

It has been checked and verified that the generator conditions are similar between our MC production and the centrally produced W+jets samples, from which the recommended NNLO cross section was taken. The small differences that are present have no significant effect on the cross section calculation. Additionally, lepton universality will allow us to apply the scale factor and uncertainty across all HNL samples, regardless of which lepton flavor(s) they couple with.

The resulting LO cross section for \PW\ $\rightarrow$ $\ell$
$\bar{\nu}$ ($\ell$ = \Pe\ , \PGm\ , \PGt\ ) is 56500 pb. The recommended NNLO value is $61526.7^{+497.1}_{-264.6}\pm 2312.7$ pb where the quoted uncertainties are respectively scale and PDF uncertainties. Assuming uncorrelated uncertainties and taking the maximum of the two asymmetric errors, the combined uncertainty is $61526.7 \pm 2365.5$ pb, an effect of $3.86\%$. This gives a final scale factor of $1.089 \pm 0.042$.

\paragraph{Other minor systematic uncertainties}
\textbf{JEC and JER}
The uncertainty in the calibration of the jet energy scale, as well
as the jet energy resolution in simulation, affect directly the veto
on \PQb jets with $\pt>25$\GeV.
The uncertainties on jet energy scale and resolution are estimated by
scaling independently the energy of jets up and down by one standard
deviation; this results in uncertainties on the final signal yields of about
0--5\% for the energy scale and 0--3\% for the energy resolution,
mostly around the lower side of these ranges, but occasionally larger
deviations happen in lower statistics search regions.
This includes the effect of the migration of events from one search
region to another.

\textbf{b tagging}
The efficiency of the \PQb jet veto is corrected for the difference
between data and simulation, using scaling factors provided by the BTV
POG~\cite{BTagCorr}.
The resulting uncertainty is found to
be below 1\% for all search regions.

\subsection{Uncertainties on the background data-driven predictions}
The background originating from SM nonprompt leptons is estimated
using data-driven techniques as described in
Sec~\ref{sec_llfakelepton}. The general features of the method were
validated in simulation (Figure~\ref{fig:mcClosure}) as a function of
the search variables. A much detailed validation in each lepton
channel separately and in bins of the search regions is also performed
in four data control samples. Although the lack of large number of events in the high displacement regions do not allow a precise validation, we find that in regions where reasonable statistics are available the method checks out good. 
Based on the agreement between prediction and observation in these data control samples and simulation, we have decided to treat the uncertainties in this way:
\begin{itemize}
\setlength\itemsep{-0.2em}
\item 30$\%$ for all channels flat;
\item additional 50$\%$ uncorrelated nuisance for the highest displacement bin and the $\mtwol > 4$\GeV bins;
\item 3 additional uncorrelated nuisances ($\mu\mu$, $e\mu$ and $ee$) to account for the differences in sources and statistic between the 3 channels in the double fakerate measurement region.
\end{itemize}
The 30$\%$ covers all the discrepancies we observe in our control region closure and 50$\%$ is motivated with the lack of statistics in the control regions for the closure test. 
The closure tests in simulation and data in each data taking period show no particular behavior, as such we consider the 30$\%$ and 50$\%$ systematic uncertainties correlated over all data-taking period and lepton channels, while the additional 3 flavor systematics are taken uncorrelated among the channels and the years. ($\mu\mu$, $e\mu$ and $ee$. For 2016 17, 20, 40 $\%$, for 2017 14, 14, 33 $\%$ and for 2018 10, 11, 25 $\%$).

Concerning the $\Pe\Pe\Pe$ and $\PGm\PGm\Pe$ channels where we observe a large discrepancy in the largest displacement region in the sideband, we have done convincing studies (see section~\ref{sec:conversion}) that the origin of it is not the nonprompt lepton background but it comes from photon conversion. One evidence for this being due to photon conversions is the fact that the excess in data appears only in channels where one lepton from $\PGg\to\Pe\Pe$ gets combined with a prompt lepton that comes from a \PZ decay (namely only in $\Pe\Pe\Pe$ and $\PGmpm\PGmmp\Pe$) to form a displaced vertex. The other channels where we do not expect such a contribution has no such an excess. Also these excess events has three leptons forming a Z mass (see Figure~\ref{fig:phoConv}) is also a strong evidence that these are coming from conversions. It is however almost guaranteed that  the contribution from conversions should be very limited in the signal region due to the \mlll upper bound of 80 GeV. 

The statistical uncertainty on the predicted nonprompt lepton background
in the majority of search regions is largely driven by the limited
number of events in data in the sideband control region. This uncertainty
is propagated to the signal region and it is treated according to Gamma-distributions, using shape parameter $\alpha =$ (1 +  the number of events in the sideband region) and the scale parameter which is the average FR for each bin.
If no events are observed in the sideband region, we consider, as error, the standard deviation of the Gamma Distribution which has the shape parameter $\alpha = 1$ and the scale parameter $\theta =$ maximum measured fake rate in the corresponding channel.  










\clearpage
\section{Results}
\subsection{Statistical analysis}
\subsection{Limits on $V_{Nl}$: Dirac HNL}
\subsection{Limits on $V_{Nl}$: Majorana HNL}
\subsection{Limits on $V_{Nl}$: Tau coupling}

\subsection{Summary}

