\chapter*{Preface}\label{chapter:introduction}
\enquote{\itshape {\small We do not know what the rules of the game are; all we are allowed to
do is to watch the playing. Of course, if we watch long enough, we may
eventually catch on to a few of the rules. The rules of the game are
what we mean by fundamental physics.}} Richard P. Feynman\\
This quote brilliantly exemplifies the need for fundamental
research. We all want to satisfy our 
curiosity and science is a process of investigating, of keeping
questioning and trying to uncover the unresolved.\\
Thus, it becomes explicit the drive and the need for the constructionStandard Model
of huge laboratories where to study the fundamental constituents of
matter.
CERN's accelerator complex and experiments are the
best place to ``watch the playing'' and ``if we watch long enough, we may
eventually catch on to a few of the rules''.

Since the 1960s, physicists had proposed quite a collection of what
they assumed to be fundamental particles and the fundamental forces
 that govern the interaction between particles themselves. This was
 the start to the development of the Standard Model of particle
 physics. At CERN we make use of the world'sStandard Model
most potent particle accelerators and experiments to question the
predictions and to test limits of the Standard Model. So far,
the Standard Model (SM) has favorably described and predicted
approximately all experimental outcomes till the latest
discoveries such as top quark observation (1995)~\cite{Abachi_1995}
and Higgs boson discovery (2012)~\cite{20121,201230}.

However cosmological and astronomical observations, together with theoretical
considerations, allude to physics beyond the Standard Model (BSM).
The observation of neutrino flavor oscillations was one of the first 
definite experimental indications of the
presence of new physics not described by the SM theory.
 Therefore, it felt crucial to investigate at the LHC
experiments the signatures of different BSM models
to try spotting the mysterious new physics. Above all, comprehend the
mechanism that discloses the neutrino mass would be a
vital flare into ``the rules of the game''.

The two results presented in the dissertation cooperate in the arduous attempt of confronting exotic BSM
models with the experimental data. The aim is to find new
physics able to describe the unexplained
physics observations not covered by the SM.\\

This thesis tries to give an overview of all the steps to follow
to experimentally investigate physics
beyond the Standard Model. The reader will encounter in order the
descriptions of: theoretical models and predictions, experiment
layouts, detector performances, data analysis and unfortunately, at the
end not a discovery but exclusion limits on specific exotic models.

After a brief description of the principles of the Standard Model in
Chapter~\ref{Chapter1}, a summary of the Large Hadron Collider (LHC)
and of Compact Muon Solenoid (CMS) components and objectives is
presented in Chapters~\ref{Chapter2} and~\ref{Chapter2_5}.\\
In Chapter~\ref{Chapter3} the main topic of the doctoral project is
introduced. The focus is on right-handed (RH) neutrino or heavy
neutral lepton (\hnl, HNL)
model. In Chapters~\ref{Chapter1} and~\ref{Chapter3} 
the relevance and the interest for the
ongoing HNL search program is illustrated, describing first the theory setting 
and then mentioning the various of experiments and results
focusing on HNL.

Two separated searches on HNLs have been performed using the data
collected by the CMS detector. The two analysis are the core of this
dissertation.\\
The HNL searches are fully described in Chapters~\ref{Chapter5}
and~\ref{Chapter6} and they target complementary phase spaces in HNL mass, $m_\hnl$
and mixing parameter, \mixpar, between heavy neutrinos and standard
model neutrinos. The study focuses first on the
moderate and high mass search and then drifts to very low mass search with
the introduction of long-lived HNL scenario.

In Chapter~\ref{Chapter5} the search for heavy neutral leptons with three
prompt charged leptons in any flavor combination of electrons and
muons is defined. It results in a comprehensive analysis that probes the production of the
\hnl in a extended mass range never investigated before at the LHC:
from 1\GeV, and up to 1.2\TeV. 
Organized into two parts, the search is optimized for
probing HNLs of masses respectively above and below that of the \PW
boson. Several analysis techniques are introduced.

In Chapter~\ref{Chapter6} the search for long-lived heavy neutral
leptons is presented. The analysis signature
consisting of one prompt charged lepton and two displaced
charged leptons originating from the secondary vertex of the \hnl
decay. Due to its unconventional signature the whole analysis process has been
quite demanding; several challenges are presented (final part of
Chapter~\ref{Chapter4}, too) and the steps taken during the three
years course are inspected and spelt out one by one.

In Chapter~\ref{Chapter7} a short summary is given together with an
inclusive outlook overview. The future prospects section tries to give a feeling of the vast landscape
where the two CMS HNL searches sit. The idea is to give sense of the
context and of the perspective from where to look these results
from.

\subsection*{Author's contribution}

The author contributed to the development and finalization of the
searches for heavy neutral leptons in multilepton lepton final
states at CMS.\\
The precise contributions of the author are specified at
the end of the two dedicated chapters
(~\ref{Chapter5},~\ref{Chapter6}) where the studies are described
and are summarized shortly below.\\

For almost the first two years of the PhD, the author
has participated in the the search for heavy neutral leptons with three
prompt charged leptons in any flavor combination of electrons and
muons. \\
The analysis
results were published in Physical Review Letters in early 2018:
\begin{itemize}
\setlength\itemsep{-0.1em}
\item Albert M Sirunyan et al. Search for heavy neutral leptons in events with three charged leptons
in proton-proton collisions at $\sqrt{s}$ = 13 TeV. Phys. Rev. Lett., 120(22):221801, 2018.
\end{itemize}

The results discussed in Chapter~\ref{Chapter5} were presented by the author at the following
conferences:
\begin{itemize}
\setlength\itemsep{-0.1em}
\item (YSF talk) Search for heavy neutral leptons (sterile
  neutrinos) with the CMS detector, plenary at La Thuile 2018, 25
  Feb-3 Mar 2018 (Italy);
\item Search for Heavy Neutral Leptons with CMS detector, poster at
  EPS-HEP2019, 10-17 Jul 2019 (Belgium);
\item Search for Heavy Neutral Leptons with CMS detector, poster at
  LP2019, 5-10 Aug 2019, University of Toronto (Canada);
\item HNL experimental overview, plenary at EOS winter Solstice
  meeting, 19 Dec 2019, Brussels (Belgium);
\item  Search for heavy neutral leptons at CMS, parallel at
  ICHEP2020, 28 Jul-6 Aug 2020 (Virtual World).
\end{itemize}
The poster presented at EPS-HEP2019 was awarded with ``poster prize of
the EPS High Energy and Particle Physics Division and the EPS HEP 2019
Conference''.\\
 
In the last three PhD years the author was the main developer for the
long-lived heavy neutral lepton search. The study targets signatures
consisting of one prompt charged lepton and two displaced
charged leptons originating from the secondary vertex of the \hnl
decay.\\
The analysis results have been collected in the Physics Analysis
Summaries, PAS which has been made public. The final paper will be
submitted by the end of the year to Journal
of High Energy Physics (JHEP):
\begin{itemize}
\setlength\itemsep{-0.1em}
\item Search for long-lived heavy neutral leptons with displaced vertices in pp collisions at $\sqrt{s}$ =
13TeV with the CMS detector. Technical report, CERN, Geneva, 2021.
\end{itemize}

The results discussed in Chapter~\ref{Chapter6} were presented by the author at the following
conferences and workshops:

\begin{itemize}
\setlength\itemsep{-0.1em}
\item HNL experimental overview, talks Searching for long-lived
  particles at the LHC: firth workshop of the LHC LLP Community, 27-29
  May 2019, CERN (Switzerland);
\item HNL experimental overview', plenary at EOS winter Solstice
  meeting, 19 Dec 2019, Brussels (Belgium);
\end{itemize}

At the European Physical Society Conference on High Energy Physics
(EPS-HEP) in July 2021, the results were for the first time publicly
presented by CMS Physics Coordinator in ``Highlights from the CMS
Experiment'' talk.

