\chapter*{Preface}\label{chapter:introduction}



%----------------------------------------------------------------------------------------

% Define some commands to keep the formatting separated from the content 
\newcommand{\keyword}[1]{\textbf{#1}}
\newcommand{\tabhead}[1]{\textbf{#1}}
\newcommand{\code}[1]{\texttt{#1}}
\newcommand{\file}[1]{\texttt{\bfseries#1}}
\newcommand{\option}[1]{\texttt{\itshape#1}}

%----------------------------------------------------------------------------------------

The observation of neutrino flavor oscillations was one of the first 
definite experimental indications of the
presence of new physics not described by the SM theory. Thus, 
comprehand the mechanism that discloses the neutrino mass would be a
vital flare into BSM physics. Therefore, it felt crucial to investigate at the LHC
experiments the signatures of different neutrino mass models
to try spotting the mysterious new physics.\\
The two results presented in the dissertation cooperate in the arduous attempt of confronting exotic BSM
models with the experimental data. The aspiration is to find new
physics able to describe the unexplained
physics observations not covered by the SM. 

We decided to focus on right-handed (RH) neutrino or heavy neutral lepton (HNL)
model. The introduction of massive RH
neutrinos provides an answer to the SM problem of the
neutrino masses via a type-I seesaw mechanism.  
In Chapters~\ref{Chapter1} and~\ref{Chapter3} we illustrated
the relevance and the interest for the
ongoing HNL search program, describing first the theory setting 
and then mentioning the various of experiments and results
focusing on HNL.\\



Searches for heavy neutral lepton, \hnl, of Majorana and Dirac natures
dacaying into a charge lepton and a \PW boson in pp collisions with
the CMS experiment at LHC are presented.
The searches target complementary phase spaces in HNL mass, $m_\hnl$
and mixing parameter, \mixpar, between heavy neutrinos and standard
model neutrinos. The study focuses first on the
moderate and high mass search to migrate to very low mass search with
the introduction of long-lived HNL scenario.\\

The search for heavy neutral leptons focuses on signatures with three
prompt charged leptons in any flavor combination of electrons and
muons. It provides a clean signal for probing the production of the
\hnl in a extended mass range never investigated before at the LHC:
from 1\GeV, and up to 1.2\TeV. The sample of pp collisions were
collected by the CMS experiment during 2016 data-taking, amounting to
an integrated luminosity of 35.9\fbinv. 
Organized into two parts, the search is optimized for
probing HNLs of masses respectively above and below that of the \PW boson. 
The data are consistent with the expected standard model
background. On the values of \mixparm and \mixpare at 95\%
 confidence level upper limits are set. 
These are the first limits obtained at a hadron collider for
 $m_\hnl <$ 40\GeV and the first direct results for $m_\hnl >$
 500\GeV.\\

The search for long-lived heavy neutral leptons targets signatures
consisting of one prompt charged lepton and two displaced
charged leptons originating from the secondary vertex of the \hnl decay.
The data were collected from the
CMS experiment in years 2016 -- 18 corresponding to an integrated
luminosity of 137\fbinv. Two interpretations are proposed considering on one hand uniquely the
Dirac HNL nature, on the other hand the Majorana HNL nature. 
No statistically significant deviation from the expected
SM background was observed. At 95\% confidence level limits were set on the mixing
parameters \mixpare and \mixparm. The excluded values are in the
ranges between $3\times 10^{-7}$ and $1\times 10^{-3}$ for masses included
between 1\GeV $< m_\hnl <$ 15\GeV. 


