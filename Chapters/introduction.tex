\chapter*{Preface}\label{chapter:introduction}
``We have two goals in front of us. One is to explain the story of our universe and why
we think it’s true, the big picture as we currently understand it. It’s a fantastic conception''






The observation of neutrino flavor oscillations was one of the first 
definite experimental indications of the
presence of new physics not described by the SM theory. Thus, 
comprehand the mechanism that discloses the neutrino mass would be a
vital flare into BSM physics. Therefore, it felt crucial to investigate at the LHC
experiments the signatures of different neutrino mass models
to try spotting the mysterious new physics.\\
The two results presented in the dissertation cooperate in the arduous attempt of confronting exotic BSM
models with the experimental data. The aspiration is to find new
physics able to describe the unexplained
physics observations not covered by the SM. 

We decided to focus on right-handed (RH) neutrino or heavy neutral lepton (HNL)
model. The introduction of massive RH
neutrinos provides an answer to the SM problem of the
neutrino masses via a type-I seesaw mechanism.  
In Chapters~\ref{Chapter1} and~\ref{Chapter3} we illustrated
the relevance and the interest for the
ongoing HNL search program, describing first the theory setting 
and then mentioning the various of experiments and results
focusing on HNL.\\



Searches for heavy neutral lepton, \hnl, of Majorana and Dirac natures
dacaying into a charge lepton and a \PW boson in pp collisions with
the CMS experiment at LHC are presented.
The searches target complementary phase spaces in HNL mass, $m_\hnl$
and mixing parameter, \mixpar, between heavy neutrinos and standard
model neutrinos. The study focuses first on the
moderate and high mass search to migrate to very low mass search with
the introduction of long-lived HNL scenario.\\

The search for heavy neutral leptons focuses on signatures with three
prompt charged leptons in any flavor combination of electrons and
muons. It provides a clean signal for probing the production of the
\hnl in a extended mass range never investigated before at the LHC:
from 1\GeV, and up to 1.2\TeV. The sample of pp collisions were
collected by the CMS experiment during 2016 data-taking, amounting to
an integrated luminosity of 35.9\fbinv. 
Organized into two parts, the search is optimized for
probing HNLs of masses respectively above and below that of the \PW boson. 
The data are consistent with the expected standard model
background. On the values of \mixparm and \mixpare at 95\%
 confidence level upper limits are set. 
These are the first limits obtained at a hadron collider for
 $m_\hnl <$ 40\GeV and the first direct results for $m_\hnl >$
 500\GeV.\\

The search for long-lived heavy neutral leptons targets signatures
consisting of one prompt charged lepton and two displaced
charged leptons originating from the secondary vertex of the \hnl decay.
The data were collected from the
CMS experiment in years 2016 -- 18 corresponding to an integrated
luminosity of 137\fbinv. Two interpretations are proposed considering on one hand uniquely the
Dirac HNL nature, on the other hand the Majorana HNL nature. 
No statistically significant deviation from the expected
SM background was observed. At 95\% confidence level limits were set on the mixing
parameters \mixpare and \mixparm. The excluded values are in the
ranges between $3\times 10^{-7}$ and $1\times 10^{-3}$ for masses included
between 1\GeV $< m_\hnl <$ 15\GeV. 


Abundantly clear at this point of the thesis, HNLs are one of the most
exciting and best-motivated potential solutions for some of the
outstanding problems of the SM. However, if they happen to exist, their
Majorana/Dirac natures, their masses, and their coupling with the SM
neutrinos are far from obvious and clear. Thus we need to adopt a
comprehensive and vast approach in seeking for HNL probing heavy
neutral leptons with MeV- and TeV-scale masses.\\


\subsection*{Author's contribution}

The author contributed to the development and finalization of the
searches for heavy neutral leptons in multilepton lepton final
states at CMS.\\
The precise contributions of the author are specified at
the end of the two dedicated chapters
(~\ref{Chapter5},~\ref{Chapter6}) where the studies are described
and are summarised shortly below.\\

For almost the first two years of the PhD, the author
has partecipated in the the search for heavy neutral leptons with three
prompt charged leptons in any flavor combination of electrons and
muons. \\
The analysis
results were published in Physical Review Letters in early 2018:
\begin{itemize}
\setlength\itemsep{-0.1em}
\item Albert M Sirunyan et al. Search for heavy neutral leptons in events with three charged leptons
in proton-proton collisions at $\sqrt{s}$ = 13 TeV. Phys. Rev. Lett., 120(22):221801, 2018.
\end{itemize}

The results discussed in Chapter~\ref{Chapter5} were presented by the author at the following
conferences:
\begin{itemize}
\setlength\itemsep{-0.1em}
\item (YSF talk) Search for heavy neutral leptons (sterile
  neutrinos) with the CMS detector, plenary at La Thuile 2018, 25
  Feb-3 Mar 2018 (Italy);
\item Search for Heavy Neutral Leptons with CMS detector, poster at
  EPS-HEP2019, 10-17 Jul 2019 (Belgium);
\item Search for Heavy Neutral Leptons with CMS detector, poster at
  LP2019, 5-10 Aug 2019, University of Toronto (Canada);
\item HNL experimental overview, plenary at EOS winter Solstice
  meeting, 19 Dec 2019, Brussels (Belgium);
\item  Search for heavy neutral leptons at CMS, parallel at
  ICHEP2020, 28 Jul-6 Aug 2020 (Virtual World).
\end{itemize}
The poster presented at EPS-HEP2019 was awarded with ``poster prize of
the EPS High Energy and Particle Physics Division and the EPS HEP 2019
Conference''.\\
 
In the last three PhD years the author was the main developer for the
lon-lived heavy neutral lepton search. The study targets signatures
consisting of one prompt charged lepton and two displaced
charged leptons originating from the secondary vertex of the \hnl
decay.\\
The analysis results have been collected in the Physics Analysis
Summaries, PAS which has been made public. The final paper will be
submitted by the end of the year to Journal
of High Energy Physics (JHEP):
\begin{itemize}
\setlength\itemsep{-0.1em}
\item Search for long-lived heavy neutral leptons with displaced vertices in pp collisions at $\sqrt{s}$ =
13TeV with the CMS detector. Technical report, CERN, Geneva, 2021.
\end{itemize}

The results discussed in Chapter~\ref{Chapter6} were presented by the author at the following
conferences and workshops:

\begin{itemize}
\setlength\itemsep{-0.1em}
\item HNL experimental overview, talks Searching for long-lived
  particles at the LHC: firth workshop of the LHC LLP Community, 27-29
  May 2019, CERN (Switzerland);
\item HNL experimental overview', plenary at EOS winter Solstice
  meeting, 19 Dec 2019, Brussels (Belgium);
\end{itemize}

At the European Physical Society Conference on High Energy Physics
(EPS-HEP) in July 2021, the results were for the first time publicly
presented by CMS Physics Coordinator in ``Highlights from the CMS
Experiment'' talk.

