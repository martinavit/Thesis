% Chapter Template

\chapter{Conclusion and outlook}\label{Chapter7} 

In this thesis searches for heavy neutral leptons in events with either three
charged leptons or displaced vertices are presented. The public
results can be found in References~\cite{Sirunyan:2018mtv} and~\cite{CMS-PAS-EXO-20-009} respectively.\\


The observation of neutrino flavor oscillations was one of the first 
definite experimental indications of the
presence of new physics not described by the SM theory. Thus, 
comprehand the mechanism that discloses the neutrino mass would be a
vital flare into BSM physics. Therefore, it felt crucial to investigate at the LHC
experiments the signatures of different neutrino mass models
to try spotting the mysterious new physics.\\
The two results presented in the dissertation cooperate in the arduous attempt of confronting exotic BSM
models with the experimental data. The aspiration is to find new
physics able to describe the unexplained
physics observations not covered by the SM. 

We decided to focus on right-handed (RH) neutrino or heavy neutral lepton (HNL)
model. The introduction of massive RH
neutrinos provides an answer to the SM problem of the
neutrino masses via a type-I seesaw mechanism.  
In Chapters~\ref{Chapter1} and~\ref{Chapter3} we illustrated
the relevance and the interest for the
ongoing HNL search program, describing first the theory setting 
and then mentioning the various of experiments and results
focusing on HNL.\\
When we hypothesize a new particle as RH neutrinos, $N_{I}$, we
are interested in their properties like the mass $M_I$ of the HNLs and
their mixing parameter, $|V_{\alpha I}|^2$  with the SM neutrino of flavor $\alpha$,
controlled by the Yukawa coupling $F_{\alpha I}$. The values of $|V_{I
  \alpha}|^2$ is unknown and the experimental
sensitivity for both the two analyses presented here is expressed in
terms of the coupling $|V_{\alpha I}|^2$
as a function of $M_I$ for a given flavor $\alpha$. 

Furthermore we introduced a list of direct HNL search results; we give an overview
of the experimental current and past landscape describing the different decay modes and
mass ranges that are targeted by the single measurement.
The strategies in direct HNL searches depend greatly on the mass
which we desire investigating. For $M_{I} > 5$\GeV, \hnl can be
produced uniquely at either LHC or at similar energy colliders, via a few
mechanisms (vector boson fusion, s-channel exchange of virtual
W-bosons or in real gauge boson decays) according to the production
energy and \hnl mass. For $M_{I} < 5$\GeV, we recur to b-factories
or fixed target experiments. \\
Special attention is paid to lepton and hadron collider
searches. The LEP results from DELPHI happen to be the best results at
low mass from collider experiment up to the publication of the results
of this dissertation. The outstanding sensitivity al low mass from
$e^{+}e^{-}$ data was surely a good motivation to invest a quite
important effort to extend as well the low mass sensitivity of
the CMS experiment.\\

Chronologically we have focused first on the
moderate and high mass search to migrate to very low mass search which
necessarily requires the inclusion of displaced scenarios.

During my first PhD year, I worked on the ``search for heavy neutral leptons in events with three charged
 leptons in proton-proton collisions at $\sqrt{s}$ =
 13\TeV''~\cite{Sirunyan:2018mtv}. It was not observed any
 statistically significant excess of signal events over the expected
SM background. At 95\% confidence level upper limits were set on the mixing
parameters \mixpare and \mixparm. The excluded values are in the
ranges between $1.2\times 10^{-5}$ and $1.8$ for masses included
between 1\GeV $< m_\hnl <$ 1.2\TeV. 
These were the first direct limits for HNL masses above 500\GeV and the first
limits obtained at hadron colliders for HNL masses below 40\GeV.
At large HNL masses, the results improved those previously published
by the ATLAS~\cite{Aad_2015} and CMS~\cite{Khachatryan_2015,Sirunyan:2018xiv}
experiments. 

The remaining time of my PhD was dedicated to the ``Search for long-lived heavy neutral leptons with displaced
vertices in pp collisions at $\sqrt{s}$ =
 13\TeV''~\cite{CMS-PAS-EXO-20-009}. The data were collected from the
CMS experiment in years 2016 -- 18 corresponding to an integrated
luminosity of 137\fbinv.
The signature consists of one prompt charged lepton and two displaced
charged leptons in any flavor combination of electrons
and muons. Two interpretations are proposed considering on one hand uniquely the
Dirac HNL nature, on the other hand the Majorana HNL nature. 
No statistically significant deviation from the expected
SM background was observed. At 95\% confidence level limits were set on the mixing
parameters \mixpare and \mixparm.
The excluded values are in the
ranges between $3\times 10^{-7}$ and $1\times 10^{-3}$ for masses included
between 1\GeV $< m_\hnl <$ 15\GeV. \\

The opportunity to work on such complementary analyses
allowed me to gain over the years considerable expertise in HNL
searches with multilepton final states at CMS.\\
Remarkable were the LL-HNL workshops to which I have actively
partecipated. They gave the chance to explore the 
state of art of HNL models and results in particular with LLP
signatures~\cite{Alimena_2020}. Furthermore we identified new open questions and
we emphasize the priority 
of complementary and synergetic searches; these discussions
led to the formulation of new searches and new opportunities
for discovery.\\

In the next section the latest sensitivity
estimations and incoming experimental results are presented. Finally
it is given 
an overview of possible future experiments
and detector upgrades. 


\section{Outlook}

\paragraph{Prompt HNL -->}
\begin{itemize}
\item Heavy Neutrinos with Dynamic Jet Vetoes
\item HNL full run2 
\item HNL prompt with \mixpart
\end{itemize}

\paragraph{Displaced HNL -->}
\begin{itemize}
\item HNL in 2L2J
\item Marco's projection for 3\abinv --> HL-LHC
\end{itemize}
Figure 7 in ~\cite{Alimena_2020}

\paragraph{Detector upgrade -->}
\begin{itemize}
\item CMS timing detectors: tracker,  CMS MTD Endcap Timing Layer,
  ~\cite{CERN-LHCC-2017-027} and muon system 
\item trigger upgrade
\end{itemize}

\paragraph{new experiments -->}
\begin{itemize}
\item mathusla200,
\item ship
\item NA62
\item codex
\end{itemize}






\vspace {5cm}








%%%%%%%%%%%%%%%%%%%%%%




