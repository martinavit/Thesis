% Chapter Template

\chapter{Conclusion and outlook}\label{Chapter7} 

In this thesis searches for heavy neutral leptons in events with either three
charged leptons or displaced vertices are presented. The public
results can be found in References~\cite{Sirunyan:2018mtv} and~\cite{CMS-PAS-EXO-20-009} respectively.\\


The observation of neutrino flavor oscillations was one of the first 
definite experimental indications of the
presence of new physics not described by the SM theory. Thus, 
comprehand the mechanism that discloses the neutrino mass would be a
vital flare into BSM physics. Therefore, it felt crucial to investigate at the LHC
experiments the signatures of different neutrino mass models
to try spotting the mysterious new physics.\\
The two results presented in the dissertation cooperate in the arduous attempt of confronting exotic BSM
models with the experimental data. The aspiration is to find new
physics able to describe the unexplained
physics observations not covered by the SM. 

We decided to focus on right-handed (RH) neutrino or heavy neutral lepton (HNL)
model. The introduction of massive RH
neutrinos provides an answer to the SM problem of the
neutrino masses via a type-I seesaw mechanism.  
In Chapters~\ref{Chapter1} and~\ref{Chapter3} we illustrated
the relevance and the interest for the
ongoing HNL search program, describing first the theory setting 
and then mentioning the various of experiments and results
focusing on HNL.\\
When we hypothesize a new particle as RH neutrinos, $N_{I}$, we
are interested in their properties like the mass $M_I$ of the HNLs and
their mixing parameter, $|V_{\alpha I}|^2$  with the SM neutrino of flavor $\alpha$,
controlled by the Yukawa coupling $F_{\alpha I}$. The values of $|V_{I
  \alpha}|^2$ is unknown and the experimental
sensitivity for both the two analyses presented here is expressed in
terms of the coupling $|V_{\alpha I}|^2$
as a function of $M_I$ for a given flavor $\alpha$. 

Furthermore we introduced a list of direct HNL search results; we give an overview
of the experimental current and past landscape describing the different decay modes and
mass ranges that are targeted by the single measurement.
The strategies in direct HNL searches depend greatly on the mass
which we desire investigating. For $M_{I} > 5$\GeV, \hnl can be
produced uniquely at either LHC or at similar energy colliders, via a few
mechanisms (vector boson fusion, s-channel exchange of virtual
W-bosons or in real gauge boson decays) according to the production
energy and \hnl mass. For $M_{I} < 5$\GeV, we recur to b-factories
or fixed target experiments. \\
Special attention is paid to lepton and hadron collider
searches. The LEP results from DELPHI happen to be the best results at
low mass from collider experiment up to the publication of the results
of this dissertation. The outstanding sensitivity al low mass from
$e^{+}e^{-}$ data was surely a good motivation to invest a quite
important effort to extend as well the low mass sensitivity of
the CMS experiment.\\

Chronologically we have focused first on the
moderate and high mass search to migrate to very low mass search which
necessarily requires the inclusion of displaced scenarios.

During my first PhD year, I worked on the ``search for heavy neutral leptons in events with three charged
 leptons in proton-proton collisions at $sqrt{s}$ =
 13\TeV''~\cite{Sirunyan:2018mtv}. It was not observed any
 statistically significant excess of signal events over the expected
SM background. At 95\% confidence level upper limits were set on the mixing
parameters \mixpare and \mixparm. The excluded values are in the
ranges between $1.2\times 10^{-5}$ and $1.8$ for masses included
between 1\GeV $< m_\hnl <$ 1.2\TeV. 
These were the first direct limits for HNL masses above 500\GeV and the first
limits obtained at hadron colliders for HNL masses below 40\GeV.
At large HNL masses, the results improved those previously published
by the ATLAS~\cite{Aad_2015} and CMS~\cite{Khachatryan_2015,Sirunyan:2018xiv}
experiments. 

The remaining time of my PhD was dedicated to the ``Search for long-lived heavy neutral leptons with displaced
vertices in pp collisions at $sqrt{s}$ =
 13\TeV''~\cite{CMS-PAS-EXO-20-009}. The data were collected from the
CMS experiment in years 2016 -- 18 corresponding to an integrated
luminosity of 137\fbinv.
The signature consists of one prompt charged lepton and two displaced
charged leptons in any flavor combination of electrons
and muons. Two interpretations are proposed considering on one hand uniquely the
Dirac HNL nature, on the other hand the Majorana HNL nature. 
No statistically significant deviation from the expected
SM background was observed. At 95\% confidence level limits were set on the mixing
parameters \mixpare and \mixparm.
The excluded values are in the
ranges between $3\times 10^{-7}$ and $1\times 10^{-3}$ for masses included
between 1\GeV $< m_\hnl <$ 15\GeV. \\

The opportunity to work on such complementary analyses
allowed me to gain over the years considerable expertise in HNL
searches with multilepton final states at CMS.\\
Remarkable were the LL-HNL workshops to which I have actively
partecipated. They were the chance to explore the 
state of art of HNL models and results in particular with LLP
signatures and to identify several open questions. Furthermore these
discussions among the experimental collaborations led to the
appearance of new searches and new opportunities
for discovery in such exotic long-lived signatures.








\vspace {5cm}








%%%%%%%%%%%%%%%%%%%%%%








\emph{Wehave discussed the future 
discovery prospects of a heavy neutrino, within the minimal
setup as well as involving extended gauge/Higgs sectors, 
with a particular emphasis on the energy frontier, in
light of the upcoming run-II phase of the LHC and the 
proposed future colliders at both energy and intensity
frontiers. Abetter picture of the neutrino portal might
 have far-reaching implications for the beyondSM
scenarios in general, including the puzzles of matter–antimatter
 asymmetry and nature ofDMin our Universe.
In this context, we should emphasize the importance 
of complementary and synergetic explorations in the lowenergy
sector at the intensity frontier, as well as cosmological
 observations at the cosmic frontier, a combination
of which is essential to fully unravel the mysteries of the neutrino
world.}


The research program carried out over the first nine years of
  the LHC at CERN has been an
unqualified success. The discovery, in 2012, at a 
center-of-mass energy of 7 and 8 TeV, of a
new particle thus far consistent with the SM Higgs boson has opened
numerous new research
directions and has begun to shed light upon the source of electroweak symmetry breaking,
vector boson scattering amplitudes, and the origin of particle masses. And the establishment
of a wide range of searches for new physics at 7, 8, and 13 TeV with the ATLAS, CMS, and
LHCb detectors—searches thus far consistent with SM expectations—has inspired new ideas
and thinking about the most prominent open issues of physics, such as the nature of DM, the
hierarchy problem, neutrino masses, and the possible existence of supersymmetry.
The overwhelming majority of searches for new physics have been performed under the
assumption that the new particles decay promptly, i.e. very close to the proton–proton IP,
leading to well-defined objects such as jets, leptons, photons, and missing transverse
momentum. Such objects are constructed requiring information from all parts of the detector
including hits close to the IP, calorimeter deposits known to be signatures of particles originating
from the IP, and muons with tracks that traverse the entirety of the detector, moving
out from the IP. However, given the large range of particle lifetimes in the SM—resulting
from general concepts such as approximately preserved symmetries, scale hierarchies, or
phase space restrictions—and the lack of clear, objective motivation related to any particular
model or theory BSM, the lifetime of hypothetical new particles is best treated as a free
parameter. This leads to a wide variety of spectacular signatures in the LHC detectors that
would evade prompt searches, and which have received modest attention compared to
searches for promptly decaying new particles. Because such signatures require significantly
customized analysis techniques and are usually performed by a smaller number of physicists
working on the experimental collaborations, a comprehensive overview and critical review of
BSM LLPs at the LHC has been performed by a community of experimentalists, theorists,
and phenomenologists. This effort ensures that such avenues of the possible discovery of new
physics at the LHC are not overlooked. The results of this initiative have been presented in the
current document.


\section{Outlook}




\subsection{Prompt HNL analysis possible improvements}
\subsection{Displaced HNL analysis possible improvements}
\subsection {HNL searches perspectives and LHC and beyond}


