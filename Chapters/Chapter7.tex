% Chapter Template

\chapter{Conclusions and outlook}\label{Chapter7} 

In this thesis searches for heavy neutral leptons in events with either three
charged leptons or displaced vertices are presented. The public
results can be found in References~\cite{Sirunyan:2018mtv} and~\cite{CMS-PAS-EXO-20-009} respectively.\\


The observation of neutrino flavor oscillations was one of the first 
definite experimental indications of the
presence of new physics not described by the SM theory. Thus, 
comprehend the mechanism that discloses the neutrino mass would be a
vital flare into BSM physics. Therefore, it felt crucial to investigate at the LHC
experiments the signatures of different neutrino mass models
to try spotting the mysterious new physics.\\
The two results presented in the dissertation cooperate in the arduous attempt of confronting exotic BSM
models with the experimental data. The aspiration is to find new
physics able to describe the unexplained
physics observations not covered by the SM. 

We decided to focus on right-handed (RH) neutrino or heavy neutral lepton (HNL)
model. The introduction of massive RH
neutrinos provides an answer to the SM problem of the
neutrino masses via a type-I seesaw mechanism.  
In Chapters~\ref{Chapter1} and~\ref{Chapter3} we illustrated
the relevance and the interest for the
ongoing HNL search program, describing first the theory setting 
and then mentioning the various of experiments and results
focusing on HNL.\\
When we hypothesize a new particle as RH neutrinos, $N_{I}$, we
are interested in their properties like the mass $M_I$ of the HNLs and
their mixing parameter, $|V_{\alpha I}|^2$  with the SM neutrino of flavor $\alpha$,
controlled by the Yukawa coupling $F_{\alpha I}$. The values of $|V_{I
  \alpha}|^2$ is unknown and the experimental
sensitivity for both the two analyses presented here is expressed in
terms of the coupling $|V_{\alpha I}|^2$
as a function of $M_I$ for a given flavor $\alpha$. 

Furthermore we introduced a list of direct HNL search results; we give an overview
of the experimental current and past landscape describing the different decay modes and
mass ranges that are targeted by the single measurement.
The strategies in direct HNL searches depend greatly on the mass
which we desire investigating. For $M_{I} > 5$\GeV, \hnl can be
produced uniquely at either LHC or at similar energy colliders, via a few
mechanisms (vector boson fusion, s-channel exchange of virtual
W-bosons or in real gauge boson decays) according to the production
energy and \hnl mass. For $M_{I} < 5$\GeV, we recur to b-factories
or fixed target experiments. \\
Special attention is paid to lepton and hadron collider
searches. The LEP results from DELPHI happen to be the best results at
low mass from collider experiment up to the publication of the results
of this dissertation. The outstanding sensitivity al low mass from
$e^{+}e^{-}$ data was surely a good motivation to invest a quite
important effort to extend as well the low mass sensitivity of
the CMS experiment.\\

Chronologically we have focused first on the
moderate and high mass search to migrate to very low mass search which
necessarily requires the inclusion of displaced scenarios.

During my first PhD year, I worked on the ``search for heavy neutral leptons in events with three charged
 leptons in proton-proton collisions at $\sqrt{s}$ =
 13\TeV''~\cite{Sirunyan:2018mtv}. It was not observed any
 statistically significant excess of signal events over the expected
SM background. At 95\% confidence level upper limits were set on the mixing
parameters \mixpare and \mixparm. The excluded values are in the
ranges between $1.2\times 10^{-5}$ and $1.8$ for masses included
between 1\GeV $< m_\hnl <$ 1.2\TeV. 
These were the first direct limits for HNL masses above 500\GeV and the first
limits obtained at hadron colliders for HNL masses below 40\GeV.
At large HNL masses, the results improved those previously published
by the ATLAS~\cite{Aad_2015} and CMS~\cite{Khachatryan_2015,Sirunyan:2018xiv}
experiments. 

The remaining time of my PhD was dedicated to the ``Search for long-lived heavy neutral leptons with displaced
vertices in pp collisions at $\sqrt{s}$ =
 13\TeV''~\cite{CMS-PAS-EXO-20-009}. The data were collected from the
CMS experiment in years 2016 -- 18 corresponding to an integrated
luminosity of 137\fbinv.
The signature consists of one prompt charged lepton and two displaced
charged leptons in any flavor combination of electrons
and muons. Two interpretations are proposed considering on one hand
uniquely the HNL Dirac nature, on the other hand the HNL Majorana nature. 
No statistically significant deviation from the expected
SM background was observed. At 95\% confidence level limits were set on the mixing
parameters \mixpare and \mixparm.
The excluded values are in the
ranges between $3\times 10^{-7}$ and $1\times 10^{-3}$ for masses included
between 1\GeV $< m_\hnl <$ 15\GeV. \\

The opportunity to work on such complementary analyses
allowed me to gain over the years considerable expertise in HNL
searches with multi-lepton final states at CMS.\\
Remarkable were the LL-HNL workshops to which I have actively
participated. They gave the chance to explore the 
state-of-the-art of HNL models and results in particular with LLP
signatures~\cite{Alimena_2020}. Furthermore we identified new open questions and
we emphasize the priority 
of complementary and synergetic searches; these discussions
led to the formulation of new searches and new opportunities
for discovery.\\

In the next section the latest sensitivity
estimations and incoming experimental results are presented. Finally
it is given 
an overview of possible future experiments
and detector upgrades. 


\section{Outlook}
Abundantly clear at this point of the thesis, HNLs are one of the most
exciting and best-motivated potential solutions for some of the
outstanding problems of the SM. However, if they happen to exist, their
Majorana/Dirac natures, their masses, and their coupling with the SM
neutrinos are far from obvious and clear. Thus we need to adopt a
comprehensive and vast approach in seeking for HNL probing heavy
neutral leptons with MeV- and TeV-scale masses.\\

For HNLs in the GeV-TeV mass ranges, we enter the domain of the
particle colliders and we are looking for prompt HNLs produced via charged current (CC) Drell-Yan
process, via gluon fusion, and via $\PW \gamma$ fusion. 
In this phase space we find a few new results a several
promising new searches.

Recently at CMS it has been carried out a search for right-handed bosons ($\PW_R$)
and RH neutrinos in the left-right symmetric model extension of the
SM~\cite{CMS-PAS-EXO-20-002} using pp collision data collected at $\sqrt{s}$ =
 13\TeV and corresponding to 137\fbinv integrated luminosity. The
 signature consists of events with two same-flavor light leptons and
 two quarks. For $m_\hnl = m_{\PW_{R}}/2$ ($m_\hnl = 200$\GeV) the
 mass of the $\PW_R$ is excluded at 95\% CL up to 4.7 (4.8) and 5.0 (5.4) TeV for the electron and muon
channel, respectively. 

At the time of the publication of the analysis presented in
Chapter~\ref{Chapter5}, the results were positively received by the community and highly appreciated
for the big effort was put into widening the mass range and
improving the sensitivity previously obtained. The
data were corresponding to an integrated luminosity of only to
35.9\fbinv collected in 2016. Thus it is desirable to perform the same
search on the full available dataset which corresponds to an integrated luminosity of 
137\fbinv. \\
The analysis is being carried out by a Ugent colleague (L. Wezenbeek)
who not only is integrating with the fullRun2 data but he is also improving
some selection strategy aspects to increase the quality of the final
results. In particular, if we recall well, one of the largest
background of~\cite{Sirunyan:2018mtv} is the one coming from nonprompt
leptons. The updated lepton ID makes use of a new machine learning
based lepton identification algorithm which was developed to be
optimal at rejecting nonprompt leptons~\footnote{This machine learning
based lepton identification algorithm was developed in the context of
the $t\overline{t}H$ analysis~\cite{Sirunyan_2018_ttH} and further improved at UGent by Willem
Verbeke and deploied later in the following outstanding
analyses~\cite{CMS:2018sgc_tzq, Sirunyan_2021_higgsmumu}.}. The algorithm exploits the
characteristics of the jet containing the lepton, and puts together all
variables that discriminate between prompt and
nonprompt leptons with a machine learning algorithm. It results in an increase in efficiency
per lepton of 20\% and for the same signal
efficiency a reduction of the
nonprompt background
by factor 10\%. Although the rest of the analysis workflow is going to be very
similar to what published in 2018, the advancements done in reducing the
major background, possible progresses in signal categorization and the
triple amount of data will bring sizable change in the final
sensitivity.

In the same analysis framework and timeline, a new exciting addition
is going to be included. For the first time at CMS, the mixing between
HNL and tau neutrinos is going to be probed. In
Figure~\ref{fig:HNL_bc8_pbc_2} is indisputable the need to dedicate
some effort in searches sensitive to tau coupling, \mixpart. The
ongoing analysis will look for HNL in events with either one light leptons
and two taus or two light leptons and one $\tau$. The inquired mass
range is between 20\GeV and 1\TeV. The main difficulties of these final
states are on one hand trying to reduce the considerable nonprompt tau background and
improve the S/B ratio; on the other hand dealing with quite large tau
\pt thresholds which automatically exclude a part of the phase space
reducing the selection acceptance. \\
Despite the numerous challenges, estimating the limits for \mixpart is
one of the most important piece of the big HNL puzzle and long waited results
by the theory community.  

For future very high HNL mass searches, an interesting suggestion comes from
the work presented in Ref.~\cite{Pascoli_2019}. It is considered the
final state $\hnl \ell + X\rightarrow 3\ell + \ptmiss + X$. The
authors use as major
discriminant techniques a dynamic
jet veto (\ie it depends on the \pt of the highest \pt lepton) and a
requirement on the scalar sum of three leptons \pt. At 14\TeV, it appears that
the proposed analysis strategy can
improve sensitivity by an order of
magnitude for $m_\hnl > 150$\GeV at $\mathcal{L} = 300$\fbinv and
3\abinv. \\

Figures~\ref{fig:HNL_bc6_pbc_2} and~\ref{fig:HNL_bc7_pbc_2} show that
the HNL mass range between 1\GeV and 20\GeV is one the most contend phase
space by so many different present and future experiments. This
results in a quite vigorous competition as well as in a very driving
experimental environment. The analysis presented in Chapter~\ref{Chapter6} sits exactly there. \\
Along these lines, it makes sense to explore almost one by one the
other players with great attention paid to future results and experiments.

In the long-lived HNL scenario, the channel with hadronic \PW decay
can also be probed, \ie $\hnl \ell \rightarrow \ell \ell q q $. The analysis is being carried out by a Ugent
colleague (B. Vermasse). A search for HNL is performed using the pp collision
data at a $\sqrt{s}$ = 13\TeV, collected by the CMS detector during Run
2. The search targets final states with a prompt lepton, a displaced lepton, a
displaced jet and a secondary displaced vertex. The machine
learning technique, called Particle Flow Network is deployed to
improved the separation between signal and background.
Separate PFN trainings
are performed, four for 1\GeV $< m_\hnl < 4$\GeV hypothesis and four for $m_\hnl > 4$\GeV hypothesis.
The output of the PFN classifier is used together with the
displacement of the SV and its mass to categorized the search regions.
The estimation of the SM background appears to be challenging for
reasons similar to the ones explored in this dissertation. The
analyzers are going to estimate the background contribution using a data-driven 
technique and then validate it in data control region. The
final sensitivity will be presented in the 2D \mixpar -- $m_\hnl$
plane as for the three lepton final state analyses.\\

The next groundbreaking occasion will happen with the start of
\textbf{High-Luminosity} LHC,
HL-LHC~\cite{ZurbanoFernandez:2020cco}. The project objective is to
ramp up luminosity by a factor of 10 further the LHC’s design value
(refer to Figure~\ref{fig:lumi})
Starting from the end of 2027, the HL-LHC will accumulate ten times
more data than the LHC throughout its operation. It will help to
detect extremely rare processes and improve the SM precision
measurements. Additionally, to fully profit from the increased quantity of data, CMS and the other
LHC experiments have launched ambitious detector upgrades.

There has been a great number of studies estimating HNL future
projections with $\mathcal{L} = 3$\abinv.\\
In the research in Ref~\cite{Drewes_2020_jan}, the authors inspect the
sensitivity of long-lived HNL searches produced in ATLAS, CMS and LHCb
detectors using expected number of events at an integrated luminosity
of 3\abinv. Furthermore a new analysis strategy is proposed. To extent
the reconstruction acceptance, the reconstruction using the muon system to identify tracks from the
SV is introduced. Even the HNL decaying outside the tracker
volume are considered and included. It was found
(Figure~\ref{fig:marco_sketch_ll}) that the exclusion
reach that can be obtained at ATLAS or CMS during HL-LHC surpasses
DELPHI's limits~\cite{Abreu:1996pa} by three orders of magnitude for
\mixparm and \mixpare.\\

\begin{figure}[h!]
\centering
    \includegraphics[clip,trim=0.cm 0cm 0cm 2cm, width=.38\textwidth]{Figures/c7/marco_god.pdf}
    \includegraphics[clip,trim=0.cm 0cm 0cm 2cm, width=.37\textwidth]{Figures/c7/marco_mu_HL.pdf}
\caption{Left, a scheme illustrating the three main challenges in
  improving the sensitivity in a displaced analisys. Right, exclusion
  limits of ATLAS, CMS (3\abinv) and LHCb (380\fbinv)
 for the HL-LHC for \mixparm. Both plots are
  from Ref.~\cite{Drewes_2020_jan}. }
\label{fig:marco_sketch_ll}
\end{figure}





The programmed ATLAS and CMS detector upgrades for the HL-LHC, at the
same time are going to be golden for the displaced vertices
searches. They will increase coverage in the forward regions, they
will contribute to have better timing and spatial resolutions, and
they will add novel features like track
triggers~\cite{Alimena_2020}.\\
The subsequent brief descriptions and results can be extended further
in
Ref.~\cite{CERN-LHCC-2017-009,CERN-LHCC-2017-011,CERN-LHCC-2017-012,CERN-LHCC-2017-027,CERN-LHCC-2017-013}.

Beforehand we need to place these detector upgrades in the right context
understanding which are the features needed to be met. At HL-LHC the
instantaneous luminosity will be a factor $\sim$5 higher than the LHC
one, with 140-200 pp collisions in each bunch crossing. This harsh
environment will make object reconstruction and particle
identification more difficult due to tracks coming from nearby
vertices. Thus better overview coverage and timing and spatial
resolution become crucial to
separate distinct events from each other. 


The CMS inner tracker will have four additional cylindrical 
layers covering the $|z| < $ 200 mm with the first layer positioned at
28 mm, and up to twelve endcap
disks, which will improve $|\eta|$ coverage going from the current
value of 2.4 to almost 4 (Figure~\ref{fig:MDT_alimena}, left).
Extra modules will be installed in
the CMS outer tracker. Correlating the signals from their sensors, the modules called \pt will be able to identify
the hit pairs (named ‘stubs’) consistent with particles above \pt =
2\GeV. Furthermore these stubs are given as input to the L1 trigger, which enables track-finding
at L1 level already. 

Additional muon chambers will be set up in the endcaps. They will be included in
the muon trigger at L1. The supplementary hits in the
endcap, with improved algorithms, will allow high trigger efficiency 
on displaced muon tracks regardless of the high occupancy environment of the HL-LHC.
 

The CMS MIP timing detector (MTD) will
consist in a barrel and an endcap parts formed by a single layer
module placed between the tracker
and calorimeters covering $|\eta|$ up to $\sim$3.
MTD will improve reconstruction by collecting timing information on
charged particles and by combining tracking with timing. The design will provide a timing resolution
at the start of HL-LHC of $\sim$ 30--40 ps for \pt threshold of
0.7\GeV, the timing resolution in the barrel will degrade to 50--60 ps
at end of HL-LHC~\cite{Marta}. The introduction of a timing detector will help for
the mitigation of pile-up effect from HL-LHC.
\begin{figure}[h]
\centering
    \includegraphics[clip,trim=0.5cm 0cm 0.cm 1.8cm, height =5cm]{Figures/c7/IPrel.pdf}
    \includegraphics[clip,trim=0.5cm 0cm 0.cm 1.3cm, height = 5.3cm]{Figures/c7/vertexrel.pdf}\\
    \includegraphics[clip,trim=0.5cm 1cm 0.cm 3cm, height = 5cm]{Figures/c7/MDT.pdf}
\caption{Top-left, relative resolution of the transverse impact
parameter as a function of the $\eta$ for the Run2 (black dots) and the
HL-LHV (red triangles) CMS tracker, using single isolated muons with
\pt of 10 GeV. Top-right, x and y position resolution of the vertex as a function of the
number of tracks associated to the vertex, with two pile-up scenarios. 
 Bottom, efficiency as a function of the timing
resolution of the MTD for reconstruction of a specific Supersymmetry
model considering events with a separation of
PV and SV by more than 3$\sigma$ in both space and time. All plots are
  from Ref~\cite{Alimena_2020}}
\label{fig:MDT_alimena}
\end{figure}

After exploring CMS outlook for the next 15 years, we have to satisfy
our curiosity and look

 at the forthcoming experimental landscape. We
focus only on the research program for long-lived particle at the LHC,
in the specific cases design to probe HNLs.

The \textbf{MATHUSLA} (MAssive Timing
Hodoscope for Ultra-Stable neutraL pArticles) experiment looks for
long-lived particle with $c\tau \gg$ 100 m. MATHUSLA is a
large, simple surface detector (see Figure~\ref{fig:mathu}) able to reconstruct SVs with proper timing
resolution. The main
part is a tracker array located on top of a 200 m $\times$ 200 m
$\times$ 20 m air-filled decay volume.
The great potential of such experiment comes from the dual geometric and timing
requirements. The trajectories of particles coming from displaced
vertices are fitted to reconstruct SVs. These SVs must satisfy the rigorous
condition that all tracks coincide in time at the SV. These demanding geometric and timing
requirements make sure it is very unlikely for backgrounds to mimic
the long-lived signal.\\
MATHUSLA can excellently probe HNLs in a region of phase space close to that reachable
with present and future fixed-target experiments like SHiP or
DUNE. The derived sensitivity is shown in Figure~\ref{fig:mathu} under
the plausible assumption of zero background. Comparing with CMS and
ATLAS projections which depend critically on the performances of the
HL-LHC detectors, it is foreseen that MATHUSLA results will have rather
better sensitivity than the main LHC detectors for those HNL mass
regimes that suffer from triggering and reconstruction restrictions.
\begin{figure}[h!]
\centering
    \includegraphics[clip,trim=0.3cm 0cm 1.cm 2cm, width=.45\textwidth]{Figures/c7/mathusla1.pdf}
    \includegraphics[clip,trim=0cm 2cm 0.5cm 3cm, width=.54\textwidth]{Figures/c7/mathusla2.pdf}
\caption{Left, simplified MATHUSLA detector
  layout~\cite{Alimena_2020}. Right, projected sensitivity in \mixparm
  - $m_\hnl$ plane for \hnl produced in W/Z decays (brown regions) and in B/D-meson decays (light red region)~\cite{Curtin_2019}.
}
\label{fig:mathu2}
\end{figure}







\paragraph{new experiments -->}
\begin{itemize}
\item ship
\item NA62
\item codex
\end{itemize}


\begin{figure}[h]
\centering
    \includegraphics[clip,trim=0.5cm 3.5cm 1.cm 3cm, width=.68\textwidth]{Figures/c7/projection_alimena.pdf}
\caption{ ~\cite{Alimena_2020}}
\label{fig:HL_alimena}
\end{figure}


\vspace {5cm}








%%%%%%%%%%%%%%%%%%%%%%





