
\begin{extraAbstract}
%\addchaptertocentry{\abstractname} % Add the abstract to the table of contents

Zoektochten naar zware neutrale leptonen (\hnl), die vervallen naar een geladen lepton en een \PW boson, worden gepresenteerd.
De \hnl kunnen zowel Dirac- als Majorana-fermionen zijn. De analyses focussen op onderling complementaire faseruimtes in de massa van het \hnl,
$m_\hnl$ en $V_{\ell \hnl}$, het matrix element dat de koppeling definieert tussen \hnl en neutrino's van het type $\ell$ in het Standaard Model.
De eerste zoektocht focust op gemiddelde en hoge massa's, terwijl de tweede studie naar lagere massa's kijkt, waarbij \hnl langlevend worden.

De eerste zoektocht naar zware neutrale leptonen mikt op vervalscenario's met drie geladen leptonen die allen prompt zijn.
De leptonen kunnen elke combinatie van electronen en muonen zijn. Dit vervalkanaal is nog niet eerder onderzocht aan de LHC
en laat toe om de productie van \hnl te onderzoeken in enkele grootte-ordes van \hnl massa, namelijk van 1\GeV tot 1.2\TeV.
De data, bestaande uit proton-proton botsingen met een totale energie van 13\TeV in het proton-proton massamiddelpunt, is verzameld met de CMS detector aan de LHC.
De totale ge\"integreerde luminositeit van proton-proton botsingen die gebruikt is, bedraagt 35.9\fbinv.
De analyse is onderverdeeld in twee delen, elk deel is geoptimaliseerd om \hnl te herkennen met massa's respectievelijk onder en boven de massa van het \PW boson.
Er zijn geen afwijkingen waargenomen ten opzichte van de voorspellingen van het Standaard Model.
Bovenlimieten worden gezet in functie van \mixparm en \mixpare in een betrouwbaarheidsinterval van 95\%.
Dit zijn de eerste directe limieten voor $m_\hnl >$ 500\GeV en de eerste limieten aan een hadron deeltjesversneller voor $m_\hnl <$ 40\GeV.

De tweede, complementaire zoektocht naar langlevende zware neutrale leptonen focust op vervalprocessen met \'e\'en prompt geladen lepton
en 2 verplaatste geladen leptonen die ontstaan in de secundaire vertex van het verplaatste \hnl verval.
De data wordt gecollecteerd door het CMS experiment aan de LHC gedurende de periode van 2016 tot 2018 en komt overeen met een ge\"integreerde luminositeit
van 137\fbinv aan een energie van 13\TeV in het proton-proton massamiddelpunt. Twee mogelijke interpretaties voor de natuur van het \hnl worden gebruikt, namelijk als Majorana- en Dirac-fermionen.
Geen statistisch significante afwijkingen van voorspellingen op basis van het Standaard Model zijn waargenomen.
In een betrouwbaarheidsinterval van 95\% zijn limieten geplaatst op de koppeling parameters \mixpare en \mixparm. De uitgesloten waardes bestrijken een gebied
tussen $3\times 10^{-7}$ en $1\times 10^{-3}$ voor \hnl massa's tussen 1\GeV $< m_\hnl <$ 15\GeV.
























\end{extraAbstract}
