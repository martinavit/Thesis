
\vspace*{0.2\textheight}

%\noindent\enquote{\itshape That is fundamentally the only courage which is demanded of us: to be brave in the face of the strangest, most singular and most inexplicable things that can befall us. The fact that human beings have been cowardly in this sense has done endless harm to life; the experiences that are called “apparitions”, the whole of the so-called “spirit world”, death, all these things that are so closely related to us, have been so crowded out of life by our daily warding them off, that the senses by which we might apprehend them are stunted. To say nothing of God. But fear of the inexplicable has not only impoverished the existence of the solitary man, it has also circumscribed the relationships between human beings, as it were lifted them up from the river bed of infinite possibilities to a fallow spot on the bank, to which nothing happens.}\bigbreak

%\hfill Rainer Maria Rilke




\enquote{\itshape We have two goals in front of us. One is to explain the story of our universe and why we think it’s true, the big picture as we currently understand it. It’s a fantastic conception. We humans are blobs of organized mud, which through the impersonal workings of nature’s patterns have developed the capacity to contemplate and cherish and engage with the intimidating complexity of the world around us. To understand ourselves, we have to understand the stuff out of which we are made, which means we have to dig deeply into the realm of particles and forces and quantum phenomena, not to mention the spectacular variety of ways that those microscopic pieces can come together to form organized systems capable of feeling and thought.

The other goal is to offer a bit of existential therapy. I want to argue that, though we are part of a universe that runs according to impersonal underlying laws, we nevertheless matter. This isn’t a scientific question—there isn’t data we can collect by doing experiments that could possibly measure the extent to which a life matters. It’s at heart a philosophical problem, one that demands that we discard the way that we’ve been thinking about our lives and their meaning for thousands of years. By the old way of thinking, human life couldn’t possibly be meaningful if we are “just” collections of atoms moving around in accordance with the laws of physics. That’s exactly what we are, but it’s not the only way of thinking about what we are. We are collections of atoms, operating independently of any immaterial spirits or influences, and we are thinking and feeling people who bring meaning into existence by the way we live our lives.

 We are small; the universe is big. It doesn’t come with an instruction manual. We have nevertheless figured out an amazing amount about how things actually work. It’s a different kind of challenge to accept the world for what it is, to face reality with a smile, and to make our lives into something valuable.}\bigbreak



\hfill Sean Carroll, 2016~\cite{citation}