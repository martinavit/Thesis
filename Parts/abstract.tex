
\begin{abstract}
%\addchaptertocentry{\abstractname} % Add the abstract to the table of contents

Searches for heavy neutral leptons, (\hnl) of Majorana and Dirac natures
decaying into a charge lepton and a \PW boson are presented.
The searches target complementary phase spaces in HNL mass ($m_\hnl$)
and in \mixpar, with $V_{\ell \hnl}$ the matrix element
defining the mixing between \hnl and the Standard Model neutrino of flavor $\ell$.
The studies focus first on the
moderate and high mass searches and then migrates to very low mass search with
the introduction of long-lived heavy neutral lepton scenario.\\

The search for heavy neutral leptons focuses on signatures with three
prompt charged leptons in any flavor combination of electrons and
muons. This provides a clean experimental signal for probing the production of the
\hnl in an extended mass range never investigated before at the LHC:
from 1\GeV, and up to 1.2\TeV. The samples of pp collisions at a
center-of-mass energy of 13 TeV were
collected by the CMS experiment at LHC during 2016 data-taking, amounting to
an integrated luminosity of 35.9\fbinv. 
Organized into two parts, the search is optimized for
probing heavy neutral leptons of masses respectively above and below that of the \PW boson. 
The data are found to be consistent with the expected Standard Model
background.  Upper limits at 95\% confidence level on the values of
\mixparm and \mixpare are set. 
These are the first limits obtained at a hadron collider for
 $m_\hnl <$ 40\GeV and the first direct results for $m_\hnl >$
 500\GeV.\\

The search for long-lived heavy neutral leptons targets signatures
consisting of one prompt charged lepton and two displaced
charged leptons originating from the secondary vertex of the \hnl decay.
The data were collected by the
CMS experiment during the Run2 period from 2016 to 2018 and corresponds to an integrated
luminosity of 137\fbinv. Two interpretations are proposed, considering
on the one hand uniquely the \hnl with 
Dirac nature, on the other hand the \hnl with Majorana nature. 
No statistically significant deviation from the expected
SM background is observed. At 95\% confidence level, limits were set on the mixing
parameters \mixpare and \mixparm. The excluded values are in the
range between $3\times 10^{-7}$ and $1\times 10^{-3}$ for masses included
between 1\GeV $< m_\hnl <$ 15\GeV. 



\end{abstract}
