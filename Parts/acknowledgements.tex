
\begin{acknowledgements}
\addchaptertocentry{\acknowledgementname} % Add the acknowledgements
                                % to the table of contents

To be added after private defense.\\

\textsf{``Si stava così bene tutti insieme, così bene, che qualcosa di straordinario doveva pur accadere.
Bastò che a un certo momento lei dicesse: -Ragazzi, avessi un po’ di spazio, come mi piacerebbe
farvi le tagliatelle!- E in quel momento tutti pensammo allo spazio che avrebbero occupato le
tonde braccia di lei muovendosi avanti e indietro con il mattarello sulla sfoglia di pasta, il petto
di lei calando sul gran mucchio di farina e uova che ingombrava il largo tagliere mentre le sue
braccia impastavano impastavano, bianche e unte d’olio fin sopra al gomito; pensammo allo
spazio che avrebbero occupato la farina, e il grano per fare la farina, e i campi per coltivare il
grano, e le montagne da cui scendeva l’acqua per irrigare i campi, e i pascoli per le mandrie di
vitelli che avrebbero dato la carne per il sugo; allo spazio che ci sarebbe voluto perché il Sole
arrivasse con i suoi raggi a maturare il grano; allo spazio perché dalle nubi di gas stellari il Sole si
condensasse e bruciasse; alle quantità di stelle e galassie e ammassi galattici in fuga nello spazio
che ci sarebbero volute per tener sospesa ogni galassia ogni nebula ogni sole ogni pianeta, e nello
stesso tempo del pensarlo questo spazio inarrestabilmente si formava, nello stesso tempo in cui la
signora Ph(i)Nko pronunciava quelle parole: -...le tagliatelle, ve’, ragazzi!- il punto che conteneva
lei e noi tutti s’espandeva in una raggiera di distanze d’anni-luce e secoli-luce e miliardi di millenniluce,
e noi sbattuti ai quattro angoli dell’universo (il signor Pbert Pberd fino a Pavia), e lei dissolta
in non so quale specie d’energia luce calore, lei signora Ph(i)Nko, quella che in mezzo al chiuso
nostro mondo meschino era stata capace d’uno slancio generoso, il primo, “Ragazzi, che tagliatelle
vi farei mangiare!”, un vero slancio d’amore generale, dando inizio nello stesso momento al
concetto di spazio, e allo spazio propriamente detto, e al tempo, e alla gravitazione universale, e
all’universo gravitante, rendendo possibili miliardi di miliardi di soli, e di pianeti, e di campi di
grano, e di signore Ph(i)Nko sparse per i continenti dei pianeti che impastano con le braccia unte
e generose infarinate, e lei da quel momento perduta, e noi a
rimpiangerla.''}

\enquote{\itshape We got along so well all together, so well that something extraordinary was bound 
to happen. It was enough for her to say, at a certain moment: "Oh, if I only had some 
room, how I'd like to make some noodles for you boys!" And in that moment we all 
thought of the space that her round arms would occupy, moving backward and forward 
with the rolling pin over the dough, her bosom leaning over the great mound of flour and 
eggs which cluttered the wide board while her arms kneaded and kneaded, white and 
shiny with oil up to the elbows; we thought of the space that the flour would occupy, and 
the wheat for the flour, and the fields to raise the wheat, and the mountains from which 
the water would flow to irrigate the fields, and the grazing lands for the herds of calves 
that would give their meat for the sauce; of the space it would take for the Sun to arrive 
with its rays, to ripen the wheat; of the space for the Sun to condense from the clouds of 
stellar gases and bum; of the quantities of stars and galaxies and galactic masses in flight 
through space which would be needed to hold suspended every galaxy, every nebula, 
every sun, every planet, and at the same time we thought of it, this space was inevitably 
being formed, at the same time that Mrs. Ph(i)Nk 0 was uttering those words: "... ah, 
what noodles, boys!" the point that contained her and all of us was expanding in a halo of 
distance in light-years and light-centuries and billions of light-millennia, and we were 
being hurled to the four corners of the universe (Mr. Pber* Pber d all the way to Pavia), and 
she, dissolved into I don't know what kind of energy-light-heat, she, Mrs. Ph(i)Nk 0 , she 
who in the midst of our closed, petty world had been capable of a generous impulse, 
"Boys, the noodles I would make for you!," a tme outburst of general love, initiating at 
the same moment the concept of space and, properly speaking, space itself, and time, and 
universal gravitation, and the gravitating universe, making possible billions and billions 
of suns, and of planets, and fields of wheat, and Mrs. Ph(i)Nk 0 s, scattered through the 
continents of the planets, kneading with floury, oil-shiny, generous amis, and she lost at 
that very moment, and we, mourning her loss.  }\bigbreak


\hfill Italo Calvino "Le Cosmicomiche", 1965.
\end{acknowledgements}